\section{Study on constrained PS uncertainty}
\label{sec:decor}
Most constrained NPs are theoretical uncertainties. So only those NPs are shown in this section.

The most significantly constrained NP is ME and PS in leptonic channels. The ME will be replaced by hdamp later.

The fit is done with uncorrelated PS variation in different regions. The pull plots and correlation matrices are shown bellow. The limits extracted using uncorrelated PS variation are also presented in Tab~\ref{tab:tthML_deco_limit} - \ref{tab:xTFW_deco_limit}, which shows that the impact of decorrelating the modelling NP across the different signal regions is minor.

\begin{figure}[H]
\centering
\includegraphics[width=0.49\textwidth]{\FCNCFigures/tthML/Limit/tuH_NuisPar_Theory_decorrreg.pdf}
\includegraphics[width=0.49\textwidth]{\FCNCFigures/tthML/Limit/tuH_NuisPar_Theory_decorrall.pdf}
\caption{ The NP pull in the leptonic channels. The PS is decorrelated in the fit in different regions as shown in the left plot. The PS is further decorrelated in the constrained regions for different contributions in the right plot. }
\label{fig:tuH_NuisPar_decorr}
\end{figure}

\begin{figure}[H]
\centering
\includegraphics[width=0.49\textwidth]{\FCNCFigures/xTFW/Limit/tuH_NuisPar_Theory_decorrreg.pdf}
\includegraphics[width=0.49\textwidth]{\FCNCFigures/xTFW/Limit/tuH_NuisPar_Theory_decorrall.pdf}
\caption{ The NP pull in the hadronic channels. The PS is decorrelated in the fit in different regions as shown in the left plot. The PS is further decorrelated in the constrained regions for different contributions in the right plot. }
\label{fig:tuH_NuisPar_decorr_had}
\end{figure}

\begin{table}
\centering
\begin{tabular}{cccc} \toprule\toprule
 & $t_l\tauhad$-2j & $t_l\tauhad$-1j & $t_h\tlhad$-2j\\\midrule
tcH merged & $4.16^{+1.69}_{-1.16}$ & $3.35^{+1.42}_{-0.94}$ & $5.77^{+2.71}_{-1.61}$\\
tuH merged & $3.41^{+1.39}_{-0.95}$ & $2.69^{+1.14}_{-0.75}$ & $2.99^{+1.26}_{-0.84}$\\
\bottomrule\bottomrule\\
\end{tabular}
\begin{tabular}{cccc} \toprule\toprule
 & $t_h\tlhad$-3j & $t_l\thadhad$ & Combined\\\midrule
tcH merged & $2.82^{+1.31}_{-0.79}$ & $0.59^{+0.26}_{-0.16}$ & $0.56^{+0.24}_{-0.16}$\\
tuH merged & $2.05^{+0.87}_{-0.57}$ & $0.42^{+0.19}_{-0.12}$ & $0.40^{+0.17}_{-0.11}$\\
\bottomrule\bottomrule\\
\end{tabular}
\caption{The expected $95\%$ CL exclusion upper limits on signal ( $\mu=1\to$~BR$(t\to Hq)=0.1\%$ ) with the Asimov in the leptonic channels with Parton Shower uncertainty being decorrelated in different regions in the fit.} 
\label{tab:tthML_deco_limit}
\end{table}



\begin{table}
\centering
\begin{tabular}{cccc} \toprule\toprule
 & $t_h\thadhad$-2j & $t_h\thadhad$-3j & Combined\\\midrule
tcH merged & $2.85^{+1.23}_{-0.80}$ & $1.04^{+0.46}_{-0.29}$ & $0.98^{+0.43}_{-0.27}$\\
tuH merged & $1.66^{+0.73}_{-0.46}$ & $0.88^{+0.39}_{-0.24}$ & $0.78^{+0.34}_{-0.22}$\\
\bottomrule\bottomrule\\
\end{tabular}
\caption{The expected $95\%$ CL exclusion upper limits on signal ( $\mu=1\to$~BR$(t\to Hq)=0.1\%$ ) with the Asimov in the hadronic channels with Parton Shower uncertainty being decorrelated in different regions in the fit.} 
\label{tab:xTFW_deco_limit}
\end{table}




\begin{figure}[H]
\centering
\includegraphics[width=0.49\textwidth]{\FCNCFigures/tthML/Limit/tuH_CorrMatrix_decorrreg.pdf}
\includegraphics[width=0.49\textwidth]{\FCNCFigures/tthML/Limit/tuH_CorrMatrix_decorrall.pdf}
\caption{ The NP correlation matrices in the leptonic channels. The PS is decorrelated in the fit in different regions as shown in the left plot. The PS is further decorrelated in the constrained regions for different contributions in the right plot. }
\label{fig:tuH_CorrMatrix_decorr}
\end{figure}

\begin{figure}[H]
\centering
\includegraphics[width=0.49\textwidth]{\FCNCFigures/xTFW/Limit/tuH_CorrMatrix_decorrreg.pdf}
\includegraphics[width=0.49\textwidth]{\FCNCFigures/xTFW/Limit/tuH_CorrMatrix_decorrall.pdf}
\caption{ The NP correlation matrices in the hadronic channels. The PS is decorrelated in the fit in different regions as shown in the left plot. The PS is further decorrelated in the constrained regions for different contributions in the right plot. }
\label{fig:tuH_CorrMatrix_decorr_had}
\end{figure}



\newpage
