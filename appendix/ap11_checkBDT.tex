\section{Study on BDT}
\label{sec:app_bdt}

A two-fold cross validation procedure is used to maximize the statistics available for training. In this appendix, The correlations distribution of BDT input variables for signal and background is
shown in Figure \ref{fig:correlation_hadhad} and in Figure \ref{fig:correlation_lephad_1}-\ref{fig:correlation_lephad_2}.  The comparison of BDT overtraining check for the testing and from the training samples are also presented in Figure~\ref{fig:comparision_hadhad}-\ref{fig:comparision_lephad}.


\begin{figure}[H]
\centering
\includegraphics[width=0.33\textwidth]{\FCNCFigures/xTFW/BDT/reg2mtau1b2jos_CorrelationMatrixB.pdf}
\includegraphics[width=0.33\textwidth]{\FCNCFigures/xTFW/BDT/reg2mtau1b2jos_CorrelationMatrixS.pdf}
\put(-240, 90){\textbf{(a)}}
\put(-80, 90){\textbf{(a)}}
\\
\includegraphics[width=0.33\textwidth]{\FCNCFigures/xTFW/BDT/reg2mtau1b3jos_CorrelationMatrixB.pdf}
\includegraphics[width=0.33\textwidth]{\FCNCFigures/xTFW/BDT/reg2mtau1b3jos_CorrelationMatrixS.pdf}
\put(-240, 90){\textbf{(b)}}
\put(-80, 90){\textbf{(b)}}
\\
\caption{ Linear correlation coefficients between the BDT input variables for background(left) and signal(right) in the different signal regions: (a) for 2j (b) for 3j  in hadronic channel.}% The Kolmogorov Test values for the training and testing BDT distributions are also indicated.
\label{fig:correlation_hadhad}
\end{figure}


\begin{figure}[H]
\centering
\includegraphics[width=0.30\textwidth]{\FCNCFigures/tthML/BDT/reg1l1tau1b1j_ss_CorrelationMatrixB.pdf}
\includegraphics[width=0.30\textwidth]{\FCNCFigures/tthML/BDT/reg1l1tau1b1j_ss_CorrelationMatrixS.pdf}
\put(-240, 90){\textbf{(a)}}
\put(-80, 90){\textbf{(a)}}
\\
\includegraphics[width=0.30\textwidth]{\FCNCFigures/tthML/BDT/reg1l1tau1b2j_ss_CorrelationMatrixB.pdf}
\includegraphics[width=0.30\textwidth]{\FCNCFigures/tthML/BDT/reg1l1tau1b2j_ss_CorrelationMatrixS.pdf}
\put(-240, 90){\textbf{(b)}}
\put(-80, 90){\textbf{(b)}}
\\
\includegraphics[width=0.30\textwidth]{\FCNCFigures/tthML/BDT/reg1l1tau1b2j_os_CorrelationMatrixB.pdf}
\includegraphics[width=0.30\textwidth]{\FCNCFigures/tthML/BDT/reg1l1tau1b2j_os_CorrelationMatrixS.pdf}
\put(-240, 90){\textbf{(c)}}
\put(-80, 90){\textbf{(c)}}
\\
\caption{ Linear correlation coefficients between the BDT input variables for background(left) and signal(right) in the different signal regions:(a) for $t_l\tauhad$-1j   (b) $t_l\tauhad$-2j for (c) for $t_h\tlhad$-2j in leptonic channel.}% The Kolmogorov Test values for the training and testing BDT distributions are also indicated.
\label{fig:correlation_lephad_1}
\end{figure}

\begin{figure}[H]
\centering
\includegraphics[width=0.30\textwidth]{\FCNCFigures/tthML/BDT/reg1l1tau1b3j_os_CorrelationMatrixB.pdf}
\includegraphics[width=0.30\textwidth]{\FCNCFigures/tthML/BDT/reg1l1tau1b3j_os_CorrelationMatrixS.pdf}
\put(-240, 90){\textbf{(d)}}
\put(-80, 90){\textbf{(d)}}
\\
\includegraphics[width=0.30\textwidth]{\FCNCFigures/tthML/BDT/reg1l2tau1bnj_os_CorrelationMatrixB.pdf}
\includegraphics[width=0.30\textwidth]{\FCNCFigures/tthML/BDT/reg1l2tau1bnj_os_CorrelationMatrixS.pdf}
\put(-240, 90){\textbf{(e)}}
\put(-80, 90){\textbf{(e)}}
\\

\caption{ Linear correlation coefficients between the BDT input variables for background(left) and signal(right) in the different signal regions:(d) for $t_h\tlhad$-3j (e) for $t_l\thadhad$ in leptonic channel.}% The Kolmogorov Test values for the training and testing BDT distributions are also indicated.
\label{fig:correlation_lephad_2}
\end{figure}



\begin{figure}[H]
\centering
\includegraphics[width=0.33\textwidth]{\FCNCFigures/xTFW/BDT/reg2mtau1b2jos_overtraining.pdf}
\put(-40, 60){\textbf{(a)}}
\includegraphics[width=0.33\textwidth]{\FCNCFigures/xTFW/BDT/reg2mtau1b3jos_overtraining.pdf}
\put(-40, 60){\textbf{(b)}}
\\
\caption{ Comparison of the BDT overtraining check distributions for the testing sample and training samples in : (a) for 2j (b) for 3j  in hadronic channel.}
\label{fig:comparision_hadhad}
\end{figure}


\begin{figure}[htb]
\centering
\includegraphics[width=0.33\textwidth]{\FCNCFigures/tthML/BDT/reg1l1tau1b1j_ss_overtraining.pdf}
\put(-40, 60){\textbf{(a)}}
\includegraphics[width=0.33\textwidth]{\FCNCFigures/tthML/BDT/reg1l1tau1b2j_ss_overtraining.pdf}
\put(-40, 60){\textbf{(b)}}
\includegraphics[width=0.33\textwidth]{\FCNCFigures/tthML/BDT/reg1l1tau1b2j_os_overtraining.pdf}
\put(-40, 60){\textbf{(c)}}
\\
\includegraphics[width=0.33\textwidth]{\FCNCFigures/tthML/BDT/reg1l1tau1b3j_os_overtraining.pdf}
\put(-40, 60){\textbf{(d)}}
\includegraphics[width=0.33\textwidth]{\FCNCFigures/tthML/BDT/reg1l2tau1bnj_os_overtraining.pdf}
\put(-40, 60){\textbf{(e)}}

\caption{ Comparison of the BDT overtraining check distributions for the testing sample and training samples in : (a) for $t_l\tauhad$-1j   (b) $t_l\tauhad$-2j  for (c) for $t_h\tlhad$-2j (d) for $t_h\tlhad$-3j (e) for $t_l\thadhad$ in leptonic channel. }
\label{fig:comparision_lephad}
\end{figure}
