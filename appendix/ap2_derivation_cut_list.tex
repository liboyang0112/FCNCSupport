\section{Derivation and framework level cuts}
\label{app:cuts}
A number of event cuts are applied before getting to the signal enhanced regions with the background suppressed. Then the DAOD$\_$HIGG8D1 (DAOD$\_$HIGG4D3) derivation is feed to {\texttt ttHMultiAna} ({\texttt xTauFramework}) to produce n-tuples for leptonic (hadronic) channels, where ttHMultiAna is AnalysisTop \cite{AnalysisTop} based. The list of event-level selection criteria is as follows:

\begin{enumerate}
\item DAOD\_HIGG8D1 (leptonic) and DAOD\_HIGG4D3 (hadronic) derivations are used for this analysis. At the derivation level, the following cuts are applied:
  \begin{itemize}
  \item In DAOD$\_$HIGG8D1, trigger skimming: all electron, muon, tau triggers; Offline skimming: at least 2 light leptons or at least 1 lepton plus 1 tau.
  \item In DAOD$\_$HIGG4D3, no trigger skimming. Offline skimming: 2taus
  \end{itemize}
%wmyao: define ttHML skim cuts
\item At the {\texttt ttHMultiAna} level, skim cuts in \cite{ATL-COM-PHYS-2018-410} are applied , then only the events passing either of the following cut are saved for the n-tuples:
  \begin{itemize}
  \item At least two light leptons passing loose identification criteria with leading lepton $\pt > 15$ GeV and subleading lepton $\pt > 5$ GeV within $|\eta|<2.6$.
  \item At least one light lepton passing loose identification criterua with $\pt >15$ GeV and $|\eta|<2.5$, and at least one hadronic tau. The tau lepton has to pass RNN loose requirement with $\pt >15$ GeV and $|\eta|<2.5$.
  \end{itemize}
\item If the tau candidate overlaps with a muon or fails Loose electron-BDT cut, the event is removed.
\item If the event neither fires single lepton nor dilepton trigger, the event is removed.
\item If the lepton fails tight lepton selection, the event is removed.
\item If the leading or sub-leading tau candidate has $\pt$ less than 25GeV or fails RNN medium criteria, the event is removed.
\item If the event has two same-charged lepton, the electrons are required to pass Loose charge BDT. (Not affecting this analysis.)
\item If the tau is tagged as $b$-jet by 70\% WP, the event is removed.



\item At the {\texttt xTauFramework} level, skim cuts are applied to reduce the ntuple size:
  \begin{itemize}
  \item No leptons.
  \item At least 1 RNN Loose tau.
  \item At least 3 jets with $\pt>30$ GeV, $|\eta|<4.5$ and passing either central or forward JVT cuts, with at least 1 b-tagged. (The jets are further required to have $|\eta|<2.5$ at n-tuple level to be consistent with leptonic channels.)
  \item Taus trigger matched.
  \item LooseBad Event Cleaning.
  \item Leading tau $\pt>40~\GeV$, sub-leading tau $\pt>30~\GeV$, two taus comes from a single vertex.
  \item Leading jet $\pt>70~\GeV$, $|\eta|<3.2$
  \item $\met>15~\GeV$.
  \item In the case of data, GRL cut as defined in Sec. \ref{sec:dataset} is also applied.
  \end{itemize}

\item At least one primary vertex exists in the event. The primary vertex is defined as the vertex that has the largest sum of track $\pt^2$ associated to it, and has at least 4 tracks with $|z_0|<100$ mm.
\end{enumerate}

The corresponding cutflow for the n-tuple level selection and each channel are given in Table \ref{tab:xTFW_cutflow_all} - \ref{tab:tthML_cutflow_reg1l2tau1bnj_os}.
\begin{table}[H]
\caption{The cutflow tables for the preselection in the hadronic channels.}
\label{tab:xTFW_cutflow_all}
\footnotesize
\input{\FCNCTables/cutflow/xTFW/cutflow_all}
\end{table}

\begin{table}[H]
\caption{The cutflow tables in the $t_h\thadhad$-2j signal region.}
\label{tab:cutflow_STHhadhad}
\input{\FCNCTables/cutflow/xTFW/cutflow_reg2mtau1b2jos}
\end{table}

\begin{table}[H]
\caption{The cutflow tables in the $t_h\thadhad$-3j signal region.}
\label{tab:cutflow_TTHhadhad}
\input{\FCNCTables/cutflow/xTFW/cutflow_reg2mtau1b3jos}
\end{table}

\begin{table}[H]
\caption{The cutflow tables for the preselection in the leptonic channels.}
\label{tab:tthML_cutflow_all}
\footnotesize
\input{\FCNCTables/cutflow/tthML/cutflow_all}
\end{table}


\begin{table}[H]
\caption{The cutflow tables in the $t_l\thad$-1j signal region.}
\input{\FCNCTables/cutflow/tthML/cutflow_reg1l1tau1b1j_ss}
\end{table}
\begin{table}[H]
\caption{The cutflow tables in the $t_h\tlhad$-2j signal region.}
\input{\FCNCTables/cutflow/tthML/cutflow_reg1l1tau1b2j_os}
\end{table}
\begin{table}[H]
\caption{The cutflow tables in the $t_l\thad$-2j signal region.}
\input{\FCNCTables/cutflow/tthML/cutflow_reg1l1tau1b2j_ss}
\end{table}
\begin{table}[H]
\caption{The cutflow tables in the $t_h\tlhad$-3j region.}
\input{\FCNCTables/cutflow/tthML/cutflow_reg1l1tau1b3j_os}
\end{table}
\begin{table}[H]
\caption{The cutflow tables in the $t_l\thadhad$ signal region.}
\label{tab:tthML_cutflow_reg1l2tau1bnj_os}
\input{\FCNCTables/cutflow/tthML/cutflow_reg1l2tau1bnj_os}
\end{table}
\newpage