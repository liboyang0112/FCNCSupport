\section{Study on HVV multilepton signal samples}
\label{sec:HVVML}

In this appendix, we presented the FCNC tHq interaction in diboson final state, considering the leptonic decays of W/Z bosons, some fraction of HVV events will contribute to our final state in the lepton+$\thad$ channels as shown in Figures \ref{fig:check_tuH_HWW} - \ref{fig:check_tcH_HWW} while the contamination is
small to the lepton + $\thadhad$ channel.

However, it is not easy to take them into account for signal as the systematics samples are still missing. We decide not to consider the HVV contribution as signal for now
and focus on the $H\rightarrow \tau\tau$ decay only. This would make our limits more conservative if no excess is observed. But in case an excess is observed, the contribution will be accounted to properly measure the signal strength.

Taking the HVV multilepton samples into account, expected limit is also extracted, small difference could been observed from the Table \ref{tab:ml_limit}

%A dedicated BDT training using HVV signal samples is performed independent of the nominal one given that the different kinematics distribution betwwen the Htautau and HVV signal samples. BDT discriminant variable for HVV samples are shown in Figure \ref{fig:check_BDT_HWW}.


\begin{figure}[htb]
  \centering
  \includegraphics[width=0.32\textwidth]{\FCNCFigures/tthML/MLYIELDS/Plots_Plots_thuaMCPY8_H7EG_MLHtautau_MLHVV_h1ltauos1b2jt_taupt.pdf}
  \includegraphics[width=0.32\textwidth]{\FCNCFigures/tthML/MLYIELDS/Plots_Plots_thuaMCPY8_H7EG_MLHtautau_MLHVV_h1ltauos1b3jt_taupt.pdf}
  \includegraphics[width=0.32\textwidth]{\FCNCFigures/tthML/MLYIELDS/Plots_Plots_thuaMCPY8_H7EG_MLHtautau_MLHVV_h1l2tau_Ltauosleadtaupt.pdf}
%\includegraphics[page=6,width=1\textwidth]{\FCNCFigures/tthML/MLYIELDS/tuH_vs_HVVML_1.png}
\\
\includegraphics[width=0.32\textwidth]{\FCNCFigures/tthML/MLYIELDS/Plots_Plots_thuaMCPY8_H7EG_MLHtautau_MLHVV_h1ltauss1b1jt_taupt.pdf}
\includegraphics[width=0.32\textwidth]{\FCNCFigures/tthML/MLYIELDS/Plots_Plots_thuaMCPY8_H7EG_MLHtautau_MLHVV_h1ltauss1b2jt_taupt.pdf}
%\includegraphics[page=6,width=0.9\textwidth]{\FCNCFigures/tthML/MLYIELDS/tuH_vs_HVVML_2.png}
\\
\caption{ Comparison of the Leading tau pt distributions for the FCNC tuH Htautau signal and FCNC tuH HVV signal samples in the leptonic channels(top-left:$t_h\tlhad$-2
j, top-middle:$t_h\tlhad$-3j, top-right:$t_l\thadhad$, bottom-left:$t_l\tauhad$-1j, bottom-right:$t_l\tauhad$-2j). Only statistical uncertainties are being shown. Underflow and overflow bins are included respectively in the first and last bins.}
\label{fig:check_tuH_HWW}
\end{figure}


\begin{figure}[htb]
  \centering
  \includegraphics[width=0.32\textwidth]{\FCNCFigures/tthML/MLYIELDS/Plots_Plots_thcaMCPY8_H7EG_MLHtautau_MLHVV_h1ltauos1b2jt_taupt.pdf}
  \includegraphics[width=0.32\textwidth]{\FCNCFigures/tthML/MLYIELDS/Plots_Plots_thcaMCPY8_H7EG_MLHtautau_MLHVV_h1ltauos1b3jt_taupt.pdf}
  \includegraphics[width=0.32\textwidth]{\FCNCFigures/tthML/MLYIELDS/Plots_Plots_thcaMCPY8_H7EG_MLHtautau_MLHVV_h1l2tau_Ltauosleadtaupt.pdf}
%\includegraphics[page=6,width=1\textwidth]{\FCNCFigures/tthML/MLYIELDS/tcH_vs_HVVML_1.png}
  \\
  \includegraphics[width=0.32\textwidth]{\FCNCFigures/tthML/MLYIELDS/Plots_Plots_thcaMCPY8_H7EG_MLHtautau_MLHVV_h1ltauss1b1jt_taupt.pdf}
  \includegraphics[width=0.32\textwidth]{\FCNCFigures/tthML/MLYIELDS/Plots_Plots_thcaMCPY8_H7EG_MLHtautau_MLHVV_h1ltauss1b2jt_taupt.pdf}
%\includegraphics[page=6,width=0.9\textwidth]{\FCNCFigures/tthML/MLYIELDS/tcH_vs_HVVML_2.png}
\\
\caption{ Comparison of the Leading tau pt distributions for the FCNC tcH Htautau signal and FCNC tcH HVV signal samples in the leptonic channels
  (top-left:$t_h\tlhad$-2j, top-middle:$t_h\tlhad$-3j, top-right:$t_l\thadhad$, bottom-left:$t_l\tauhad$-1j, bottom-right:$t_l\tauhad$-2j). Only statistical uncertainties are being shown. Underflow and overflow bins are included respectively in the first and last bins.}
\label{fig:check_tcH_HWW}
\end{figure}



\begin{table}[htb]
\caption{The expected $95\%$ CL exclusion upper limits on signal ( $\mu=1\to$~BR$(t\to Hq)=0.1\%$ ) with HVV samples into account in the leptonic channel, all uncertainties included.}
\label{tab:ml_limit}
\input{\FCNCTables/limits_merged_signal}
\end{table}


%\begin{figure}[htb]
%\centering
%\includegraphics[page=6,width=0.30\textwidth]{\FCNCFigures/tthML/MLBDT/reg1l1tau1b1j_ss_vetobtagwp70_highmet/BDTG_test.pdf}
%\includegraphics[page=6,width=0.30\textwidth]{\FCNCFigures/tthML/MLBDT/reg1l1tau1b2j_os_vetobtagwp70_highmet/BDTG_test.pdf}
%\includegraphics[page=6,width=0.30\textwidth]{\FCNCFigures/tthML/MLBDT/reg1l1tau1b2j_ss_vetobtagwp70_highmet/BDTG_test.pdf}
%\\
%\includegraphics[page=6,width=0.30\textwidth]{\FCNCFigures/tthML/MLBDT/reg1l1tau1b3j_os_vetobtagwp70_highmet/BDTG_test.pdf}
%\includegraphics[page=6,width=0.30\textwidth]{\FCNCFigures/tthML/MLBDT/reg1l2tau1bnj_os/BDTG_test.pdf}
%\\
%\caption{ Comparison of the BDT discriminant distributions for the background and the FCNC HVV multilepton signal in the leptonic channel. Only statistical uncertainties are being shown. Underflow and overflow bins are included respectively in the first and last bins. Empty data bins here are always blinded based on our strategy.}
%\label{fig:check_BDT_HWW}
%\end{figure}

\clearpage

