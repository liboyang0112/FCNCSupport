\section{Pruning and Correlation matrix}
The pruning algorithm removes systematic uncertainties that have a  $\leqq 1\%$ impact on normalisation or those that cause a negligibly small impact on the shape of the final discriminant (all bins are within $1\%$ of the nominal shape). A summary of the pruned nuisance parameters is provided in this appendix.
The corresponding correlation matrix for the fitted nuisance parameters can also be found here. The corresponding NP pull, constrain and correlations are shown below.

Pruning setting  which is chosen to be 0.1\% is also investigated, the final limit results in hadronic channel is found here in Table~\ref{tab:limit_pruned}. From the table, small influence is observed.


\subsection{Results for Asimov fit}

\begin{figure}[H]
\centering

\includegraphics[width=0.30\textwidth]{\FCNCFigures/xTFW/Limit/tuH_NuisPar.pdf}
\includegraphics[width=0.30\textwidth]{\FCNCFigures/xTFW/Limit/fcnc_uh_NuisPar.pdf}
\includegraphics[width=0.30\textwidth]{\FCNCFigures/xTFW/Limit/fcnc_prod_uh_NuisPar.pdf}
\\
\includegraphics[width=0.30\textwidth]{\FCNCFigures/xTFW/Limit/tcH_NuisPar.pdf}
\includegraphics[width=0.30\textwidth]{\FCNCFigures/xTFW/Limit/fcnc_ch_NuisPar.pdf}
\includegraphics[width=0.30\textwidth]{\FCNCFigures/xTFW/Limit/fcnc_prod_ch_NuisPar.pdf}
\caption{The pull distributions of asimov fit in hadronic channels in terms of tuH merged signal(upper left), $\bar{t}t\to bWuH$ signal (upper middle), $ug\to tH$ signal(upper right), tcH merged signal(lower left), $\bar{t}t\to bWcH$ signal (lower middle) and $cg\to tH$ signal(lower right).}
\label{fig:tcH_NuisPar}
\end{figure}

%\begin{figure}[H]
%\centering
%\includegraphics[width=0.49\textwidth]{\FCNCFigures/xTFW/Limit/fcnc_ch_NuisPar.pdf}
%\includegraphics[width=0.49\textwidth]{\FCNCFigures/xTFW/Limit/fcnc_uh_NuisPar.pdf}
%\caption{ The pull distributions of asimov fit in hadronic channels in terms of $\bar{t}t\to bWcH$ signal (left) and $\bar{t}t%\to bWuH$ signal (right) . }
%\label{fig:fcnc_chuh_NuisPar}
%\end{figure}
%
%\begin{figure}[H]
%\centering
%\includegraphics[width=0.49\textwidth]{\FCNCFigures/xTFW/Limit/fcnc_prod_ch_NuisPar.pdf}
%\includegraphics[width=0.49\textwidth]{\FCNCFigures/xTFW/Limit/fcnc_prod_uh_NuisPar.pdf}
%\caption{ The pull distributions of asimov fit in hadronic channels  in terms of $cg\to tH$ signal (left) and $ug\to tH$ %signal (right) . }
%\label{fig:fcnc_prod_chuh_NuisPar}
%\end{figure}



\begin{figure}[H]
\centering
\includegraphics[width=0.15\textwidth]{\FCNCFigures/tthML/Limit/tuH_NuisPar.pdf}
\includegraphics[width=0.15\textwidth]{\FCNCFigures/tthML/Limit/fcnc_uh_NuisPar.pdf}
\includegraphics[width=0.15\textwidth]{\FCNCFigures/tthML/Limit/fcnc_prod_uh_NuisPar.pdf}
\\
\includegraphics[width=0.15\textwidth]{\FCNCFigures/tthML/Limit/tcH_NuisPar.pdf}
\includegraphics[width=0.15\textwidth]{\FCNCFigures/tthML/Limit/fcnc_ch_NuisPar.pdf}
\includegraphics[width=0.15\textwidth]{\FCNCFigures/tthML/Limit/fcnc_prod_ch_NuisPar.pdf}
\caption{The pull distributions of asimov fit in leptonic channels in terms of tuH merged signal(upper left), $\bar{t}t\to bWuH$ signal (upper middle), $ug\to tH$ signal(upper right), tcH merged signal(lower left), $\bar{t}t\to bWcH$ signal (lower middle) and $cg\to tH$ signal(lower right). }
\label{fig:fcnc_tthml_NuisPar}
\end{figure}



%\begin{figure}[H]
%\centering
%\includegraphics[width=0.30\textwidth]{\FCNCFigures/tthML/Limit/tcH_NuisPar.pdf}
%\caption{ The pull distributions of asimov fit in leptonic channels in terms of tcH merged signal. }
%\label{fig:fcnc_tthml_ch_NuisPar}
%\end{figure}
%
%
%\begin{figure}[H]
%\centering
%\includegraphics[width=0.30\textwidth]{\FCNCFigures/tthML/Limit/fcnc_ch_NuisPar.pdf}
%\includegraphics[width=0.30\textwidth]{\FCNCFigures/tthML/Limit/fcnc_uh_NuisPar.pdf}
%\caption{ The pull distributions of asimov fit in leptonic channels  in terms of $\bar{t}t\to bWcH$ signal (left) and $\bar{t}%t\to bWuH$ signal (right) . }
%\label{fig:fcnc_tthml_chuh_NuisPar}
%\end{figure}
%
%
%\begin{figure}[H]
%\centering
%\includegraphics[width=0.30\textwidth]{\FCNCFigures/tthML/Limit/fcnc_prod_ch_NuisPar.pdf}
%\includegraphics[width=0.30\textwidth]{\FCNCFigures/tthML/Limit/fcnc_prod_uh_NuisPar.pdf}
%\caption{ The pull distributions of asimov fit in leptonic channels  in terms of $cg\to tH$ signal (left) and $ug\to tH$ %signal (right) . }
%\label{fig:fcnc_tthml_prod_chuh_NuisPar}
%\end{figure}
%






%%\begin{figure}[H]
%%\centering
%%\includegraphics[width=0.40\textwidth]{\FCNCFigures/xTFW/Limit/tcH_NuisPar.pdf}
%%\includegraphics[width=0.40\textwidth]{\FCNCFigures/tthML/Limit/tcH_NuisPar.pdf}
%%\caption{ The pull distributions of asimov fit in hadronic channels (left) combined and leptonic channels combined (right) in terms of tcH merged signal.}
%%\label{fig:tcH_NuisPar}
%%\end{figure}
%%
%%\begin{figure}[H]
%%\centering
%%\includegraphics[width=0.49\textwidth]{\FCNCFigures/xTFW/Limit/fcnc_ch_NuisPar.pdf}
%%\includegraphics[width=0.49\textwidth]{\FCNCFigures/tthML/Limit/fcnc_ch_NuisPar.pdf}
%%\caption{ The pull distributions of asimov fit in hadronic channels (left) combined and leptonic channels combined (right) in terms of $\bar{t}t\to bWcH$ signal . }
%%\label{fig:fcnc_ch_NuisPar}
%%\end{figure}
%%
%%\begin{figure}[H]
%%\centering
%%\includegraphics[width=0.49\textwidth]{\FCNCFigures/xTFW/Limit/fcnc_uh_NuisPar.pdf}
%%\includegraphics[width=0.49\textwidth]{\FCNCFigures/tthML/Limit/fcnc_uh_NuisPar.pdf}
%%\caption{ The pull distributions of asimov fit in hadronic channels (left) combined and leptonic channels combined (right) in terms of $\bar{t}t\to bWuH$ signal . }
%%\label{fig:fcnc_uh_NuisPar}
%%\end{figure}
%%
%%\begin{figure}[H]
%%\centering
%%\includegraphics[width=0.49\textwidth]{\FCNCFigures/xTFW/Limit/fcnc_prod_ch_NuisPar.pdf}
%%\includegraphics[width=0.49\textwidth]{\FCNCFigures/tthML/Limit/fcnc_prod_ch_NuisPar.pdf}
%%\caption{ The pull distributions of asimov fit in hadronic channels (left) combined and leptonic channels combined (right) in terms of $cg\to tH$ signal . }
%%\label{fig:fcnc_prod_ch_NuisPar}
%%\end{figure}
%%
%%\begin{figure}[H]
%%\centering
%%\includegraphics[width=0.49\textwidth]{\FCNCFigures/xTFW/Limit/fcnc_prod_uh_NuisPar.pdf}
%%\includegraphics[width=0.49\textwidth]{\FCNCFigures/tthML/Limit/fcnc_prod_uh_NuisPar.pdf}
%%\caption{ The pull distributions of asimov fit in hadronic channels (left) combined and leptonic channels combined (right) in terms of $ug\to tH$ signal . }
%%\label{fig:fcnc_prod_uh_NuisPar}
%%\end{figure}

\begin{figure}[H]
\centering
\includegraphics[width=1\textwidth]{\FCNCFigures/xTFW/Limit/tcH_CorrMatrix.pdf}
\caption{ The asimov fit correlation matrix ($\%$) of different NPs, with a threshold of $1\%$ for hadronic channels combined in terms of tcH merged signal.}
\label{fig:tcH_CorrMatrix_1}
\end{figure}

\begin{figure}[H]
\centering
\includegraphics[width=1\textwidth]{\FCNCFigures/tthML/Limit/tcH_CorrMatrix.pdf}
\caption{ The asimov fit correlation matrix ($\%$) of different NPs, with a threshold of $1\%$ for leptonic channels combined in terms of tcH merged signal.}
\label{fig:tcH_CorrMatrix_2}
\end{figure}


\begin{figure}[H]
\centering
\includegraphics[width=1\textwidth]{\FCNCFigures/xTFW/Limit/fcnc_ch_CorrMatrix.pdf}
\caption{ The asimov fit correlation matrix ($\%$) of different NPs, with a threshold of $1\%$ for hadronic channels combined in terms of $\bar{t}t\to bWcH$ signal . }
\label{fig:fcnc_ch_CorrMatrix_1}
\end{figure}

\begin{figure}[H]
\centering
\includegraphics[width=1\textwidth]{\FCNCFigures/tthML/Limit/fcnc_ch_CorrMatrix.pdf}
\caption{ The asimov fit correlation matrix ($\%$) of different NPs, with a threshold of $1\%$ for  leptonic channels combined in terms of $\bar{t}t\to bWcH$ signal. }
\label{fig:fcnc_ch_CorrMatrix_2}
\end{figure}


\begin{figure}[H]
\centering
\includegraphics[width=1\textwidth]{\FCNCFigures/xTFW/Limit/fcnc_uh_CorrMatrix.pdf}
\caption{ The asimov fit correlation matrix ($\%$) of different NPs, with a threshold of $1\%$ for hadronic channels combined in terms of $\bar{t}t\to bWuH$ signal.}
\label{fig:fcnc_uh_CorrMatrix_1}
\end{figure}



\begin{figure}[H]
\centering
\includegraphics[width=1\textwidth]{\FCNCFigures/tthML/Limit/fcnc_uh_CorrMatrix.pdf}
\caption{ The asimov fit correlation matrix ($\%$) of different NPs, with a threshold of $1\%$ for  leptonic channels combined in terms of $\bar{t}t\to bWuH$ signal.}
\label{fig:fcnc_uh_CorrMatrix_2}
\end{figure}



\begin{figure}[H]
\centering
\includegraphics[width=1\textwidth]{\FCNCFigures/xTFW/Limit/fcnc_prod_ch_CorrMatrix.pdf}
\caption{ The asimov fit correlation matrix ($\%$) of different NPs, with a threshold of $1\%$ for hadronic channels combined in terms of $cg\to tH$ signal.}
\label{fig:fcnc_prod_ch_CorrMatrix_1}
\end{figure}



\begin{figure}[H]
\centering
\includegraphics[width=1\textwidth]{\FCNCFigures/tthML/Limit/fcnc_prod_ch_CorrMatrix.pdf}
\caption{ The asimov fit correlation matrix ($\%$) of different NPs, with a threshold of $1\%$ for leptonic channels combined in terms of $cg\to tH$ signal.}
\label{fig:fcnc_prod_ch_CorrMatrix_2}
\end{figure}

\begin{figure}[H]
\centering
\includegraphics[width=1\textwidth]{\FCNCFigures/xTFW/Limit/fcnc_prod_uh_CorrMatrix.pdf}
\caption{ The asimov fit correlation matrix ($\%$) of different NPs, with a threshold of $1\%$ for hadronic channels combined in terms of $ug\to tH$ signal.}
\label{fig:fcnc_prod_uh_CorrMatrix_1}
\end{figure}

\begin{figure}[H]
\centering
\includegraphics[width=1\textwidth]{\FCNCFigures/tthML/Limit/fcnc_prod_uh_CorrMatrix.pdf}
\caption{ The asimov fit correlation matrix ($\%$) of different NPs, with a threshold of $1\%$ for leptonic channels combined in terms of $ug\to tH$ signal.}
\label{fig:fcnc_prod_uh_CorrMatrix_2}
\end{figure}


\begin{figure}[H]
\centering
\includegraphics[width=0.3\textwidth]{\FCNCFigures/unblinded/xTFW/tuH_Pruning_0.pdf}
\includegraphics[width=0.3\textwidth]{\FCNCFigures/unblinded/xTFW/tuH_Pruning_1.pdf}
\includegraphics[width=0.3\textwidth]{\FCNCFigures/unblinded/xTFW/tuH_Pruning_2.pdf}
\caption{ Summary of the pruned nuisance parameters in the fit to the  Asimov dataset under the S+B hypothesis in hadronic channels in terms of tuH merged signal. The systematic uncertainties that did not survive the pruning are indicated in red. Green corresponds to the uncertainties for which both shape and normalisation is kept, yellow(orange) indicates that shape (normalisation) is kept. Grey indicates that the uncertainty is not present.}
\label{fig:xTFW_pruning_0}
\end{figure}

\begin{figure}[H]
\centering
\includegraphics[width=0.99\textwidth]{\FCNCFigures/tthML/Limit/tuH_Pruning_0.pdf}
\caption{ Summary of the pruned nuisance parameters in the fit to the  Asimov dataset under the S+B hypothesis in leptonic channels part 1 in terms of tuH merged signal. The systematic uncertainties that did not survive the pruning are indicated in red. Green corresponds to the uncertainties for which both shape and normalisation is kept, yellow(orange) indicates that shape (normalisation) is kept. Grey indicates that the uncertainty is not present.}
\label{fig:tthML_pruning_0}
\end{figure}

\begin{figure}[H]
\centering
\includegraphics[width=0.99\textwidth]{\FCNCFigures/tthML/Limit/tuH_Pruning_1.pdf}
\caption{ Summary of the pruned nuisance parameters in the fit to the  Asimov dataset under the S+B hypothesis in leptonic channels part 2 in terms of tuH merged signal. The systematic uncertainties that did not survive the pruning are indicated in red. Green corresponds to the uncertainties for which both shape and normalisation is kept, yellow(orange) indicates that shape (normalisation) is kept. Grey indicates that the uncertainty is not present. }
\label{fig:tthML_pruning_1}
\end{figure}

\begin{figure}[H]
\centering
\includegraphics[width=0.99\textwidth]{\FCNCFigures/tthML/Limit/tuH_Pruning_2.pdf}
\caption{ Summary of the pruned nuisance parameters in the fit to the  Asimov dataset under the S+B hypothesis in leptonic channels part 3 in terms of tuH merged signal. The systematic uncertainties that did not survive the pruning are indicated in red. Green corresponds to the uncertainties for which both shape and normalisation is kept, yellow(orange) indicates that shape (normalisation) is kept. Grey indicates that the uncertainty is not present.}
\label{fig:tthML_pruning_2}
\end{figure}

\begin{figure}[H]
\centering
\includegraphics[width=0.45\textwidth]{\FCNCFigures/xTFW/Limit/tuH_Ranking.pdf}
\includegraphics[width=0.45\textwidth]{\FCNCFigures/tthML/Limit/tuH_Ranking.pdf}
\caption{ Ranking plot from a fit to a signal-plus-background Asimov dataset for (left) hadronic channels and (right)
leptonic channels in terms of tuH merged signal. All two sided NPs ( with up and down) are symmetrized by "TWOSIDED" and one sided NPs (PS, hdamp, MET, fake estimation etc.) are symmetrized by "ONESIDED". The fitted values of the most important nuisance parameters and their impact on the measured
signal strength are shown. The black points, which are plotted according to the bottom horizontal scale, show the deviation
of each of the fitted nuisance parameters,$\hat{\theta}$, from $\theta_{0}$, which is the nominal value of that nuisance parameter, in units of the
pre-fit standard deviation $\Delta\theta$. The black error bars show the post-fit errors, $\sigma_{\theta}$ , which are close to 1 if these data do not
provide any further constraint on that uncertainty. Conversely, a value of $\sigma_{\theta}$ much smaller than 1 indicates a significant
reduction with respect to the original uncertainty. The nuisance parameters are sorted according to the post-fit effect of each on $\mu$ (hashed blue area),
with those with the largest impact at the top. The scale NP is for a variation of normalization and factorization. Only the leading 30 nuisance parameters are shown. The post-fit effect on $\mu$,
shown according to the top horizontal scale, is calculated by fixing the corresponding nuisance parameter at $\hat{\theta}\pm \sigma_{\theta}$ and redoing the fit. The difference between the default and modified $\mu$, $\Delta\mu$, represents the effect of the systematic uncertainty
in question on $\mu$.}
\label{fig:fcnc_rank_asimov}
\end{figure}

\begin{figure}[H]
\centering
\includegraphics[width=0.45\textwidth]{\FCNCFigures/xTFW/Limit/tcH_Ranking.pdf}
\includegraphics[width=0.45\textwidth]{\FCNCFigures/tthML/Limit/tcH_Ranking.pdf}
\caption{ Ranking plot from a fit to a signal-plus-background Asimov dataset for (left) hadronic channels and (right)
leptonic channels in terms of tcH merged signal. All two sided NPs ( with up and down) are symmetrized by "TWOSIDED" and one sided NPs (PS, hdamp, MET, fake estimation etc.) are symmetrized by "ONESIDED". The fitted values of the most important nuisance parameters and their impact on the measured
signal strength are shown. The black points, which are plotted according to the bottom horizontal scale, show the deviation
of each of the fitted nuisance parameters,$\hat{\theta}$, from $\theta_{0}$, which is the nominal value of that nuisance parameter, in units of the
pre-fit standard deviation $\Delta\theta$. The black error bars show the post-fit errors, $\sigma_{\theta}$ , which are close to 1 if these data do not
provide any further constraint on that uncertainty. Conversely, a value of $\sigma_{\theta}$ much smaller than 1 indicates a significant
reduction with respect to the original uncertainty. The nuisance parameters are sorted according to the post-fit effect of each on $\mu$ (hashed blue area),
with those with the largest impact at the top. The scale NP is for a variation of normalization and factorization. Only the leading 30 nuisance parameters are shown. The post-fit effect on $\mu$,
shown according to the top horizontal scale, is calculated by fixing the corresponding nuisance parameter at $\hat{\theta}\pm \sigma_{\theta}$ and redoing the fit. The difference between the default and modified $\mu$, $\Delta\mu$, represents the effect of the systematic uncertainty
in question on $\mu$.}
\label{fig:fcnc_rank_asimov_tcH}
\end{figure}


\begin{figure}[H]
\centering
\includegraphics[width=1\textwidth]{\FCNCFigures/xTFW/Limit/tuH_CorrMatrix.pdf}
\caption{ Correlation matrix corresponding to the fit to the Asimov dataset under the signal-plus-background hypothesis. Only nuisance parameters with a correlation coefficient of at least 1\% with any other parameter are displayed for hadronic channels combined.}
\label{fig:fcnc_correl_asimov_1}
\end{figure}


\begin{figure}[H]
\centering
\includegraphics[width=1\textwidth]{\FCNCFigures/tthML/Limit/tuH_CorrMatrix.pdf}
\caption{ Correlation matrix corresponding to the fit to the Asimov dataset under the signal-plus-background hypothesis. Only nuisance parameters with a correlation coefficient of at least 1\% with any other parameter are displayed for leptonic channels combined.}
\label{fig:fcnc_correl_asimov_2}
\end{figure}

\begin{table}[H]
\caption{The expected $95\%$ CL exclusion upper limits on signal ( $\mu=1\to~\mathcal{B}(t\to Hq)=0.1\%$ ) with the Asimov (B-only) in the hadronic channels with pruning option $=0.1\%$, all uncertainties included.}
\label{tab:limit_pruned}
\input{\FCNCTables/xTFW/limits_prune}
\end{table}

























\newpage
\subsection{Results for unblinding fit}

In this chapter, the fitted results using unblinded data are presented, the fit precedure is same as the previous one but with real detector-collected data instead of Asimov data.


\begin{figure}[H]
\centering
\includegraphics[width=0.30\textwidth]{\FCNCFigures/unblinded/xTFW/tuH_NuisPar.pdf}
\includegraphics[width=0.30\textwidth]{\FCNCFigures/unblinded/xTFW/fcnc_uh_NuisPar.pdf}
\includegraphics[width=0.30\textwidth]{\FCNCFigures/unblinded/xTFW/fcnc_prod_uh_NuisPar.pdf}
\caption{The pull distributions of ublinded fit in hadronic channels in terms of tuH merged signal(left), tuH decay signal(middle) and tuH production signal(right).}
\label{fig:tuH_NuisPar_unblind_had}
\end{figure}

\begin{figure}[H]
\centering
\includegraphics[width=0.30\textwidth]{\FCNCFigures/unblinded/xTFW/tcH_NuisPar.pdf}
\includegraphics[width=0.30\textwidth]{\FCNCFigures/unblinded/xTFW/fcnc_ch_NuisPar.pdf}
\includegraphics[width=0.30\textwidth]{\FCNCFigures/unblinded/xTFW/fcnc_prod_uh_NuisPar.pdf}
\caption{The pull distributions of ublinded fit in hadronic channels in terms of tcH merged signal(left), tcH decay signal(middle) and tcH production signal(right).}
\label{fig:tcH_NuisPar_unblind_had}
\end{figure}




\begin{figure}[H]
\centering
\includegraphics[width=0.30\textwidth]{\FCNCFigures/unblinded/tthML/tuH_NuisPar.pdf}
\includegraphics[width=0.30\textwidth]{\FCNCFigures/unblinded/tthML/fcnc_uh_NuisPar.pdf}
\includegraphics[width=0.30\textwidth]{\FCNCFigures/unblinded/tthML/fcnc_prod_uh_NuisPar.pdf}
\caption{The pull distributions of ublinded fit in leptonic channels in terms of tuH merged signal(left), tuH decay signal(middle) and tuH production signal(right).}
\label{fig:tuH_NuisPar_unblind_lep}
\end{figure}

\begin{figure}[H]
\centering
\includegraphics[width=0.30\textwidth]{\FCNCFigures/unblinded/tthML/tcH_NuisPar.pdf}
\includegraphics[width=0.30\textwidth]{\FCNCFigures/unblinded/tthML/fcnc_ch_NuisPar.pdf}
\includegraphics[width=0.30\textwidth]{\FCNCFigures/unblinded/tthML/fcnc_prod_uh_NuisPar.pdf}
\caption{The pull distributions of ublinded fit in leptonic channels in terms of tcH merged signal(left), tcH decay signal(middle) and tcH production signal(right).}
\label{fig:tcH_NuisPar_unblind_lep}
\end{figure}




\begin{figure}[H]
\centering
\includegraphics[width=0.25\textwidth]{\FCNCFigures/closureCheck/NuisPar_comp_tuH.pdf}
\includegraphics[width=0.25\textwidth]{\FCNCFigures/closureCheck/NuisPar_comp_tcH.pdf}
\caption{The pull distributions of ublinded fit in combined channels(lep+had) in terms of tuH merged signal(left), tcH merged signal(right).}
\label{fig:NuisPar_unblind_lep_had}
\end{figure}



\begin{figure}[htb]
\centering
\includegraphics[width=0.45\textwidth]{\FCNCFigures/unblinded/xTFW/tuH_Ranking.pdf}
\includegraphics[width=0.45\textwidth]{\FCNCFigures/unblinded/tthML/tuH_Ranking.pdf}
\caption{ Ranking plot from a fit to unblind dataset for (left) hadronic channels and (right)
leptonic channels in terms of tuH merged signal. All two sided NPs ( with up and down) are symmetrized by "TWOSIDED" and one sided NPs (PS, hdamp, MET, fake estimation etc.) are symmetrized by "ONESIDED". The fitted values of the most important nuisance parameters and their impact on the measured
signal strength are shown. The black points, which are plotted according to the bottom horizontal scale, show the deviation
of each of the fitted nuisance parameters,$\hat{\theta}$, from $\theta_{0}$, which is the nominal value of that nuisance parameter, in units of the
pre-fit standard deviation $\Delta\theta$. The black error bars show the post-fit errors, $\sigma_{\theta}$ , which are close to 1 if these data do not
provide any further constraint on that uncertainty. Conversely, a value of $\sigma_{\theta}$ much smaller than 1 indicates a significant
reduction with respect to the original uncertainty. The nuisance parameters are sorted according to the post-fit effect of each on $\mu$ (hashed blue area),
with those with the largest impact at the top. The scale NP is for a variation of normalization and factorization. Only the leading 30 nuisance parameters are shown. The post-fit effect on $\mu$,
shown according to the top horizontal scale, is calculated by fixing the corresponding nuisance parameter at $\hat{\theta}\pm \sigma_{\theta}$ and
redoing the fit. The difference between the default and modified $\mu$, $\Delta\mu$, represents the effect of the systematic uncertainty
in question on $\mu$.}
\label{fig:fcnc_rank_data}
\end{figure}

\begin{figure}[htb]
\centering
\includegraphics[width=0.45\textwidth]{\FCNCFigures/unblinded/xTFW/tcH_Ranking.pdf}
\includegraphics[width=0.45\textwidth]{\FCNCFigures/unblinded/tthML/tcH_Ranking.pdf}
\caption{ Ranking plot from a fit to unblind dataset for (left) hadronic channels and (right)
leptonic channels in terms of tcH merged signal. All two sided NPs ( with up and down) are symmetrized by "TWOSIDED" and one sided NPs (PS, hdamp, MET, fake estimation etc.) are symmetrized by "ONESIDED". The fitted values of the most important nuisance parameters and their impact on the measured
signal strength are shown. The black points, which are plotted according to the bottom horizontal scale, show the deviation
of each of the fitted nuisance parameters,$\hat{\theta}$, from $\theta_{0}$, which is the nominal value of that nuisance parameter, in units of the
pre-fit standard deviation $\Delta\theta$. The black error bars show the post-fit errors, $\sigma_{\theta}$ , which are close to 1 if these data do not
provide any further constraint on that uncertainty. Conversely, a value of $\sigma_{\theta}$ much smaller than 1 indicates a significant
reduction with respect to the original uncertainty. The nuisance parameters are sorted according to the post-fit effect of each on $\mu$ (hashed blue area),
with those with the largest impact at the top. The scale NP is for a variation of normalization and factorization. Only the leading 30 nuisance parameters are shown. The post-fit effect on $\mu$,
shown according to the top horizontal scale, is calculated by fixing the corresponding nuisance parameter at $\hat{\theta}\pm \sigma_{\theta}$ and
redoing the fit. The difference between the default and modified $\mu$, $\Delta\mu$, represents the effect of the systematic uncertainty
in question on $\mu$.}
\label{fig:fcnc_rank_data_tcH}
\end{figure}





\begin{figure}[htb]
\centering
\includegraphics[width=1\textwidth]{\FCNCFigures/unblinded/xTFW/tuH_CorrMatrix.pdf}
\caption{ Correlation matrix corresponding to the fit to unblind dataset. Only nuisance parameters with a correlation coefficient of at least 1\% with any other parameter are displayed for hadronic channels combined.}
\label{fig:fcnc_correl_data_1}
\end{figure}


\begin{figure}[htb]
\centering
\includegraphics[width=1\textwidth]{\FCNCFigures/unblinded/tthML/tuH_CorrMatrix.pdf}
\caption{ Correlation matrix corresponding to the fit to unblind dataset. Only nuisance parameters with a correlation coefficient of at least 1\% with any other parameter are displayed for leptonic channels combined.}
\label{fig:fcnc_correl_data_2}
\end{figure}





\begin{figure}[H]
\centering
\includegraphics[width=1\textwidth]{\FCNCFigures/unblinded/xTFW/tcH_CorrMatrix.pdf}
\caption{ The unblined fit correlation matrix ($\%$) of different NPs, with a threshold of $1\%$ for hadronic channels combined in terms of tcH merged signal.}
\label{fig:tcH_CorrMatrix_1}
\end{figure}

\begin{figure}[H]
\centering
\includegraphics[width=1\textwidth]{\FCNCFigures/unblinded/tthML/tcH_CorrMatrix.pdf}
\caption{ The unblined fit correlation matrix ($\%$) of different NPs, with a threshold of $1\%$ for leptonic channels combined in terms of tcH merged signal.}
\label{fig:tcH_CorrMatrix_2}
\end{figure}


\begin{figure}[H]
\centering
\includegraphics[width=1\textwidth]{\FCNCFigures/unblinded/xTFW/fcnc_ch_CorrMatrix.pdf}
\caption{ The unblined fit correlation matrix ($\%$) of different NPs, with a threshold of $1\%$ for hadronic channels combined in terms of $\bar{t}t\to bWcH$ signal . }
\label{fig:fcnc_ch_CorrMatrix_1}
\end{figure}

\begin{figure}[H]
\centering
\includegraphics[width=1\textwidth]{\FCNCFigures/unblinded/tthML/fcnc_ch_CorrMatrix.pdf}
\caption{ The unblined fit correlation matrix ($\%$) of different NPs, with a threshold of $1\%$ for  leptonic channels combined in terms of $\bar{t}t\to bWcH$ signal. }
\label{fig:fcnc_ch_CorrMatrix_2}
\end{figure}


\begin{figure}[H]
\centering
\includegraphics[width=1\textwidth]{\FCNCFigures/unblinded/xTFW/fcnc_uh_CorrMatrix.pdf}
\caption{ The unblined fit correlation matrix ($\%$) of different NPs, with a threshold of $1\%$ for hadronic channels combined in terms of $\bar{t}t\to bWuH$ signal.}
\label{fig:fcnc_uh_CorrMatrix_1}
\end{figure}



\begin{figure}[H]
\centering
\includegraphics[width=1\textwidth]{\FCNCFigures/unblinded/tthML/fcnc_uh_CorrMatrix.pdf}
\caption{ The unblined fit correlation matrix ($\%$) of different NPs, with a threshold of $1\%$ for  leptonic channels combined in terms of $\bar{t}t\to bWuH$ signal.}
\label{fig:fcnc_uh_CorrMatrix_2}
\end{figure}



\begin{figure}[H]
\centering
\includegraphics[width=1\textwidth]{\FCNCFigures/unblinded/xTFW/fcnc_prod_ch_CorrMatrix.pdf}
\caption{ The unblined fit correlation matrix ($\%$) of different NPs, with a threshold of $1\%$ for hadronic channels combined in terms of $cg\to tH$ signal.}
\label{fig:fcnc_prod_ch_CorrMatrix_1}
\end{figure}



\begin{figure}[H]
\centering
\includegraphics[width=1\textwidth]{\FCNCFigures/unblinded/tthML/fcnc_prod_ch_CorrMatrix.pdf}
\caption{ The unblined fit correlation matrix ($\%$) of different NPs, with a threshold of $1\%$ for leptonic channels combined in terms of $cg\to tH$ signal.}
\label{fig:fcnc_prod_ch_CorrMatrix_2}
\end{figure}

\begin{figure}[H]
\centering
\includegraphics[width=1\textwidth]{\FCNCFigures/unblinded/xTFW/fcnc_prod_uh_CorrMatrix.pdf}
\caption{ The unblined fit correlation matrix ($\%$) of different NPs, with a threshold of $1\%$ for hadronic channels combined in terms of $ug\to tH$ signal.}
\label{fig:fcnc_prod_uh_CorrMatrix_1}
\end{figure}

\begin{figure}[H]
\centering
\includegraphics[width=1\textwidth]{\FCNCFigures/unblinded/tthML/fcnc_prod_uh_CorrMatrix.pdf}
\caption{ The unblined fit correlation matrix ($\%$) of different NPs, with a threshold of $1\%$ for leptonic channels combined in terms of $ug\to tH$ signal.}
\label{fig:fcnc_prod_uh_CorrMatrix_2}
\end{figure}











\newpage
