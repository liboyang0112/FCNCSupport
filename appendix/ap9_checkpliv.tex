\section{PLIV cut efficiency study for electron and muon from $\tau$ decays}
\label{sec:CheckPLIV}

The PLIV has been trained to better reject non-prompt leptons and fakes produced in hadron decays. Since tau has a finite lifetime, we have to check the
recommended PLIV scale factors can be applied to the electron and muon from tau decays using $Z\rightarrow \tau\tau\rightarrow e\mu$ enriched control sample
where the electron and muon is required to back-to-back ($\Delta\phi_{e\mu}>2.9$). The data are in good agreement well with the MC prediction with and without
PLIV cuts with the recommended PLIV SF for the invariant mass of electron and muon between 50 and 85 GeV as shon in Figure~\ref{fig:ap9_checkpliv}. The efficiencies of
the PLIV tight cut in data(MC) are $0.741\pm 0.0074$ ($0.757\pm 0.011$) for tau-electron and $0.785\pm 0.010$($0.803\pm 0.011$) for tau-muon, where the uncertainties are
dominated by the fake subtraction in the data using the excess of same-sign events and the MC statistics. The ratio of the efficiencies for tau-lepton between data
and MC (scale factor) is 0.98+-0.02(total), which is consistent with 1 at a sigma level for both electron and muon. 

\begin{figure}[H]
\centering
\includegraphics[width=0.32\textwidth]{\FCNCFigures/tthML/checkPLIV/Plots_hmemuos_ele_inclusive.pdf}
\includegraphics[width=0.32\textwidth]{\FCNCFigures/tthML/checkPLIV/Plots_hmemuos_eleplv_inclusive.pdf}
\includegraphics[width=0.32\textwidth]{\FCNCFigures/tthML/checkPLIV/Plots_hmemuos_muoplv_inclusive.pdf}
\caption{ The $m_{e,\mu}$ distribution from $Z\rightarrow\tau\tau\rightarrow e\mu$ decay without PLIV cut (left); with PLIV cut on electron only (middle); and
  with PLIV cut on muon only (right). The data points are in good agreement with the MC prediction where the PLIV SF applied correctly.}
\label{fig:ap9_checkpliv}
\end{figure}

