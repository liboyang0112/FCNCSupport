\section{BDT Optimisation steps}
\label{sec:BDTOptim}

The calculated optimisation factor defined as $L=S_\mathrm{avg}-xS_\mathrm{diff}$ for each step taken during the BDT nCuts and nTrees optimisation in each signal region are shown below. The $L$ value is in 1\%.

\begin{table}[H]
\caption{The BDT optimisation steps for $l\tauhad$ 1j region.}
\input{\FCNCTables/BDT/Optim_reg1l1tau1b1j_ss.tex}
\end{table}
\begin{table}
\caption{The BDT optimisation steps for $l\tauhad$ 2j region.}
\input{\FCNCTables/BDT/Optim_reg1l1tau1b2j_ss_1.tex}
\end{table}
\begin{table}
\caption{The BDT optimisation steps for $l\tauhad$ 2j region.}
\input{\FCNCTables/BDT/Optim_reg1l1tau1b2j_ss_2.tex}
\end{table}
\begin{table}
\caption{The BDT optimisation steps for STH $\tlhad$ region.}
\input{\FCNCTables/BDT/Optim_reg1l1tau1b2j_os.tex}
\end{table}
\begin{table}
\caption{The BDT optimisation steps for TTH $\tlhad$ region.}
\input{\FCNCTables/BDT/Optim_reg1l1tau1b3j_os_1.tex}
\end{table}
\begin{table}
\caption{The BDT optimisation steps for TTH $\tlhad$ region.}
\input{\FCNCTables/BDT/Optim_reg1l1tau1b3j_os_2.tex}
\end{table}
\begin{table}
\caption{The BDT optimisation steps for $l\thadhad$ region.}
\input{\FCNCTables/BDT/Optim_reg1l2tau1bnj_os.tex}
\end{table}
\begin{table}
\caption{The BDT optimisation steps for STH $\thadhad$ region.}
\input{\FCNCTables/BDT/Optim_reg2mtau1b2jos.tex}
\end{table}
\begin{table}
\caption{The BDT optimisation steps for TTH $\thadhad$ region.}
\input{\FCNCTables/BDT/Optim_reg2mtau1b3jos.tex}
\end{table}