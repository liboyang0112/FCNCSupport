\section{Systematics variation plots}
\label{sec:sys_variation}


This appendix summaries the normalization+shape effects from the most constrained systematic uncertainties which shoule be thoroughly understood across different channels considered in the search. These uncertainties are only displayed for tuH signal. In the figures, "Original" refers to the raw systematic uncertainty whereas "Modified" corresponds to the actual systematic uncertainty used in the statistical analysis, resulting after the application of a smoothing algorithm. Smoothing is set as 40 in the TRexFitter job options.

Two NPs (JET\_Flavor\_Composition, JET\_Pileup\_RhoTopology) in the lepton channels have their impact one-sided on $\mu$ and they are anti-correlated to each other
at 35\%. Work is in progress by doing NLL scan vs $\mu$ by fixing one of NPs to understand the correlation better.

In the final fit, we chose to set the symmetrisation scheme of these two NPs to MAXIMUM instead of TWOSIDED to avoid the one-sided impact, the limit table shown in \ref{tab:tthML_max_vs_twosided} is derived with TWOSIDED setting, the limit in the body (Tab.\ref{tab:tab_tthML_limit}) is presented using MAXIMUM setting.

%Both of the ONESIDE systematics Pileup\_Rho\_Topology and JET\_Flavor\_Composition are checked via the negative log-likelihood scan as shown in Figure~\ref{fig:tthML_n%llscan_1} to investigate the possible cause to one-side effect on the signal strength $\mu$.
\subsection{Validation plots}
In this chapter, we present the major NPs in both of channels. They are:

In leptonic channel:
\begin{itemize}
	\item $\bar{t}t$ PS in Figure \ref{fig:tthML_ttbarPS}.
	\item $\bar{t}t$ ME in Figure \ref{fig:tthML_ttbarME}.
	\item $\bar{t}t$ hdamp in Figure \ref{fig:tthML_ttbarhdamp}.
	\item Scale in Figure \ref{fig:tthML_scale}.
	\item TES\_DETECTOR in Figure \ref{fig:tthML_TES_DETECTOR}.
	\item FSR in Figure \ref{fig:tthML_FSR}.
	\item ABCD electron in Figure \ref{fig:tthML_ABCD_electron}.
	\item Pileup\_Rho\_Topology in Figure \ref{fig:tthML_Pileup_Rho_Topology}.
	\item JET\_Flavor\_Composition in Figure \ref{fig:tthML_JET_Flavor_Composition}.
	\item btag\_B\_0 in Figure \ref{fig:tthML_btag_B_0}.
	\item btag\_B\_1 in Figure \ref{fig:tthML_btag_B_1}.
	\item btag\_B\_3 in Figure \ref{fig:tthML_btag_B_3}.
	\item btag\_B\_37 in Figure \ref{fig:tthML_btag_B_37}.
	\item btag\_C\_0 in Figure \ref{fig:tthML_btag_C_0}.
	\item btag\_C\_1 in Figure \ref{fig:tthML_btag_C_1}.
	\item btag\_C\_5 in Figure \ref{fig:tthML_btag_C_5}.
	\item btag\_C\_7 in Figure \ref{fig:tthML_btag_C_7}.
	\item btag\_C\_8 in Figure \ref{fig:tthML_btag_C_8}.

\end{itemize}

In hadronic channel:
\begin{itemize}
	\item $\bar{t}t$ PS in Figure \ref{fig:xTFW_ttbarPS}.
	\item $\bar{t}t$ ME in Figure \ref{fig:xTFW_ttbarME}.
	\item $\bar{t}t$ hdamp in Figure \ref{fig:xTFW_ttbarhdamp}.
	\item btag\_B\_0 in Figure \ref{fig:xTFW_btag_B_0}.
	\item btag\_B\_1 in Figure \ref{fig:xTFW_btag_B_1}.
	\item btag\_B\_3 in Figure \ref{fig:xTFW_btag_B_3}.
	\item btag\_B\_37 in Figure \ref{fig:xTFW_btag_B_37}.
	\item btag\_C\_0 in Figure \ref{fig:xTFW_btag_C_0}.
	\item btag\_C\_1 in Figure \ref{fig:xTFW_btag_C_1}.
	\item btag\_C\_5 in Figure \ref{fig:xTFW_btag_C_5}.
	\item btag\_C\_7 in Figure \ref{fig:xTFW_btag_C_7}.
	\item btag\_C\_8 in Figure \ref{fig:xTFW_btag_C_8}.
\end{itemize}

\begin{figure}[H]
\centering
\includegraphics[width=0.32\textwidth]{\FCNCFigures/tthML/trexfitter/ttbarPS/reg1l1tau1b1j_ss_Combined_ttbarPS.pdf}
\includegraphics[width=0.32\textwidth]{\FCNCFigures/tthML/trexfitter/ttbarPS/reg1l1tau1b2j_os_Combined_ttbarPS.pdf}
\includegraphics[width=0.32\textwidth]{\FCNCFigures/tthML/trexfitter/ttbarPS/reg1l1tau1b2j_ss_Combined_ttbarPS.pdf}
\includegraphics[width=0.32\textwidth]{\FCNCFigures/tthML/trexfitter/ttbarPS/reg1l1tau1b3j_os_Combined_ttbarPS.pdf}
\includegraphics[width=0.32\textwidth]{\FCNCFigures/tthML/trexfitter/ttbarPS/reg1l2tau1bnj_os_Combined_ttbarPS.pdf}
\caption{The background change caused by $\bar{t}t$ PS variation in leptonic channels.}
\label{fig:tthML_ttbarPS}
\end{figure}

\begin{figure}[H]
\centering
\includegraphics[width=0.32\textwidth]{\FCNCFigures/tthML/trexfitter/ttbarME/reg1l1tau1b1j_ss_Combined_ttbarME.pdf}
\includegraphics[width=0.32\textwidth]{\FCNCFigures/tthML/trexfitter/ttbarME/reg1l1tau1b2j_os_Combined_ttbarME.pdf}
\includegraphics[width=0.32\textwidth]{\FCNCFigures/tthML/trexfitter/ttbarME/reg1l1tau1b2j_ss_Combined_ttbarME.pdf}
\includegraphics[width=0.32\textwidth]{\FCNCFigures/tthML/trexfitter/ttbarME/reg1l1tau1b3j_os_Combined_ttbarME.pdf}
\includegraphics[width=0.32\textwidth]{\FCNCFigures/tthML/trexfitter/ttbarME/reg1l2tau1bnj_os_Combined_ttbarME.pdf}
\caption{The background change caused by $\bar{t}t$ ME variation in leptonic channels.}
\label{fig:tthML_ttbarME}
\end{figure}

\begin{figure}[H]
\centering
\includegraphics[width=0.32\textwidth]{\FCNCFigures/tthML/trexfitter/ttbarhdamp/reg1l1tau1b1j_ss_Combined_ttbarhdamp.pdf}
\includegraphics[width=0.32\textwidth]{\FCNCFigures/tthML/trexfitter/ttbarhdamp/reg1l1tau1b2j_os_Combined_ttbarhdamp.pdf}
\includegraphics[width=0.32\textwidth]{\FCNCFigures/tthML/trexfitter/ttbarhdamp/reg1l1tau1b2j_ss_Combined_ttbarhdamp.pdf}
\includegraphics[width=0.32\textwidth]{\FCNCFigures/tthML/trexfitter/ttbarhdamp/reg1l1tau1b3j_os_Combined_ttbarhdamp.pdf}
\includegraphics[width=0.32\textwidth]{\FCNCFigures/tthML/trexfitter/ttbarhdamp/reg1l2tau1bnj_os_Combined_ttbarhdamp.pdf}
\caption{The background change caused by $\bar{t}t$ hdamp variation in leptonic channels.}
\label{fig:tthML_ttbarhdamp}
\end{figure}


\begin{figure}[H]
\centering
\includegraphics[width=0.32\textwidth]{\FCNCFigures/tthML/trexfitter/scale/reg1l1tau1b1j_ss_Combined_scale.pdf}
\includegraphics[width=0.32\textwidth]{\FCNCFigures/tthML/trexfitter/scale/reg1l1tau1b2j_os_Combined_scale.pdf}
\includegraphics[width=0.32\textwidth]{\FCNCFigures/tthML/trexfitter/scale/reg1l1tau1b2j_ss_Combined_scale.pdf}
\includegraphics[width=0.32\textwidth]{\FCNCFigures/tthML/trexfitter/scale/reg1l1tau1b3j_os_Combined_scale.pdf}
\includegraphics[width=0.32\textwidth]{\FCNCFigures/tthML/trexfitter/scale/reg1l2tau1bnj_os_Combined_scale.pdf}
\caption{The background change caused by scale variation in leptonic channels.}
\label{fig:tthML_scale}
\end{figure}

\begin{figure}[H]
\centering
\includegraphics[width=0.32\textwidth]{\FCNCFigures/tthML/trexfitter/TES_DETECTOR/reg1l1tau1b1j_ss_Combined_TES_DETECTOR.pdf}
\includegraphics[width=0.32\textwidth]{\FCNCFigures/tthML/trexfitter/TES_DETECTOR/reg1l1tau1b2j_os_Combined_TES_DETECTOR.pdf}
\includegraphics[width=0.32\textwidth]{\FCNCFigures/tthML/trexfitter/TES_DETECTOR/reg1l1tau1b2j_ss_Combined_TES_DETECTOR.pdf}
\includegraphics[width=0.32\textwidth]{\FCNCFigures/tthML/trexfitter/TES_DETECTOR/reg1l1tau1b3j_os_Combined_TES_DETECTOR.pdf}
\includegraphics[width=0.32\textwidth]{\FCNCFigures/tthML/trexfitter/TES_DETECTOR/reg1l2tau1bnj_os_Combined_TES_DETECTOR.pdf}
\caption{The background change caused by TES\_DETECTOR variation in leptonic channels.}
\label{fig:tthML_TES_DETECTOR}
\end{figure}

\begin{figure}[H]
\centering
\includegraphics[width=0.32\textwidth]{\FCNCFigures/tthML/trexfitter/FSR/reg1l1tau1b1j_ss_Combined_FSR.pdf}
\includegraphics[width=0.32\textwidth]{\FCNCFigures/tthML/trexfitter/FSR/reg1l1tau1b2j_os_Combined_FSR.pdf}
\includegraphics[width=0.32\textwidth]{\FCNCFigures/tthML/trexfitter/FSR/reg1l1tau1b2j_ss_Combined_FSR.pdf}
\includegraphics[width=0.32\textwidth]{\FCNCFigures/tthML/trexfitter/FSR/reg1l1tau1b3j_os_Combined_FSR.pdf}
\includegraphics[width=0.32\textwidth]{\FCNCFigures/tthML/trexfitter/FSR/reg1l2tau1bnj_os_Combined_FSR.pdf}
\caption{The background change caused by $\bar{t}t$ FSR variation in leptonic channels.}
\label{fig:tthML_FSR}
\end{figure}

\begin{figure}[H]
\centering
\includegraphics[width=0.32\textwidth]{\FCNCFigures/tthML/trexfitter/ABCD_electron/reg1l1tau1b1j_ss_Combined_ABCD_electron.pdf}
\includegraphics[width=0.32\textwidth]{\FCNCFigures/tthML/trexfitter/ABCD_electron/reg1l1tau1b2j_os_Combined_ABCD_electron.pdf}
\includegraphics[width=0.32\textwidth]{\FCNCFigures/tthML/trexfitter/ABCD_electron/reg1l1tau1b2j_ss_Combined_ABCD_electron.pdf}
\includegraphics[width=0.32\textwidth]{\FCNCFigures/tthML/trexfitter/ABCD_electron/reg1l1tau1b3j_os_Combined_ABCD_electron.pdf}
\caption{The background change caused by ABCD electron variation in leptonic channels.}
\label{fig:tthML_ABCD_electron}
\end{figure}

\begin{figure}[H]
\centering
\includegraphics[width=0.32\textwidth]{\FCNCFigures/tthML/trexfitter/JET_Pileup_RhoTopology/reg1l1tau1b1j_ss_Combined_JET_Pileup_RhoTopology.pdf}
\includegraphics[width=0.32\textwidth]{\FCNCFigures/tthML/trexfitter/JET_Pileup_RhoTopology/reg1l1tau1b2j_os_Combined_JET_Pileup_RhoTopology.pdf}
\includegraphics[width=0.32\textwidth]{\FCNCFigures/tthML/trexfitter/JET_Pileup_RhoTopology/reg1l1tau1b2j_ss_Combined_JET_Pileup_RhoTopology.pdf}
\includegraphics[width=0.32\textwidth]{\FCNCFigures/tthML/trexfitter/JET_Pileup_RhoTopology/reg1l1tau1b3j_os_Combined_JET_Pileup_RhoTopology.pdf}
\includegraphics[width=0.32\textwidth]{\FCNCFigures/tthML/trexfitter/JET_Pileup_RhoTopology/reg1l2tau1bnj_os_Combined_JET_Pileup_RhoTopology.pdf}
\caption{The background change caused by JET\_Pileup\_RhoTopology in leptonic channels.}
\label{fig:tthML_Pileup_Rho_Topology}
\end{figure}

\begin{figure}[H]
\centering
\includegraphics[width=0.32\textwidth]{\FCNCFigures/tthML/trexfitter/JET_Flavor_Composition/reg1l1tau1b1j_ss_Combined_JET_Flavor_Composition.pdf}
\includegraphics[width=0.32\textwidth]{\FCNCFigures/tthML/trexfitter/JET_Flavor_Composition/reg1l1tau1b2j_os_Combined_JET_Flavor_Composition.pdf}
\includegraphics[width=0.32\textwidth]{\FCNCFigures/tthML/trexfitter/JET_Flavor_Composition/reg1l1tau1b2j_ss_Combined_JET_Flavor_Composition.pdf}
\includegraphics[width=0.32\textwidth]{\FCNCFigures/tthML/trexfitter/JET_Flavor_Composition/reg1l1tau1b3j_os_Combined_JET_Flavor_Composition.pdf}
\includegraphics[width=0.32\textwidth]{\FCNCFigures/tthML/trexfitter/JET_Flavor_Composition/reg1l2tau1bnj_os_Combined_JET_Flavor_Composition.pdf}
\caption{The background change caused by JET\_Flavor\_Composition in leptonic channels.}
\label{fig:tthML_JET_Flavor_Composition}
\end{figure}






\begin{figure}[H]
\centering
\includegraphics[width=0.32\textwidth]{\FCNCFigures/tthML/trexfitter/btag_B_0/reg1l1tau1b1j_ss_Combined_btag_B_0.pdf}
\includegraphics[width=0.32\textwidth]{\FCNCFigures/tthML/trexfitter/btag_B_0/reg1l1tau1b2j_os_Combined_btag_B_0.pdf}
\includegraphics[width=0.32\textwidth]{\FCNCFigures/tthML/trexfitter/btag_B_0/reg1l1tau1b2j_ss_Combined_btag_B_0.pdf}
\includegraphics[width=0.32\textwidth]{\FCNCFigures/tthML/trexfitter/btag_B_0/reg1l1tau1b3j_os_Combined_btag_B_0.pdf}
\includegraphics[width=0.32\textwidth]{\FCNCFigures/tthML/trexfitter/btag_B_0/reg1l2tau1bnj_os_Combined_btag_B_0.pdf}
\caption{The background change caused by btag\_B\_0 in leptonic channels.}
\label{fig:tthML_btag_B_0}
\end{figure}

\begin{figure}[H]
\centering
\includegraphics[width=0.32\textwidth]{\FCNCFigures/tthML/trexfitter/btag_B_1/reg1l1tau1b1j_ss_Combined_btag_B_1.pdf}
\includegraphics[width=0.32\textwidth]{\FCNCFigures/tthML/trexfitter/btag_B_1/reg1l1tau1b2j_os_Combined_btag_B_1.pdf}
\includegraphics[width=0.32\textwidth]{\FCNCFigures/tthML/trexfitter/btag_B_1/reg1l1tau1b2j_ss_Combined_btag_B_1.pdf}
\includegraphics[width=0.32\textwidth]{\FCNCFigures/tthML/trexfitter/btag_B_1/reg1l1tau1b3j_os_Combined_btag_B_1.pdf}
\includegraphics[width=0.32\textwidth]{\FCNCFigures/tthML/trexfitter/btag_B_1/reg1l2tau1bnj_os_Combined_btag_B_1.pdf}
\caption{The background change caused by btag\_B\_1 in leptonic channels.}
\label{fig:tthML_btag_B_1}
\end{figure}

\begin{figure}[H]
\centering
\includegraphics[width=0.32\textwidth]{\FCNCFigures/tthML/trexfitter/btag_B_3/reg1l1tau1b1j_ss_Combined_btag_B_3.pdf}
\includegraphics[width=0.32\textwidth]{\FCNCFigures/tthML/trexfitter/btag_B_3/reg1l1tau1b2j_os_Combined_btag_B_3.pdf}
\includegraphics[width=0.32\textwidth]{\FCNCFigures/tthML/trexfitter/btag_B_3/reg1l1tau1b2j_ss_Combined_btag_B_3.pdf}
\includegraphics[width=0.32\textwidth]{\FCNCFigures/tthML/trexfitter/btag_B_3/reg1l1tau1b3j_os_Combined_btag_B_3.pdf}
\includegraphics[width=0.32\textwidth]{\FCNCFigures/tthML/trexfitter/btag_B_3/reg1l2tau1bnj_os_Combined_btag_B_3.pdf}
\caption{The background change caused by btag\_B\_3 in leptonic channels.}
\label{fig:tthML_btag_B_3}
\end{figure}

\begin{figure}[H]
\centering
\includegraphics[width=0.32\textwidth]{\FCNCFigures/tthML/trexfitter/btag_B_37/reg1l1tau1b1j_ss_Combined_btag_B_37.pdf}
\includegraphics[width=0.32\textwidth]{\FCNCFigures/tthML/trexfitter/btag_B_37/reg1l1tau1b2j_os_Combined_btag_B_37.pdf}
\includegraphics[width=0.32\textwidth]{\FCNCFigures/tthML/trexfitter/btag_B_37/reg1l1tau1b2j_ss_Combined_btag_B_37.pdf}
\includegraphics[width=0.32\textwidth]{\FCNCFigures/tthML/trexfitter/btag_B_37/reg1l1tau1b3j_os_Combined_btag_B_37.pdf}
\includegraphics[width=0.32\textwidth]{\FCNCFigures/tthML/trexfitter/btag_B_37/reg1l2tau1bnj_os_Combined_btag_B_37.pdf}
\caption{The background change caused by btag\_B\_37 in leptonic channels.}
\label{fig:tthML_btag_B_37}
\end{figure}

\begin{figure}[H]
\centering
\includegraphics[width=0.32\textwidth]{\FCNCFigures/tthML/trexfitter/btag_C_0/reg1l1tau1b1j_ss_Combined_btag_C_0.pdf}
\includegraphics[width=0.32\textwidth]{\FCNCFigures/tthML/trexfitter/btag_C_0/reg1l1tau1b2j_os_Combined_btag_C_0.pdf}
\includegraphics[width=0.32\textwidth]{\FCNCFigures/tthML/trexfitter/btag_C_0/reg1l1tau1b2j_ss_Combined_btag_C_0.pdf}
\includegraphics[width=0.32\textwidth]{\FCNCFigures/tthML/trexfitter/btag_C_0/reg1l1tau1b3j_os_Combined_btag_C_0.pdf}
\includegraphics[width=0.32\textwidth]{\FCNCFigures/tthML/trexfitter/btag_C_0/reg1l2tau1bnj_os_Combined_btag_C_0.pdf}
\caption{The background change caused by btag\_C\_0 in leptonic channels.}
\label{fig:tthML_btag_C_0}
\end{figure}

\begin{figure}[H]
\centering
\includegraphics[width=0.32\textwidth]{\FCNCFigures/tthML/trexfitter/btag_C_1/reg1l1tau1b1j_ss_Combined_btag_C_1.pdf}
\includegraphics[width=0.32\textwidth]{\FCNCFigures/tthML/trexfitter/btag_C_1/reg1l1tau1b2j_os_Combined_btag_C_1.pdf}
\includegraphics[width=0.32\textwidth]{\FCNCFigures/tthML/trexfitter/btag_C_1/reg1l1tau1b2j_ss_Combined_btag_C_1.pdf}
\includegraphics[width=0.32\textwidth]{\FCNCFigures/tthML/trexfitter/btag_C_1/reg1l1tau1b3j_os_Combined_btag_C_1.pdf}
\includegraphics[width=0.32\textwidth]{\FCNCFigures/tthML/trexfitter/btag_C_1/reg1l2tau1bnj_os_Combined_btag_C_1.pdf}
\caption{The background change caused by btag\_C\_1 in leptonic channels.}
\label{fig:tthML_btag_C_1}
\end{figure}

\begin{figure}[H]
\centering
\includegraphics[width=0.32\textwidth]{\FCNCFigures/tthML/trexfitter/btag_C_5/reg1l1tau1b1j_ss_Combined_btag_C_5.pdf}
\includegraphics[width=0.32\textwidth]{\FCNCFigures/tthML/trexfitter/btag_C_5/reg1l1tau1b2j_os_Combined_btag_C_5.pdf}
\includegraphics[width=0.32\textwidth]{\FCNCFigures/tthML/trexfitter/btag_C_5/reg1l1tau1b2j_ss_Combined_btag_C_5.pdf}
\includegraphics[width=0.32\textwidth]{\FCNCFigures/tthML/trexfitter/btag_C_5/reg1l1tau1b3j_os_Combined_btag_C_5.pdf}
\includegraphics[width=0.32\textwidth]{\FCNCFigures/tthML/trexfitter/btag_C_5/reg1l2tau1bnj_os_Combined_btag_C_5.pdf}
\caption{The background change caused by btag\_C\_5 in leptonic channels.}
\label{fig:tthML_btag_C_5}
\end{figure}

\begin{figure}[H]
\centering
\includegraphics[width=0.32\textwidth]{\FCNCFigures/tthML/trexfitter/btag_C_7/reg1l1tau1b1j_ss_Combined_btag_C_7.pdf}
\includegraphics[width=0.32\textwidth]{\FCNCFigures/tthML/trexfitter/btag_C_7/reg1l1tau1b2j_os_Combined_btag_C_7.pdf}
\includegraphics[width=0.32\textwidth]{\FCNCFigures/tthML/trexfitter/btag_C_7/reg1l1tau1b2j_ss_Combined_btag_C_7.pdf}
\includegraphics[width=0.32\textwidth]{\FCNCFigures/tthML/trexfitter/btag_C_7/reg1l1tau1b3j_os_Combined_btag_C_7.pdf}
\includegraphics[width=0.32\textwidth]{\FCNCFigures/tthML/trexfitter/btag_C_7/reg1l2tau1bnj_os_Combined_btag_C_7.pdf}
\caption{The background change caused by btag\_C\_7 in leptonic channels.}
\label{fig:tthML_btag_C_7}
\end{figure}

\begin{figure}[H]
\centering
\includegraphics[width=0.32\textwidth]{\FCNCFigures/tthML/trexfitter/btag_C_8/reg1l1tau1b1j_ss_Combined_btag_C_8.pdf}
\includegraphics[width=0.32\textwidth]{\FCNCFigures/tthML/trexfitter/btag_C_8/reg1l1tau1b2j_os_Combined_btag_C_8.pdf}
\includegraphics[width=0.32\textwidth]{\FCNCFigures/tthML/trexfitter/btag_C_8/reg1l1tau1b2j_ss_Combined_btag_C_8.pdf}
\includegraphics[width=0.32\textwidth]{\FCNCFigures/tthML/trexfitter/btag_C_8/reg1l1tau1b3j_os_Combined_btag_C_8.pdf}
\includegraphics[width=0.32\textwidth]{\FCNCFigures/tthML/trexfitter/btag_C_8/reg1l2tau1bnj_os_Combined_btag_C_8.pdf}
\caption{The background change caused by btag\_C\_8 in leptonic channels.}
\label{fig:tthML_btag_C_8}
\end{figure}



%\begin{figure}[H]
%\centering
%\includegraphics[width=0.32\textwidth]{\FCNCFigures/tthML/Limit/tuH_NLLscan_JET_Flavor_Composition.pdf}
%\includegraphics[width=0.32\textwidth]{\FCNCFigures/tthML/Limit/tuH_NLLscan_JET_Pileup_RhoTopology.pdf}
%\\
%\includegraphics[width=0.32\textwidth]{\FCNCFigures/tthML/Limit/tcH_NLLscan_JET_Flavor_Composition.pdf}
%\includegraphics[width=0.32\textwidth]{\FCNCFigures/tthML/Limit/tcH_NLLscan_JET_Pileup_RhoTopology.pdf}
%\\
%\includegraphics[width=0.32\textwidth]{\FCNCFigures/tthML/Limit/fcnc_uh_NLLscan_JET_Flavor_Composition.pdf}
%\includegraphics[width=0.32\textwidth]{\FCNCFigures/tthML/Limit/fcnc_uh_NLLscan_JET_Pileup_RhoTopology.pdf}
%\\
%\includegraphics[width=0.32\textwidth]{\FCNCFigures/tthML/Limit/fcnc_ch_NLLscan_JET_Flavor_Composition.pdf}
%\includegraphics[width=0.32\textwidth]{\FCNCFigures/tthML/Limit/fcnc_ch_NLLscan_JET_Pileup_RhoTopology.pdf}
%\\
%\includegraphics[width=0.32\textwidth]{\FCNCFigures/tthML/Limit/fcnc_prod_uh_NLLscan_JET_Flavor_Composition.pdf}
%\includegraphics[width=0.32\textwidth]{\FCNCFigures/tthML/Limit/fcnc_prod_uh_NLLscan_JET_Pileup_RhoTopology.pdf}
%\\
%\includegraphics[width=0.32\textwidth]{\FCNCFigures/tthML/Limit/fcnc_prod_ch_NLLscan_JET_Flavor_Composition.pdf}
%\includegraphics[width=0.32\textwidth]{\FCNCFigures/tthML/Limit/fcnc_prod_ch_NLLscan_JET_Pileup_RhoTopology.pdf}
%\\
%\caption{NLL scans of Pileup\_Rho\_Topology and JET\_Flavor\_Composition in leptonic channel.}
%\label{fig:tthML_nllscan_1}
%\end{figure}




\begin{figure}[H]
\centering
\includegraphics[width=0.45\textwidth]{\FCNCFigures/xTFW/trexfitter/ttbarPS/reg2mtau1b2jos_vetobtagwp70_highmet_Combined_ttbarPS.pdf}
\includegraphics[width=0.45\textwidth]{\FCNCFigures/xTFW/trexfitter/ttbarPS/reg2mtau1b3jos_vetobtagwp70_highmet_Combined_ttbarPS.pdf}
\caption{The background change caused by $\bar{t}t$ PS variation in hadronic channels.}
\label{fig:xTFW_ttbarPS}
\end{figure}

\begin{figure}[H]
\centering
\includegraphics[width=0.45\textwidth]{\FCNCFigures/xTFW/trexfitter/ttbarME/reg2mtau1b2jos_vetobtagwp70_highmet_Combined_ttbarME.pdf}
\includegraphics[width=0.45\textwidth]{\FCNCFigures/xTFW/trexfitter/ttbarME/reg2mtau1b3jos_vetobtagwp70_highmet_Combined_ttbarME.pdf}
\caption{The background change caused by $\bar{t}t$ ME variation in hadronic channels.}
\label{fig:xTFW_ttbarME}
\end{figure}

\begin{figure}[H]
\centering
\includegraphics[width=0.45\textwidth]{\FCNCFigures/xTFW/trexfitter/ttbarhdamp/reg2mtau1b2jos_vetobtagwp70_highmet_Combined_ttbarhdamp.pdf}
\includegraphics[width=0.45\textwidth]{\FCNCFigures/xTFW/trexfitter/ttbarhdamp/reg2mtau1b3jos_vetobtagwp70_highmet_Combined_ttbarhdamp.pdf}
\caption{The background change caused by $\bar{t}t$ hdamp variation in hadronic channels.}
\label{fig:xTFW_ttbarhdamp}
\end{figure}

\begin{figure}[H]
\centering
\includegraphics[width=0.32\textwidth]{\FCNCFigures/xTFW/trexfitter/btag_B_0/reg2mtau1b2jos_vetobtagwp70_highmet_Combined_btag_B_0.pdf}
\includegraphics[width=0.32\textwidth]{\FCNCFigures/xTFW/trexfitter/btag_B_0/reg2mtau1b3jos_vetobtagwp70_highmet_Combined_btag_B_0.pdf}
\caption{The background change caused by btag\_B\_0 in hadronic channels.}
\label{fig:xTFW_btag_B_0}
\end{figure}

\begin{figure}[H]
\centering
\includegraphics[width=0.32\textwidth]{\FCNCFigures/xTFW/trexfitter/btag_B_1/reg2mtau1b2jos_vetobtagwp70_highmet_Combined_btag_B_1.pdf}
\includegraphics[width=0.32\textwidth]{\FCNCFigures/xTFW/trexfitter/btag_B_1/reg2mtau1b3jos_vetobtagwp70_highmet_Combined_btag_B_1.pdf}
\caption{The background change caused by btag\_B\_1 in hadronic channels.}
\label{fig:xTFW_btag_B_1}
\end{figure}

\begin{figure}[H]
\centering
\includegraphics[width=0.32\textwidth]{\FCNCFigures/xTFW/trexfitter/btag_B_3/reg2mtau1b2jos_vetobtagwp70_highmet_Combined_btag_B_3.pdf}
\includegraphics[width=0.32\textwidth]{\FCNCFigures/xTFW/trexfitter/btag_B_3/reg2mtau1b3jos_vetobtagwp70_highmet_Combined_btag_B_3.pdf}
\caption{The background change caused by btag\_B\_3 in hadronic channels.}
\label{fig:xTFW_btag_B_3}
\end{figure}

\begin{figure}[H]
\centering
\includegraphics[width=0.32\textwidth]{\FCNCFigures/xTFW/trexfitter/btag_B_37/reg2mtau1b2jos_vetobtagwp70_highmet_Combined_btag_B_37.pdf}
\includegraphics[width=0.32\textwidth]{\FCNCFigures/xTFW/trexfitter/btag_B_37/reg2mtau1b3jos_vetobtagwp70_highmet_Combined_btag_B_37.pdf}
\caption{The background change caused by btag\_B\_37 in hadronic channels.}
\label{fig:xTFW_btag_B_37}
\end{figure}

\begin{figure}[H]
\centering
\includegraphics[width=0.32\textwidth]{\FCNCFigures/xTFW/trexfitter/btag_C_0/reg2mtau1b2jos_vetobtagwp70_highmet_Combined_btag_C_0.pdf}
\includegraphics[width=0.32\textwidth]{\FCNCFigures/xTFW/trexfitter/btag_C_0/reg2mtau1b3jos_vetobtagwp70_highmet_Combined_btag_C_0.pdf}
\caption{The background change caused by btag\_C\_0 in hadronic channels.}
\label{fig:xTFW_btag_C_0}
\end{figure}

\begin{figure}[H]
\centering
\includegraphics[width=0.32\textwidth]{\FCNCFigures/xTFW/trexfitter/btag_C_1/reg2mtau1b2jos_vetobtagwp70_highmet_Combined_btag_C_1.pdf}
\includegraphics[width=0.32\textwidth]{\FCNCFigures/xTFW/trexfitter/btag_C_1/reg2mtau1b3jos_vetobtagwp70_highmet_Combined_btag_C_1.pdf}
\caption{The background change caused by btag\_C\_1 in hadronic channels.}
\label{fig:xTFW_btag_C_1}
\end{figure}

\begin{figure}[H]
\centering
\includegraphics[width=0.32\textwidth]{\FCNCFigures/xTFW/trexfitter/btag_C_5/reg2mtau1b2jos_vetobtagwp70_highmet_Combined_btag_C_5.pdf}
\includegraphics[width=0.32\textwidth]{\FCNCFigures/xTFW/trexfitter/btag_C_5/reg2mtau1b3jos_vetobtagwp70_highmet_Combined_btag_C_5.pdf}
\caption{The background change caused by btag\_C\_5 in hadronic channels.}
\label{fig:xTFW_btag_C_5}
\end{figure}

\begin{figure}[H]
\centering
\includegraphics[width=0.32\textwidth]{\FCNCFigures/xTFW/trexfitter/btag_C_7/reg2mtau1b2jos_vetobtagwp70_highmet_Combined_btag_C_7.pdf}
\includegraphics[width=0.32\textwidth]{\FCNCFigures/xTFW/trexfitter/btag_C_7/reg2mtau1b3jos_vetobtagwp70_highmet_Combined_btag_C_7.pdf}
\caption{The background change caused by btag\_C\_7 in hadronic channels.}
\label{fig:xTFW_btag_C_7}
\end{figure}

\begin{figure}[H]
\centering
\includegraphics[width=0.32\textwidth]{\FCNCFigures/xTFW/trexfitter/btag_C_8/reg2mtau1b2jos_vetobtagwp70_highmet_Combined_btag_C_8.pdf}
\includegraphics[width=0.32\textwidth]{\FCNCFigures/xTFW/trexfitter/btag_C_8/reg2mtau1b3jos_vetobtagwp70_highmet_Combined_btag_C_8.pdf}
\caption{The background change caused by btag\_C\_8 in hadronic channels.}
\label{fig:xTFW_btag_C_8}
\end{figure}

\subsection{ ttbar parton shower systematic check}
For ttbar Parton Shower untertainty, Pythia + Herwig 7.0.X samples are used to derived the impact of the PS systematic. Given the high rank of PS uncertaity in leptonic channel and the availibity of the newer PW+H7.1 samples(hereafer referred to as PS71 systematic), the limit with PS71 applied are checked and shown here Table.\ref{tab:tthML_ttbarPS_vs_ttbarPS71}. The impact on the limits is small in the leptonic channels. 

\begin{table}[H]
\caption{The expected $95\%$ CL exclusion upper limits on signal ( $\mu=1\to~\mathcal{B}(t\to Hq)=0.1\%$ ) with the Asimov (B-only) in the leptonic channels,  with JET\_Flavor\_Composition and JET\_Pileup\_RhoTopology set to TWOSIDED.}
\label{tab:tthML_max_vs_twosided}
\centering
\begin{tabular}{cccc} \toprule\toprule
 & $t_l\tauhad$-2j & $t_l\tauhad$-1j & $t_h\tlhad$-2j\\\midrule
$\bar{t}t\to bWcH$ & $4.29^{+1.79}_{-1.20}$ & $3.90^{+1.81}_{-1.09}$ & $6.86^{+3.25}_{-1.92}$\\
$cg\to tH$ &  / & $86.70^{+35.90}_{-24.23}$ & $36.82^{+15.21}_{-10.29}$\\
tcH merged & $4.12^{+1.68}_{-1.15}$ & $3.64^{+1.59}_{-1.02}$ & $5.78^{+2.66}_{-1.61}$\\
$\bar{t}t\to bWuH$ & $3.99^{+1.69}_{-1.12}$ & $3.58^{+1.67}_{-1.00}$ & $6.68^{+3.00}_{-1.87}$\\
$ug\to tH$ & $22.47^{+8.99}_{-6.28}$ & $16.57^{+6.85}_{-4.63}$ & $5.48^{+2.28}_{-1.53}$\\
tuH merged & $3.35^{+1.37}_{-0.94}$ & $2.84^{+1.21}_{-0.79}$ & $3.01^{+1.26}_{-0.84}$\\
\bottomrule\bottomrule\\
\end{tabular}
\begin{tabular}{cccc} \toprule\toprule
 & $t_h\tlhad$-3j & $t_l\thadhad$ & Combined\\\midrule
$\bar{t}t\to bWcH$ & $3.42^{+1.63}_{-0.95}$ & $0.64^{+0.29}_{-0.18}$ & $0.60^{+0.26}_{-0.17}$\\
$cg\to tH$ & $56.60^{+26.72}_{-15.82}$ & $7.79^{+3.84}_{-2.18}$ & $7.34^{+3.52}_{-2.05}$\\
tcH merged & $3.13^{+1.40}_{-0.87}$ & $0.59^{+0.26}_{-0.16}$ & $0.55^{+0.24}_{-0.15}$\\
$\bar{t}t\to bWuH$ & $3.26^{+1.55}_{-0.91}$ & $0.61^{+0.29}_{-0.17}$ & $0.57^{+0.26}_{-0.16}$\\
$ug\to tH$ & $9.83^{+5.20}_{-2.75}$ & $1.62^{+0.85}_{-0.45}$ & $1.49^{+0.74}_{-0.42}$\\
tuH merged & $2.27^{+0.95}_{-0.64}$ & $0.42^{+0.19}_{-0.12}$ & $0.39^{+0.17}_{-0.11}$\\
\bottomrule\bottomrule\\
\end{tabular}
\end{table}


\begin{table}[H]
\caption{The expected $95\%$ CL exclusion upper limits on signal ( $\mu=1\to~\mathcal{B}(t\to Hq)=0.1\%$ ) with the Asimov (B-only) in the leptonic channels, all uncertainties with PS71 systematic included.}
\label{tab:tthML_ttbarPS_vs_ttbarPS71}
\input{\FCNCTables/tthML/limits_PS71sys}
\end{table}


\subsection{partial unblined fit in low BDT region}
In this analysis and FCNC H(bb) analysis, an uncertainty on hdamp(referred to as hdamp) for ttbar samples is used instead of the generator uncertainty (PW vs amc@NLO, referred to as ME). We explore the difference between these two types of ttbar theoretical uncertainties based on the following setups(Fig.\ref{fig:xTFW_ME_vs_MEhdamp}). Hereafter, the all sys means all systematics except ME and hdamp. The impact on the final limits is small in the leptonic channels. 
\begin{itemize}
	\item compare all sys + ME  vs  all sys + ME + hdamp
	\item compare all sys + ME  vs  all sys + hdamp 
	\item compare all sys + ME  vs  all sys
\end{itemize}

\begin{figure}[H]
\centering
\includegraphics[width=0.32\textwidth]{\FCNCFigures/xTFW/trexfitter/ME_hdamp/ME_vs_MEhdamp_NuisPar_comp.pdf}
\includegraphics[width=0.32\textwidth]{\FCNCFigures/xTFW/trexfitter/ME_hdamp/ME_vs_hdamp_NuisPar_comp.pdf}
\includegraphics[width=0.32\textwidth]{\FCNCFigures/xTFW/trexfitter/ME_hdamp/ME_vs_nohdamp_NuisPar_comp.pdf}
\caption{The NP pulls with different settings in hadronic channels.(left:ME vs ME+hdamp, middle:ME vs hdamp, right:ME vs nohdamp)}
\label{fig:xTFW_ME_vs_MEhdamp}
\end{figure}

		

Before unblinding procedure, small mismodelling could be found  in $t_h\tlhad$-3j and $t_l\tauhad$-1j signal regions.Therefore, it would be useful to run a realistic Asimov fit to see if this problem with the background modelling is resolved. A BONLY fit with data in the CRs and SR up to 0.2 in the BDT discriminant distributions is performed. This is similar to the strategy used by the FCNC H(bb) analysis~\cite{ANA-TOPQ-2018-41-INT1}. The results are shown in Figure~\ref{fig:tthML_partial_unblinding_tuH}
\iffalse and Figure~\ref{fig:tthML_partial_unblinding_tcH} \fi. The background modeling is improved by a factor of 2 in $t_l\tauhad$-1j channel. After rescaling the background from the fits in Table~\ref{tab:tthML_rescaled_factor}), we have evaluated the impact on the limits is small as shown in Table~\ref{tab:tthML_rescaled_limit_1j}-\ref{tab:tthML_rescaled_limit_3j}. 


\begin{figure}[H]
\centering
\includegraphics[width=0.3\textwidth]{\FCNCFigures/tthML/partial_unblinding/tuH_reg1l1tau1b1j_ss.pdf}
\includegraphics[width=0.3\textwidth]{\FCNCFigures/tthML/partial_unblinding/tuH_reg1l1tau1b1j_ss_postFit.pdf}
\\
\includegraphics[width=0.3\textwidth]{\FCNCFigures/tthML/partial_unblinding/tuH_reg1l1tau1b3j_os.pdf}
\includegraphics[width=0.3\textwidth]{\FCNCFigures/tthML/partial_unblinding/tuH_reg1l1tau1b3j_os_postFit.pdf}
\\
\includegraphics[width=0.3\textwidth]{\FCNCFigures/tthML/partial_unblinding/tuH_Summary.pdf}
\includegraphics[width=0.3\textwidth]{\FCNCFigures/tthML/partial_unblinding/tuH_Summary_postFit.pdf}
\caption{Pre-fit(left) and Post-fit(right) BDT discriminants in the SRs $t_h\tlhad$-3j and $t_l\tauhad$-1j. The histograms show the observed data compared to the MC-based model. The distributions of the BDT discriminants are unblinded up to a value of 0.2}
\label{fig:tthML_partial_unblinding_tuH}
\end{figure}




\begin{figure}[H]
\centering
\includegraphics[width=0.32\textwidth]{\FCNCFigures/tthML/partial_unblinding/3j_NormFactors.pdf}
\includegraphics[width=0.32\textwidth]{\FCNCFigures/tthML/partial_unblinding/1j_NormFactors.pdf}
\\
\caption{Normalisation factors in the SRs $t_h\tlhad$-3j(top) and $t_l\tauhad$-1j(bottom). }
\label{fig:tthML_partial_unblinding_NF}
\end{figure}



\newpage

\begin{figure}[H]
\centering
\includegraphics[width=0.32\textwidth]{\FCNCFigures/tthML/partial_unblinding/3j_NuisPar.pdf}
\includegraphics[width=0.32\textwidth]{\FCNCFigures/tthML/partial_unblinding/1j_NuisPar.pdf}
\caption{Pulls and constraints of the nuisance parameters for the partially unblinded background-only fit to the observed data in the SRs $t_h\tlhad$-3j(left) and $t_l\tauhad$-1j(right).}
\label{fig:tthML_partial_unblinding_pull}
\end{figure}

\newpage

\begin{figure}[H]
\centering
\includegraphics[width=0.32\textwidth]{\FCNCFigures/tthML/partial_unblinding/3j_CorrMatrix.pdf}
\includegraphics[width=0.32\textwidth]{\FCNCFigures/tthML/partial_unblinding/1j_CorrMatrix.pdf}
\caption{Systematic uncertainty covariance matrix for the partially unblinded background-only fit to the observed data in the SRs $t_h\tlhad$-3j(left) and $t_l\tauhad$-1j(right).}
\label{fig:tthML_partial_unblinding_corre}
\end{figure}



%\begin{figure}[H]
%\centering
%\includegraphics[width=0.32\textwidth]{\FCNCFigures/tthML/partial_unblinding/tcH_reg1l1tau1b1j_ss.pdf}
%\includegraphics[width=0.32\textwidth]{\FCNCFigures/tthML/partial_unblinding/tcH_reg1l1tau1b1j_ss_postFit.pdf}
%\\
%\includegraphics[width=0.32\textwidth]{\FCNCFigures/tthML/partial_unblinding/tcH_reg1l1tau1b3j_os.pdf}
%\includegraphics[width=0.32\textwidth]{\FCNCFigures/tthML/partial_unblinding/tcH_reg1l1tau1b3j_os_postFit.pdf}
%\\
%\includegraphics[width=0.32\textwidth]{\FCNCFigures/tthML/partial_unblinding/tcH_Summary.pdf}
%\includegraphics[width=0.32\textwidth]{\FCNCFigures/tthML/partial_unblinding/tcH_Summary_postFit.pdf}
%\caption{Pre-fit(left) and Post-fit(right) BDT discriminants in the SRs $t_h\tlhad$-3j and $t_l\tauhad$-1j in terms of tcH signal. The histograms show the observed data compared to the MC-based model. %The distributions of the BDT discriminants are unblinded up to a value of 0.2}
%\label{fig:tthML_partial_unblinding_tcH}
%\end{figure}



%%\begin{figure}[H]
%%\centering
%%\includegraphics[width=0.32\textwidth]{\FCNCFigures/tthML/partial_unblinding/tuH_reg1l1tau1b1j_ss_beforeNF.pdf}
%%\includegraphics[width=0.32\textwidth]{\FCNCFigures/tthML/partial_unblinding/tuH_reg1l1tau1b1j_ss_postFit_afterNF.pdf}
%%\\
%%\includegraphics[width=0.32\textwidth]{\FCNCFigures/tthML/partial_unblinding/tuH_reg1l1tau1b3j_os_beforeNF.pdf}
%%\includegraphics[width=0.32\textwidth]{\FCNCFigures/tthML/partial_unblinding/tuH_reg1l1tau1b3j_os_postFit_afterNF.pdf}
%%\\
%%\includegraphics[width=0.32\textwidth]{\FCNCFigures/tthML/partial_unblinding/tuH_reg1l1tau1b3j_os_NF.pdf}
%%\includegraphics[width=0.32\textwidth]{\FCNCFigures/tthML/partial_unblinding/tuH_reg1l1tau1b1j_ss_NF.pdf}
%%\caption{Pre-fit(left) and Post-fit(right) BDT discriminants in the SRs $t_h\tlhad$-3j(medium two) and $t_l\tauhad$-1j(top two) in terms of tuH signal with region-dependent norm factor applied to $t\%%bar{t}$ sample. The fitted norm factors are presented in the bottom.}
%%\label{fig:tthML_partial_unblinding_tuH_NF}
%%\end{figure}



%\begin{figure}[H]
%\centering
%\includegraphics[width=0.32\textwidth]{\FCNCFigures/tthML/partial_unblinding/tcH_reg1l1tau1b1j_ss.pdf}
%\includegraphics[width=0.32\textwidth]{\FCNCFigures/tthML/partial_unblinding/tcH_reg1l1tau1b1j_ss_postFit.pdf}
%\\
%\includegraphics[width=0.32\textwidth]{\FCNCFigures/tthML/partial_unblinding/tcH_reg1l1tau1b3j_os.pdf}
%\includegraphics[width=0.32\textwidth]{\FCNCFigures/tthML/partial_unblinding/tcH_reg1l1tau1b3j_os_postFit.pdf}
%\\
%\includegraphics[width=0.32\textwidth]{\FCNCFigures/tthML/partial_unblinding/tcH_reg1l1tau1b3j_os_NF.pdf}
%\includegraphics[width=0.32\textwidth]{\FCNCFigures/tthML/partial_unblinding/tcH_reg1l1tau1b1j_ss_NF.pdf}
%\caption{Pre-fit(left) and Post-fit(right) BDT discriminants in the SRs $t_h\tlhad$-3j(medium two) and $t_l\tauhad$-1j(top two) in terms of tcH signal with region-depedent norm factor applied to $t\bar%{t}$ sample. The fitted norm factors are presented in the bottom.}
%\label{fig:tthML_partial_unblinding_tcH_NF}
%\end{figure}




%%\begin{table}
%%\caption{The expected $95\%$ CL exclusion upper limits on signal ( $\mu=1\to$~BR$(t\to Hq)=0.1\%$ ) with the Asimov (B-only) in the leptonic channels after rescaling the background from the fits.}
%%\label{tab:tthML_rescaled_limit}
%%\centering
%%\begin{tabular}{cccc} \toprule\toprule
%% & $t_l\tauhad$-2j & $t_l\tauhad$-1j & $t_h\tlhad$-2j\\\midrule
%%tcH merged & $4.16^{+1.69}_{-1.16}$ & $8.30^{+3.05}_{-2.32}$ & $5.77^{+2.71}_{-1.61}$\\
%%tuH merged & $3.41^{+1.39}_{-0.95}$ & $6.04^{+2.08}_{-1.69}$ & $2.99^{+1.26}_{-0.84}$\\
%%\bottomrule\bottomrule\\
%%\end{tabular}
%%\begin{tabular}{cccc} \toprule\toprule
%% & $t_h\tlhad$-3j & $t_l\thadhad$ & Combined\\\midrule
%%tcH merged & $2.71^{+1.25}_{-0.76}$ & $0.59^{+0.26}_{-0.16}$ & $0.55^{+0.24}_{-0.15}$\\
%%tuH merged & $1.96^{+0.83}_{-0.55}$ & $0.42^{+0.19}_{-0.12}$ & $0.40^{+0.17}_{-0.11}$\\
%%\bottomrule\bottomrule\\
%%\end{tabular}
%%\end{table}
%%
%%
%%\begin{table}
%%\caption{The expected $95\%$ CL exclusion upper limits on signal ( $\mu=1\to$~BR$(t\to Hq)=0.1\%$ ) with the Asimov (B-only) in the leptonic channels after rescaling the background from the fits.}
%%\label{tab:tthML_rescaled_limit_1}
%%\centering
%%\begin{tabular}{cccc} \toprule\toprule
%% & $t_l\tauhad$-2j & $t_l\tauhad$-1j & $t_h\tlhad$-2j\\\midrule
%%tcH merged & $4.16^{+1.69}_{-1.16}$ & $6.36^{+1.08}_{-1.78}$ & $5.77^{+2.71}_{-1.61}$\\
%%tuH merged & $3.41^{+1.39}_{-0.95}$ & $4.82^{+0.83}_{-1.35}$ & $2.99^{+1.26}_{-0.84}$\\
%%\bottomrule\bottomrule\\
%%\end{tabular}
%%\begin{tabular}{cccc} \toprule\toprule
%% & $t_h\tlhad$-3j & $t_l\thadhad$ & Combined\\\midrule
%%tcH merged & $2.74^{+1.27}_{-0.76}$ & $0.59^{+0.26}_{-0.16}$ & $0.55^{+0.24}_{-0.15}$\\
%%tuH merged & $1.98^{+0.84}_{-0.55}$ & $0.42^{+0.19}_{-0.12}$ & $0.40^{+0.17}_{-0.11}$\\
%%\bottomrule\bottomrule\\
%%\end{tabular}
%%\end{table}

\begin{table}
\begin{center} 
\begin{tabular}{ccc}\toprule\toprule
& $t_l\tauhad$-1j & $t_h\tlhad$-3j \\\midrule
  QCD Fake                 & 0.90  & 0.95 \\
  W-jet fake               & 1.25  & 1.21 \\
  Other Fake               & 0.99  & 0.76 \\
  b Fake                   & 1.14  & 1.20 \\
  SM Higgs                 & 0.81  & 0.94 \\
  $\bar{t}tV$              & 0.99  & 0.98 \\
  Diboson                  & 0.98  & 0.89 \\
  Z$\rightarrow \tau\tau$  & 0.83  & 0.93 \\
  Lep fake                 & 0.95  & 0.92 \\
  $\bar{t}t$               & 1.01  & 0.98 \\
  Rare                     & 0.95  & 0.89 \\
\bottomrule\bottomrule\\
\end{tabular} 
\caption{Scale factor derived from the BONLY fit in the low BDT region in $t_l\tauhad$-1j and $t_h\tlhad$-3j regions.} 
\label{tab:tthML_rescaled_factor}
\end{center} 
\end{table}


\begin{table}
%%\centering
%%\begin{tabular}{cccc} \toprule\toprule
%% & $t_l\tauhad$-2j & $t_l\tauhad$-1j & $t_h\tlhad$-2j\\\midrule
%%tcH merged & $4.16^{+1.69}_{-1.16}$ & $3.52^{+1.49}_{-0.98}$ & $5.77^{+2.71}_{-1.61}$\\
%%tuH merged & $3.41^{+1.39}_{-0.95}$ & $2.82^{+1.18}_{-0.79}$ & $2.99^{+1.26}_{-0.84}$\\
%%\bottomrule\bottomrule\\
%%\end{tabular}
%%\begin{tabular}{cccc} \toprule\toprule
%% & $t_h\tlhad$-3j & $t_l\thadhad$ & Combined\\\midrule
%%tcH merged & $2.74^{+1.27}_{-0.76}$ & $0.59^{+0.26}_{-0.16}$ & $0.55^{+0.24}_{-0.15}$\\
%%tuH merged & $1.98^{+0.84}_{-0.55}$ & $0.42^{+0.19}_{-0.12}$ & $0.39^{+0.17}_{-0.11}$\\
%%\bottomrule\bottomrule\\
%%\end{tabular}
\centering
\begin{tabular}{cccc} \toprule\toprule
 & $t_l\tauhad$-2j & $t_l\tauhad$-1j & $t_h\tlhad$-2j\\\midrule
tcH merged & $4.33^{+1.75}_{-1.21}$ & $3.52^{+1.49}_{-0.98}$ & $5.54^{+2.62}_{-1.55}$\\
tuH merged & $3.54^{+1.44}_{-0.99}$ & $2.82^{+1.18}_{-0.79}$ & $2.88^{+1.22}_{-0.80}$\\
\bottomrule\bottomrule\\
\end{tabular}
\begin{tabular}{cccc} \toprule\toprule
 & $t_h\tlhad$-3j & $t_l\thadhad$ & Combined\\\midrule
tcH merged & $2.71^{+1.27}_{-0.76}$ & $0.59^{+0.26}_{-0.16}$ & $0.55^{+0.24}_{-0.15}$\\
tuH merged & $1.96^{+0.83}_{-0.55}$ & $0.42^{+0.18}_{-0.12}$ & $0.39^{+0.17}_{-0.11}$\\
\bottomrule\bottomrule\\
\end{tabular}
\caption{The expected $95\%$ CL exclusion upper limits on signal ( $\mu=1\to~\mathcal{B}(t\to Hq)=0.1\%$ ) with the Asimov in the leptonic channels after rescaling the background using scale factors derived in  $t_l\tauhad$-1j region.} 
\label{tab:tthML_rescaled_limit_1j}
\end{table}

\begin{table}
\centering
\begin{tabular}{cccc} \toprule\toprule
 & $t_l\tauhad$-2j & $t_l\tauhad$-1j & $t_h\tlhad$-2j\\\midrule
tcH merged & $4.21^{+1.71}_{-1.18}$ & $3.38^{+1.44}_{-0.94}$ & $5.57^{+2.63}_{-1.56}$\\
tuH merged & $3.44^{+1.40}_{-0.96}$ & $2.70^{+1.13}_{-0.75}$ & $2.90^{+1.23}_{-0.81}$\\
\bottomrule\bottomrule\\
\end{tabular}
\begin{tabular}{cccc} \toprule\toprule
 & $t_h\tlhad$-3j & $t_l\thadhad$ & Combined\\\midrule
tcH merged & $2.74^{+1.27}_{-0.76}$ & $0.58^{+0.26}_{-0.16}$ & $0.54^{+0.24}_{-0.15}$\\
tuH merged & $1.98^{+0.84}_{-0.55}$ & $0.42^{+0.18}_{-0.12}$ & $0.39^{+0.17}_{-0.11}$\\
\bottomrule\bottomrule\\
\end{tabular}
\caption{The expected $95\%$ CL exclusion upper limits on signal ( $\mu=1\to~\mathcal{B}(t\to Hq)=0.1\%$ ) with the Asimov in the leptonic channels after rescaling the background using scale factors derived in $t_h\tlhad$-3j region.} 
\label{tab:tthML_rescaled_limit_3j}
\end{table}

\begin{figure}[H]
\centering
\includegraphics[width=0.3\textwidth]{\FCNCFigures/tthML/partial_unblinding/tuH_1j_Summary.pdf}
\includegraphics[width=0.3\textwidth]{\FCNCFigures/tthML/partial_unblinding/tuH_1j_Summary_postFit.pdf}
\\
\includegraphics[width=0.3\textwidth]{\FCNCFigures/tthML/partial_unblinding/tuH_3j_Summary.pdf}
\includegraphics[width=0.3\textwidth]{\FCNCFigures/tthML/partial_unblinding/tuH_3j_Summary_postFit.pdf}
\caption{Pre-fit(left) and Post-fit(right) BDT discriminants after scaling using SFs derived in the $t_h\tlhad$-3j(bottom two) and $t_l\tauhad$-1j(top two) in terms of tuH signal. }
\label{fig:tthML_partial_unblinding_tuH}
\end{figure}


\subsection{Quark-gluon fraction check}
Quark-gluon fraction is also considered when we derived the Jet Flavour Composition systematics, the relative difference between default 50-50\% fraction and calculated qg
fraction(Fig.~\ref{fig:cal_qg_frac}) are shown in Figure~\ref{fig:cal_and_defau_qg_frac} using partial $t\bar{t}$ samples. Given the statistical error, the descrepancy between these two settings could be negelected.


\begin{figure}[H]
\centering
\includegraphics[width=0.3\textwidth]{\FCNCFigures/tthML/partial_unblinding/qgfraction_2j.pdf}
\includegraphics[width=0.3\textwidth]{\FCNCFigures/tthML/partial_unblinding/qgfraction_3j.pdf}
\\
\caption{Comparision between default qg fraction and calculated qg fraction in 2j(left) region and 3j(right) region in hadronic channel.}
\label{fig:cal_and_defau_qg_frac}
\end{figure}

\begin{figure}[H]
\centering
\includegraphics[width=0.3\textwidth]{\FCNCFigures/tthML/partial_unblinding/qgfraction.pdf}
\\
\caption{Calculated qg fraction using $t\bar{t}$ samples.}
\label{fig:cal_qg_frac}
\end{figure}

Futhermore, in the final pull distribution, the Jet Flavour Composition is pulled down to the 1sigma boudary, a cross check with decorelating this NP in fifferent signal region to see the pulls. We can conclude that this NP is reasonable in all of the signal regions, whose fitted value is within the 1 sigma error of nominal as shown in Figure \ref{fig:decorrelated_JFC}.
\begin{figure}[H]
\centering
\includegraphics[width=0.32\textwidth]{\FCNCFigures/unblinded/decorrelated/tuH_NuisPar.pdf}
\includegraphics[width=0.32\textwidth]{\FCNCFigures/unblinded/decorrelated/tcH_NuisPar.pdf}
\\
\caption{Comparison of pulls distribution with decorrelated Jet Flavour Composition in SRs in leptonic channel in terms of tuH merged signal(left) and tcH merged signal(right).}
\label{fig:decorrelated_JFC}
\end{figure}



\subsection{fake tau calibration SF check}
The scale factors for the fake taus calibration in the Monte Carlo is described in the chapter of background estimation in body of note, the nominal value of the scale factors will vary along with other uncertainties in the final fit, these varied scale factors is shown here.

\begin{figure}[H]
\centering
\includegraphics[width=0.32\textwidth]{\FCNCFigures/tthML/SF_VS_NP/output_1p_sf_b_fake_pt2535.pdf}
\includegraphics[width=0.32\textwidth]{\FCNCFigures/tthML/SF_VS_NP/output_1p_sf_b_fake_pt3545.pdf}
\includegraphics[width=0.32\textwidth]{\FCNCFigures/tthML/SF_VS_NP/output_1p_sf_b_fake_pt45125.pdf}
\\
\caption{Comparison of scale factor for b fake tau with pt from 25GeV to 35GeV(left), 35GeV to 45GeV(middle) adn 45GeV to 125GeV(right) in one prong bin in leptonic channel.}
\label{fig:1p_sf_b_fake_pt2535}
\end{figure}

\begin{figure}[H]
\centering
\includegraphics[width=0.32\textwidth]{\FCNCFigures/tthML/SF_VS_NP/output_3p_sf_b_fake_pt2535.pdf}
\includegraphics[width=0.32\textwidth]{\FCNCFigures/tthML/SF_VS_NP/output_3p_sf_b_fake_pt3545.pdf}
\includegraphics[width=0.32\textwidth]{\FCNCFigures/tthML/SF_VS_NP/output_3p_sf_b_fake_pt45125.pdf}
\\
\caption{Comparison of scale factor for b fake tau with pt from 25GeV to 35GeV(left), 35GeV to 45GeV(middle) adn 45GeV to 125GeV(right) in three prong bin in leptonic channel.}
\label{fig:3p_sf_b_fake_pt2535}
\end{figure}
	





\begin{figure}[H]
\centering
\includegraphics[width=0.32\textwidth]{\FCNCFigures/tthML/SF_VS_NP/output_1p_sf_other_fake_pt2535.pdf}	
\includegraphics[width=0.32\textwidth]{\FCNCFigures/tthML/SF_VS_NP/output_1p_sf_other_fake_pt3545.pdf}	
\includegraphics[width=0.32\textwidth]{\FCNCFigures/tthML/SF_VS_NP/output_1p_sf_other_fake_pt45125.pdf}
\\
\caption{Comparison of scale factor for other fake tau with pt from 25GeV to 35GeV(left), 35GeV to 45GeV(middle) adn 45GeV to 125GeV(right) in one prong bin in leptonic channel.}
\label{fig:1p_sf_b_fake_pt2535}
\end{figure}

\begin{figure}[H]
\centering
\includegraphics[width=0.32\textwidth]{\FCNCFigures/tthML/SF_VS_NP/output_3p_sf_other_fake_pt2535.pdf}
\includegraphics[width=0.32\textwidth]{\FCNCFigures/tthML/SF_VS_NP/output_3p_sf_other_fake_pt3545.pdf}
\includegraphics[width=0.32\textwidth]{\FCNCFigures/tthML/SF_VS_NP/output_3p_sf_other_fake_pt45125.pdf}
\\
\caption{Comparison of scale factor for other fake tau with pt from 25GeV to 35GeV(left), 35GeV to 45GeV(middle) adn 45GeV to 125GeV(right) in three prong bin in leptonic channel.}
\label{fig:3p_sf_b_fake_pt2535}
\end{figure}





\begin{figure}[H]
\centering
\includegraphics[width=0.32\textwidth]{\FCNCFigures/tthML/SF_VS_NP/output_1p_sf_w_jet_fake_os_pt2535.pdf}
\includegraphics[width=0.32\textwidth]{\FCNCFigures/tthML/SF_VS_NP/output_1p_sf_w_jet_fake_os_pt3545.pdf}
\includegraphics[width=0.32\textwidth]{\FCNCFigures/tthML/SF_VS_NP/output_1p_sf_w_jet_fake_os_pt45125.pdf}
\\
\caption{Comparison of scale factor for w fake tau(os) with pt from 25GeV to 35GeV(left), 35GeV to 45GeV(middle) adn 45GeV to 125GeV(right) in one prong bin in leptonic channel.}
\label{fig:1p_sf_b_fake_pt2535}
\end{figure}

\begin{figure}[H]
\centering
\includegraphics[width=0.32\textwidth]{\FCNCFigures/tthML/SF_VS_NP/output_3p_sf_w_jet_fake_os_pt2535.pdf}
\includegraphics[width=0.32\textwidth]{\FCNCFigures/tthML/SF_VS_NP/output_3p_sf_w_jet_fake_os_pt3545.pdf}
\includegraphics[width=0.32\textwidth]{\FCNCFigures/tthML/SF_VS_NP/output_3p_sf_w_jet_fake_os_pt45125.pdf}
\\
\caption{Comparison of scale factor for w fake tau(os) with pt from 25GeV to 35GeV(left), 35GeV to 45GeV(middle) adn 45GeV to 125GeV(right) in three prong bin in leptonic channel.}
\label{fig:3p_sf_b_fake_pt2535}
\end{figure}


\begin{figure}[H]
\centering
\includegraphics[width=0.32\textwidth]{\FCNCFigures/tthML/SF_VS_NP/output_1p_sf_w_jet_fake_ss_pt2535.pdf}
\includegraphics[width=0.32\textwidth]{\FCNCFigures/tthML/SF_VS_NP/output_1p_sf_w_jet_fake_ss_pt3545.pdf}
\includegraphics[width=0.32\textwidth]{\FCNCFigures/tthML/SF_VS_NP/output_1p_sf_w_jet_fake_ss_pt45125.pdf}
\\
\caption{Comparison of scale factor for w fake tau(ss) with pt from 25GeV to 35GeV(left), 35GeV to 45GeV(middle) adn 45GeV to 125GeV(right) in one prong bin in leptonic channel.}
\label{fig:1p_sf_b_fake_pt2535}
\end{figure}

\begin{figure}[H]
\centering
\includegraphics[width=0.32\textwidth]{\FCNCFigures/tthML/SF_VS_NP/output_3p_sf_w_jet_fake_ss_pt2535.pdf}
\includegraphics[width=0.32\textwidth]{\FCNCFigures/tthML/SF_VS_NP/output_3p_sf_w_jet_fake_ss_pt3545.pdf}
\includegraphics[width=0.32\textwidth]{\FCNCFigures/tthML/SF_VS_NP/output_3p_sf_w_jet_fake_ss_pt45125.pdf}
\\
\caption{Comparison of scale factor for w fake tau(ss) with pt from 25GeV to 35GeV(left), 35GeV to 45GeV(middle) adn 45GeV to 125GeV(right) in three prong bin in leptonic channel.}
\label{fig:3p_sf_b_fake_pt2535}
\end{figure}


For further check of the fit before looking at the real data, another BONLY fit with data in the CRs and SR up to 0.2 in the BDT discriminant distributions is made. In the $t_h\tlhad$-3j and $t_l\tauhad$-1j signal regions, the fake tau correction factors could have been floated. After the unblinded fit, we can derived the nomarlization factor for each samples, then a realistic Asimov fit is performed after scaling the samples. The results are shown in Figure. The impact on the limits have been evaluated in Table~\ref{tab:tthML_rescaled_limit_fake_1j}-\ref{tab:tthML_rescaled_limit_fake_3j}, which is small.

\begin{table}
\centering
\begin{tabular}{cccc} \toprule\toprule
 & $t_l\tauhad$-2j & $t_l\tauhad$-1j & $t_h\tlhad$-2j\\\midrule
tcH merged & $4.18^{+1.70}_{-1.17}$ & $3.88^{+1.62}_{-1.09}$ & $4.98^{+2.38}_{-1.39}$\\
tuH merged & $3.39^{+1.38}_{-0.95}$ & $3.10^{+1.28}_{-0.87}$ & $2.58^{+1.10}_{-0.72}$\\
\bottomrule\bottomrule\\
\end{tabular}
\begin{tabular}{cccc} \toprule\toprule
 & $t_h\tlhad$-3j & $t_l\thadhad$ & Combined\\\midrule
tcH merged & $2.37^{+1.14}_{-0.66}$ & $0.56^{+0.25}_{-0.16}$ & $0.51^{+0.22}_{-0.14}$\\
tuH merged & $1.69^{+0.73}_{-0.47}$ & $0.40^{+0.18}_{-0.11}$ & $0.37^{+0.16}_{-0.10}$\\
\bottomrule\bottomrule\\
\end{tabular}
\caption{The expected $95\%$ CL exclusion upper limits on signal ( $\mu=1\to~\mathcal{B}(t\to Hq)=0.1\%$ ) with the Asimov in the leptonic channels after rescaling the background using scale factors derived in $t_l\tauhad$-1j region.} 
\label{tab:tthML_rescaled_limit_fake_1j}
\end{table}



\begin{table}
\centering
\begin{tabular}{cccc} \toprule\toprule
 & $t_l\tauhad$-2j & $t_l\tauhad$-1j & $t_h\tlhad$-2j\\\midrule
tcH merged & $4.21^{+1.71}_{-1.18}$ & $3.38^{+1.44}_{-0.94}$ & $5.57^{+2.63}_{-1.56}$\\
tuH merged & $3.44^{+1.40}_{-0.96}$ & $2.70^{+1.13}_{-0.75}$ & $2.90^{+1.23}_{-0.81}$\\
\bottomrule\bottomrule\\
\end{tabular}
\begin{tabular}{cccc} \toprule\toprule
 & $t_h\tlhad$-3j & $t_l\thadhad$ & Combined\\\midrule
tcH merged & $2.74^{+1.27}_{-0.76}$ & $0.58^{+0.26}_{-0.16}$ & $0.54^{+0.24}_{-0.15}$\\
tuH merged & $1.98^{+0.84}_{-0.55}$ & $0.42^{+0.18}_{-0.12}$ & $0.39^{+0.17}_{-0.11}$\\
\bottomrule\bottomrule\\
\end{tabular}
\caption{The expected $95\%$ CL exclusion upper limits on signal ( $\mu=1\to~\mathcal{B}(t\to Hq)=0.1\%$ ) with the Asimov in the leptonic channels after rescaling the background using scale factors derived in $t_h\tlhad$-3j region.} 
\label{tab:tthML_rescaled_limit_fake_3j}
\end{table}




\begin{figure}[H]
\centering
\includegraphics[width=0.7\textwidth]{\FCNCFigures/tthML/partial_unblinding/fake_plot/3j_NormFactors.pdf}
\includegraphics[width=0.7\textwidth]{\FCNCFigures/tthML/partial_unblinding/fake_plot/1j_NormFactors.pdf}
\\
\caption{Normalisation factors for fake tau contibutions in the SRs $t_h\tlhad$-3j(left) and $t_l\tauhad$-1j(right). }
\label{fig:tthML_partial_unblinding_NF_fake}
\end{figure}

\newpage

\begin{figure}[H]
\centering
\includegraphics[width=0.32\textwidth]{\FCNCFigures/tthML/partial_unblinding/fake_plot/3j_NuisPar.pdf}
\includegraphics[width=0.32\textwidth]{\FCNCFigures/tthML/partial_unblinding/fake_plot/1j_NuisPar.pdf}
\caption{Pulls and constraints of the nuisance parameters for the partially unblinded background-only fit to the observed data in the SRs $t_h\tlhad$-3j(left) and $t_l\tauhad$-1j(right).}
\label{fig:tthML_partial_unblinding_pull_fake}
\end{figure}

\newpage

\begin{figure}[H]
\centering
\includegraphics[width=0.32\textwidth]{\FCNCFigures/tthML/partial_unblinding/fake_plot/3j_CorrMatrix.pdf}
\includegraphics[width=0.32\textwidth]{\FCNCFigures/tthML/partial_unblinding/fake_plot/1j_CorrMatrix.pdf}
\caption{Systematic uncertainty covariance matrix for the partially unblinded background-only fit to the observed data in the SRs $t_h\tlhad$-3j(left) and $t_l\tauhad$-1j(right).}
\label{fig:tthML_partial_unblinding_corre_fake}
\end{figure}



\begin{figure}[H]
\centering
\includegraphics[width=0.32\textwidth]{\FCNCFigures/tthML/partial_unblinding/fake_plot/1j_Summary.pdf}
\includegraphics[width=0.32\textwidth]{\FCNCFigures/tthML/partial_unblinding/fake_plot/1j_Summary_postFit.pdf}
\\
\includegraphics[width=0.32\textwidth]{\FCNCFigures/tthML/partial_unblinding/fake_plot/3j_Summary.pdf}
\includegraphics[width=0.32\textwidth]{\FCNCFigures/tthML/partial_unblinding/fake_plot/3j_Summary_postFit.pdf}
\caption{Pre-fit(left) and Post-fit(right) BDT discriminants after scaling using SFs derived in the $t_h\tlhad$-3j(bottom two) and $t_l\tauhad$-1j(top two) in terms of tuH signal. }
\label{fig:tthML_partial_unblinding_tuH}
\end{figure}

\subsection{partial unblined fit VS full unblined fit}
Since the leptonic channel is the most sensitive channel in this analysis, and we also conducted the partial data study in low BDT region in the previous chapter, we compare the results using full data with those from partial data as a cross check.
No significant difference between the pull of NPs are observed in these two fits as shown in Figure~\ref{fig:tthML_partialVSfull}.

\begin{figure}[H]
\centering
\includegraphics[width=0.32\textwidth]{\FCNCFigures/unblinded/partial_VS_full/tuH_NuisPar_comp.pdf}
\includegraphics[width=0.32\textwidth]{\FCNCFigures/unblinded/partial_VS_full/tcH_NuisPar_comp.pdf}
\\
\caption{Comparison of pulls distribution with partial data and full data in leptonic channel in terms of tuH signal(left) and in terms of tcH signal(right).}
\label{fig:tthML_partialVSfull}
\end{figure}

\subsection{Htautau branching ratio theoretical uncertainty check}
After fitting to the unblinded data, we find the HttBR NP disappeared in the ranking plot in terms of tuH signal(see left part in Figure \ref{fig:fcnc_rank_data}) in hadronic channel. This is mainly due to the fact that the fitted signal strength($\mu=0.04$) in this channel is too small and we have made a sainty check for this. In terms of tcH signal, this NP shows up since the fitted strength is larger than for tuH signal. The signal strength cancellation is in work between two signal region(one is positive and the other is negative), which lead to a small fitted value in combined region. From the plots \ref{fig:xTFW_tuH_NF}-\ref{fig:xTFW_tcH_23j_NF}, we can the fitted value is larger, the HttBR is ranked higher, which is expected because the HttBR will affect the normalization of signal yields.


\begin{figure}[H]
\centering
\includegraphics[width=1\textwidth]{\FCNCFigures/unblinded/checkHttBR/tuH_combined_NormFactors.pdf}
\\
\caption{Signal strength for combined SR in terms of tuH merged signal in hadronic channel.}
\label{fig:xTFW_tuH_NF}
\end{figure}



\begin{figure}[H]
\centering
\includegraphics[width=0.3\textwidth]{\FCNCFigures/unblinded/checkHttBR/tuH_2j_ranking.pdf}
\includegraphics[width=0.45\textwidth]{\FCNCFigures/unblinded/checkHttBR/tuH_2j_NormFactors.pdf}
\\
\includegraphics[width=0.3\textwidth]{\FCNCFigures/unblinded/checkHttBR/tuH_3j_ranking.pdf}
\includegraphics[width=0.45\textwidth]{\FCNCFigures/unblinded/checkHttBR/tuH_3j_NormFactors.pdf}
\caption{Raking plots(left two) and signal strength(right two) for 2j(upper two) and 3j(lower two) SR in terms of tuH merged signal in hadronic channel.}
\label{fig:xTFW_tuH_23j_NF}
\end{figure}


\begin{figure}[H]
\centering
\includegraphics[width=0.32\textwidth]{\FCNCFigures/unblinded/checkHttBR/tcH_combined_NormFactors.pdf}
\\
\caption{Signal strength for combined SR in terms of tcH merged signal in hadronic channel.}
\label{fig:xTFW_tcH_NF}
\end{figure}

\begin{figure}[H]
\centering
\includegraphics[width=0.3\textwidth]{\FCNCFigures/unblinded/checkHttBR/tcH_2j_ranking.pdf}
\includegraphics[width=0.45\textwidth]{\FCNCFigures/unblinded/checkHttBR/tcH_2j_NormFactors.pdf}
\\
\includegraphics[width=0.3\textwidth]{\FCNCFigures/unblinded/checkHttBR/tcH_3j_ranking.pdf}
\includegraphics[width=0.45\textwidth]{\FCNCFigures/unblinded/checkHttBR/tcH_3j_NormFactors.pdf}
\caption{Raking plots(left two) and signal strength(right two) for 2j(upper two) and 3j(lower two) SR in terms of tcH merged signal in hadronic channel.}
\label{fig:xTFW_tcH_23j_NF}
\end{figure}



\subsection{Background only fits to the unblinded data}

We have performed a background-only fit to the unblinded data to see if the excess of events is due to the background fluctuation.
The background-only Pre-fit and post-fit plots ate shown in \ref{fig:tthML_remove_signal_1}-\ref{fig:tthML_remove_signal_2}, where the excess of events remains the same
in the $t_l\thadhad$ high BDT region. The fitted pull of NPs are also consistent with what obtained in the S+B fits as shown in Figure~\ref{fig:tthML_remove_signal_2}.

\begin{figure}[H]
\centering
\includegraphics[width=0.4\textwidth]{\FCNCFigures/unblinded/remove_signal_in_plots/reg1l1tau1b1j_ss.pdf}
\includegraphics[width=0.4\textwidth]{\FCNCFigures/unblinded/remove_signal_in_plots/reg1l1tau1b1j_ss_postFit.pdf}
\\
\includegraphics[width=0.4\textwidth]{\FCNCFigures/unblinded/remove_signal_in_plots/reg1l1tau1b2j_ss.pdf}
\includegraphics[width=0.4\textwidth]{\FCNCFigures/unblinded/remove_signal_in_plots/reg1l1tau1b2j_ss_postFit.pdf}
\caption{Comparison of data and MC prediction before fit(left) and after fit(right) in $t_l\tauhad$-1j (top) region, and $t_l\tauhad$-2j region(bottom).}
\label{fig:tthML_remove_signal_1}
\end{figure}

\begin{figure}[H]
\centering
\includegraphics[width=0.4\textwidth]{\FCNCFigures/unblinded/remove_signal_in_plots/reg1l1tau1b2j_os.pdf}
\includegraphics[width=0.4\textwidth]{\FCNCFigures/unblinded/remove_signal_in_plots/reg1l1tau1b2j_os_postFit.pdf}
\\
\includegraphics[width=0.4\textwidth]{\FCNCFigures/unblinded/remove_signal_in_plots/reg1l1tau1b3j_os.pdf}
\includegraphics[width=0.4\textwidth]{\FCNCFigures/unblinded/remove_signal_in_plots/reg1l1tau1b3j_os_postFit.pdf}
\\
\includegraphics[width=0.4\textwidth]{\FCNCFigures/unblinded/remove_signal_in_plots/reg1l2tau1bnj_os.pdf}
\includegraphics[width=0.4\textwidth]{\FCNCFigures/unblinded/remove_signal_in_plots/reg1l2tau1bnj_os_postFit.pdf}
\\
\caption{Comparison of data and MC prediction before fit(left) and after fit(right) in $t_h\tlhad$-2j region(top), $t_h\tlhad$-3j region(middle) and $t_l\thadhad$ (bottom).}
\label{fig:tthML_remove_signal_2}
\end{figure}

\begin{figure}[H]
\centering
\includegraphics[width=0.30\textwidth]{\FCNCFigures/unblinded/NuisPar_comp.pdf}
\caption{Comparison of pulls based on background-only fit and S+B fit using unblinded data in terms of tcH merged signal in leptonic channel.}
\label{fig:tthML_remove_signal_2}
\end{figure}




\newpage
