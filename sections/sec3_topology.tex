\section{Reconstruction of event topology}
\label{sec:reconstruction}

To comply with the signal topology, in each channel, exactly one jet should be tagged as a $b$-jet. 

In $t_ht(qH)$ events, the jet from $t\to qH$, denoted as the FCNC jet, should be a hard narrow jet considering it's from the decay chain $t\to qH\to q\tau\tau$, with taus reconstructed as leptons or $\tauhad$. So all jets from the top hadronic decay and the FCNC jet should pass the jet selection mentioned in the Sec \ref{sec:obj_reco}. There should be at least four jets among which the one with smallest $\Delta R(p^{\mu}_{\text{jet}},p^{\mu}_{\tau1}+p^{\mu}_{\tau2})$ is considered as FCNC jet since the top decay products are likely boosted close together than other
jets, as shown in Figure \ref{fig:FCNCDr}. If there are more than 2 jets beside FCNC jet and $b$-jet, the jets from $W$ boson decay are chosen from the combination which have the invariant mass close to W resonance. There is the chance that one of the jets fails the $\pt$ requirement and not reconstructed. This kind of events will fall into $t_hH$ channel. The FCNC top resonance is still reconstructable given the big chance that the jet which is missing is from $W$ decay.

\begin{figure}[H]
\centering
\includegraphics[width=0.4\textwidth]{\FCNCFigures/truthDr/FCNCDr.pdf}
\caption{The $\Delta R$ between FCNC jet and di-tau candidates (visible) versus $\Delta R$ between other jet and di-tau candidates.}
\label{fig:FCNCDr}
\end{figure}

In $t_hH$ events, there are 3 jets coming from top hadronic decay including the b-jet. So a Higgs resonance formed by the taus and a top resonance formed by the jets are expected.

In $t_hH$ and $t_ht(qH)$ channels, the $\chi^2$ fit
%method introduced in \cite{fcnc_PRD} 
is used to recontruct the ditau mass and momentum by taking the $\tau$ decay kinematics into account. To determine the 4-momenta of the invisible decay products of the tau decays, the following $\chi^2$ in Eq.~\ref{eq:eq2}, based on the collinear approximation is used.

%probability functions above and the constraints from the tau mass, the Higgs mass and the measured $\met$, is defined,
\begin{eqnarray}
\begin{array}{ll}
\chi^2 = 
%-2\ln \mathcal{P}_1 -2\ln \mathcal{P}_2 + \left( \frac{m_{\tau_1}^{\text{fit}} - 1.78}{\sigma_{\tau}} \right)^2 +  \left( \frac{m_{\tau_2}^{\text{fit}} - 1.78}{\sigma_{\tau}} \right)^2 +  
\left( \frac{m_{\tau\tau}^{\text{fit}} - 125}{\sigma_{\tau\tau}} \right)^2 + \left( \frac{E_{x,\text{miss}}^{\text{fit}} - E_{x,\text{miss}}}{\sigma_{\text{miss,x}}} \right)^2 + \left( \frac{E_{y,\text{miss}}^{\text{fit}} - E_{y,\text{miss}}}{\sigma_{\text{miss,y}}} \right)^2 ,
\end{array}
\label{eq:eq2}
\end{eqnarray}
%where $\mathcal{P}_i(\Delta R)$ are the probability distributions of the angular distance of the visible and invisible decay products in the tau decay, parametrized as a function of the momentum of the tau lepton. In the $\taulep$ mode where two neutrinos are present, it is extended to be the joint probability distribution of $\Delta R$ and $m_{\text{mis}}$ with $m_{\text{mis}}$ being the invariant mass of the neutrinos, denoted by $\mathcal{P}(\Delta R, m_{\text{mis}})$. These probability density functions are obtained from the MC simulation. Figure \ref{fig:tau_prob} illustrates the distributions of $\mathcal{P}(\Delta R, m_{\text{mis}})$ for $\taulep$, and $\mathcal{P}(\Delta R)$ for $\tauhad$, with the original tau's momentum in the range 60 GeV$<p<80$ GeV. Figures \ref{fig:chi2terms_hh3j}-\ref{fig:chi2terms_lh4j} in App. \ref{app:reco} show the distributions for each term in Eq. \ref{eq:eq2}.

%\input{\FCNCFigures/tex/tau_probs}

In Eq. \ref{eq:eq2}, the free parameters are the energy ratio of invisible decay products for each tau decay.
The Higgs mass resolution is set to 20~GeV according to \cite{Htautau}. The $\met$ resolution is parametrized as~\cite{met_sys}
\begin{equation}
\sigma_{\text{miss,x(y)}}=13.1 + 0.50\sqrt{\Sigma E_\text{T}},
\label{eq:eq7}
\end{equation}
where $\Sigma E_\text{T}$ (in GeV) is the scalar sum of transverse energy depositions of all objects and clusters. The invisible 4-momenta are obtained by minimizing the combined $\chi^2$ for each event. By adding the Higgs mass constraint term in the kinematic fit, not only is the Higgs mass resolution improved, but also the resolutions of the Higgs boson's four-momentum, and the mass of the top from which the Higgs comes. Figure \ref{fig:chi2} shows the distributions of $\chi^2$ in different regions. Good agreement between data and background predictions are achieved.

\input{\FCNCFigures/tex/chi2}
\input{\FCNCFigures/tex/x12fit}

In $t_l\thadhad$ channels, a Higgs resonance formed by the taus is expected. Additionally for $t_lt(qH)$ with $H\to \thadhad$ events, a top resonance formed by the c/u jet and Higgs is expected.

Due to the large amount of neutrinos produced in leptonic channels with a huge degree of freedom. The kinematic fit to reconstruct the neutrinos is given up in $t_l\thadhad$ and $t_l\thad$ channels. The kinematics calculated directly from visible particles and $\met$ are used as BDT input.

With the event topology reconstructed, a number of variables are defined for signal and background separation. Their distributions can be found later in this section, and some of their explanations are as follows. In the following explanations, di-tau refers to the two $\tauhad$ candidates.  In the case in which there is only one $\tauhad$, di-tau referring to the lepton and the $\tauhad$ candidate.

\begin{enumerate}

\item $E^{T}_{miss}$ is the missing transverse momentum.
\item $p_{T,\tau}$ is the transverse momentum of the leading tau candidate.
\item $p_{T,sub-\tau}$ is the transverse momentum of the sub-leading tau candidate.
\item $p_{T,l}$ is the transverse momentum of the leading lepton.
\item $\chi^2$ is derived from kinematic fitting for the neutrinos.
\item $m_{t,SM}$ is the invariant mass of the $b$-jet and the two jets from the $W$ decay, and reflects the top mass in the decay $t\to Wb \to j_1j_2b$. This variable is only defined for the 4-jet $t_hH$ and $t_ht(qH)$ events.
\item $m^{T}_{W}$ is the transverse mass calculated from the lepton and $\met$ in the leptonic channels, defined as
\begin{equation}
m^{T}_{W} = \sqrt{2 p_{\text{T,lep}} E_{\text{T}}^{\text{miss}} \left(1-\cos\Delta\phi_{\text{lep,miss}} \right)}.  
\end{equation}
\item $m_{\tau,\tau}$ is the invariant mass of the tau candidates and reconstructed neutrinos in $t_hH$ and $t_ht(qH)$ channels. 
\item $m_{W}$ is the reconstructed invariant mass of the hadronic $W$ boson from SM top quark.
\item $m_{t,FCNC}$ is the invariant mass of the FCNC-decaying top quark reconstructed from di-tau candidates, FCNC-jet and reconstructed neutrinos.
\item $m_{\tau\tau,vis}$ is the visible invariant mass of the di-tau candidates.
\item $p_{T,\tau\tau,vis}$ is the $\pt$ of the di-tau candidates.
\item $m_{t,FCNC,vis}$ is the reconstructed visible mass of the FCNC-decaying top quark.
\item $m_{t,SM,vis}$ is the invariant mass of the lepton and the b-jet, which reflects the visible SM top mass.
\item $M(\tau\tau qjet,min)$ is the visible mass of the di-tau candidates (include leptonic tau) and the light-flavor jet, minimized by choosing different jet, reflects the invariant masss of the visible FCNC top decaying product, an alternative of variable $m_{t,FCNC,vis}$.
\item $M(wjet1,wjet2,min)$ is the invariant mass of two light-flavor jet, minimized by choosing different jets, reflects the invariant mass of the W candidate, an alternative of $m_{W}$.
\item $E^{T}_{miss}$ centrality is a measure of how central the $\met$ lies between the two tau candidates in the transverse plane, and is defined as
\begin{eqnarray}
\begin{array}{l}
\met~\text{centrality} = {(x+y)}/{\sqrt{x^2+y^2}}, \\
\text{with}~x=\frac{\sin(\phi_{\text{miss}}-\phi_{\tau_1})}{\sin(\phi_{\tau_2}-\phi_{\tau_1})},  y=\frac{\sin(\phi_{\tau_2}-\phi_{\text{miss}})}{\sin(\phi_{\tau_2}-\phi_{\tau_1})} ,
\end{array}
\label{eq:eq3}
\end{eqnarray}
\item $E_{\nu,i}/E_{\tau,i},i=1,2$ is the momentum fraction carried by the visible decay products from the tau mother. It is based on the best-fit 4-momentum of the neutrino(s) according to the event reconstruction algorithm mentioned earlier in this section. For the $\tauhad$ decay mode, the visible decay products carry most of the tau energy since there is only a single neutrino in the final state, which is evident in the excess around 1 in Figure \ref{fig:x12_fit}. 
\item $\Delta R(l+bjet,\tau\tau)$ is the angular distance between the lepton+b-jet and di-tau candidates.
\item $\Delta R(l,bjet)$ is the angular distance between the lepton and b-jet.
\item $\Delta R(\tau,bjet)$ is the angular distance between the tau and b-jet. If there are two taus in the event, the leading one is selected for the calculation.
\item $\eta_{\tau,max}$ is the maximum $\eta$ value among the tau candidates.
\item $\Delta R(l,\tau)$ is the angular distance between the lepton and the closest tau candidate in the leptonic channels.
\item $\Delta R(\tau,qjet)$ is the angular distance between the tau and the reconstructed fcnc jet.
\item $\Delta R(\tau,\tau)$ is the angular distance between two tau candidates, in case of $t_h\tlhad$ channels, the definition is the same as $\Delta R(l,\tau)$.
\item $\Delta\phi(\tau\tau,P^{T}_{miss})$ is the azimuthal angle between the $\met$ and di-tau $\pt$.

The distributions of those variables in the signal regions (SRs) are shown in:
\begin{itemize}
	\item $t_l\thad$-1j: Figure  		\ref{fig:var_reg1l1tau1b1j_ss_vetobtagwp70_highmet_1}
								-	\ref{fig:var_reg1l1tau1b1j_ss_vetobtagwp70_highmet}
	\item $t_l\thad$-2j: Figure  		\ref{fig:var_reg1l1tau1b2j_ss_vetobtagwp70_highmet_1}
								-	\ref{fig:var_reg1l1tau1b2j_ss_vetobtagwp70_highmet}
	\item $t_h\tlhad$-2j: Figure 		\ref{fig:var_reg1l1tau1b2j_os_vetobtagwp70_highmet_1}
							 	-	\ref{fig:var_reg1l1tau1b2j_os_vetobtagwp70_highmet}
	\item $t_h\tlhad$-3j: Figure 		\ref{fig:var_reg1l1tau1b3j_os_vetobtagwp70_highmet_1}
								-	\ref{fig:var_reg1l1tau1b3j_os_vetobtagwp70_highmet}
	\item $t_l\thadhad$: Figure 		\ref{fig:var_reg1l2tau1bnj_os_1}
								-	\ref{fig:var_reg1l2tau1bnj_os}
	\item $t_h\thadhad$-2j: Figure 	\ref{fig:var_reg2mtau1b2jos_vetobtagwp70_highmet_1}
								-	\ref{fig:var_reg2mtau1b2jos_vetobtagwp70_highmet}
	\item $t_h\thadhad$-3j: Figure 	\ref{fig:var_reg2mtau1b3jos_vetobtagwp70_highmet_1}
								-	\ref{fig:var_reg2mtau1b3jos_vetobtagwp70_highmet}
\end{itemize}

Since $t_l\thadhad$  is the most sensitive signal region in this analysis, and most of the bins in the distributions are blinded. The same sign control region is provided to 
check the agreement between data and the background prediction. The distributions of those variables in this CR are shown in  Figure  \ref{fig:var_reg1l2tau1bnj_ss_1} -  \ref{fig:var_reg1l2tau1bnj_ss}.

The data are in good agreement with the background prediction of the standard model.

%Only statistical uncertainties are being shown in these distributions presented until chapter 12. Underflow and overflow bins are included respectively in the first and last bins. Empty bins in SRs are always blinded based on our strategy.

\input{\FCNCFigures/tex/variables/reg1l1tau1b1j_ss_vetobtagwp70_highmet}
\input{\FCNCFigures/tex/variables/reg1l1tau1b2j_ss_vetobtagwp70_highmet}
\input{\FCNCFigures/tex/variables/reg1l1tau1b2j_os_vetobtagwp70_highmet}
\input{\FCNCFigures/tex/variables/reg1l1tau1b3j_os_vetobtagwp70_highmet}
\input{\FCNCFigures/tex/variables/reg1l2tau1bnj_os}
\input{\FCNCFigures/tex/variables/reg1l2tau1bnj_ss}
\input{\FCNCFigures/tex/variables/reg2mtau1b2jos_vetobtagwp70_highmet}
\input{\FCNCFigures/tex/variables/reg2mtau1b3jos_vetobtagwp70_highmet}


\end{enumerate}
