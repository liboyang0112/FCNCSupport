\subsection{MC fake $\tauhad$ estimation (obsolete)}
\label{sec:sf_method_obsolete}

To estimate the fake tau background from MC, we use the same Data-Driven (DD) Scale Factor (SF) method developed for the ATLAS 
ttH multi-lepton analysis~\cite{ATL-COM-PHYS-2018-410} 
in which the SF is derived in the opposite-sign dilepton + $\tauhad$ ($2lOS$+1$\tauhad$) control region by comparing the rate of observed fake-tau events to 
the MC simulation. The scale factors are parametrized as a function of $\pt$ for 1- and 3-prong $\tauhad$ separately and the final 
SF are summarized in Table~\ref{tab:faketauSF}. The systematics on the scale factors are derived by comparing the values in the nominal control region to
those obtained in either enriched $t\bar t$ or $Z$ boson control regions respectively.

%%% I am here 
\begin{table}[htb]
\caption{ The fake-tau SF measured from the control regions as a function of tau $\pt$ where uncertainty is divided into statistical and 
systematic obtained from control regions enriched in $t\bar t$ and $Z$ + jets. The $\pt$ bins for 1-prong(3-prong) $\tauhad$ are listed separately.}
\centering
\begin{tabular}{|c|c|c|c|} \hline
type of $\tauhad$  & 25-45(25-50) GeV & 45-70(50-75) GeV  & $>$ 70(75) GeV \\ \hline
1-prong & 1.05$\pm$0.04$\pm$0.05 & 0.94$\pm$0.08$\pm$0.21 & 0.64$\pm$0.10$\pm$0.07 \\
3-prong & 1.25$\pm$0.10$\pm$0.41 & 1.30$\pm$0.32$\pm$0.72 & 0.52$\pm$0.30$\pm$0.64 \\ \hline
\end{tabular}
\label{tab:faketauSF}
\end{table}

These SF are then applied to correct the normalization of MC yields in the for both leptonic and hadronic channels. 
To validate the fake-tau estimate, we have compared the leading $\tauhad$ $\pt$ between data and MC prediction using the events in 
the same-sign control region where the di-tau have the same-sign charge (SS).
The distributions are shown in Figure~\ref{fig:lhsf_validation}. The data are in good agreement with MC predictions, 
which indicated the fake tau is well modeled in MC. 
