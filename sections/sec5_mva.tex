\section{MVA analysis}
\label{sec:mva}

In this section, we investigate the sensitivity of probing signal using one of the Multi-Variate Analysis (MVA) methods, the Gradient Boosted Decision Trees (BDT) method~\cite{BDT,BDT2}, with the {\texttt TMVA} software package. The BDT output score is in the range between -1 and 1. The most signal-like events have scores near 1 while the most background-like events have scores near -1.

The signal topology and kinematics are different across all the channels. To maximize the overall sensitivity, separated BDTG trainings are applied to each the signal region. A number of variables as the BDT inputs are used to train and test events in each signal region for maximal signal acceptance and background rejection. They are listed in Table~\ref{tab:importance_tthML} and Table~\ref{tab:importance_xTFW}. The most sensitive variables distributions are shown in Figure \ref{fig:mva_input_hadhad}-\ref{fig:mva_input_lhadhad}

\begin{table}
\caption{The importance (in \%) of each variables used in the BDTG training for leptonic channels, the two numbers in the each block are from the two training folds.}
\label{tab:importance_tthML}
\input{\FCNCTables/Importance/tthML/Merged}
\end{table}


\begin{table}
\caption{The importance (in \%) of each variables used in the BDTG training for hadronic channels, the two numbers in the each block are from the two training folds.}
\label{tab:importance_xTFW}
\input{\FCNCTables/Importance/xTFW/Merged}
\end{table}

The signal and background samples are randomly divided into two equal parts (denoted as even and odd parity events). The BDT is trained with one part, and tested on the other part. It is always ensured that the BDT derived from the training events is not applied to the same events, but only to the independent test ones. The sum of MC all background processes, corrected normalized, are used in the training and testing. %The Gradient BDT parameters used are listed in Table \ref{tab:bdt_pars}. 
With the \texttt{IgnoreNegWeightsInTraining} option, only MC events with positive MC weights are used in the traning. In The hadronic channels, due to the fake is modelled by the not medium control region, the data events in the not medium control region is also used in the training with a weight of 0.3 to take the FFs into account. The comparison of BDT performances in test-odd and test-even samples is given in Figure \ref{fig:overtrain_hadhad}-\ref{fig:overtrain_lhadhad}. The BDT parameters \texttt{NTrees} and \texttt{nCuts} are tuned such that the test-odd and test-even agrees, and the signal sensitivity is optimised.

The importance factors\footnote{
The importance is evaluated as the total separation gain that this variable had in the decision trees (weighted by the number of events). It is normalized to all variables together, which have an importance of 1.
}
of different variables used in the training is listed in Table \ref{tab:importance_tthML} and \ref{tab:importance_xTFW}. The two numbers in each block represent the importance factor of the two models trained from even and odd parts. The consistency of these factors implies that the training models are stable.


\input{\FCNCFigures/tex/BDTinput}

As a cross check, the comparisons between BDT distributions in testing samples, as well as the test even and test odd ROC curves, are shown in Figure \ref{fig:overtrain_hadhad} and \ref{fig:overtrain_lephad}.

\input{\FCNCFigures/tex/BDT}

The statistical only significance based on BDT discriminant is shown in Table \ref{tab:tthML_significance}, \ref{tab:xTFW_significance}.

\begin{table}
\caption{The statistical only significance in leptonic channels based on BDT discriminant.}
\label{tab:tthML_significance}
\centering
\begin{tabular}{|c|c|c|c|c|} \hline
 & 1l1tau1b1j ss e  highmet & 1l1tau1b1j ss e  lowmet & 1l1tau1b1j ss mu  highmet & 1l1tau1b1j ss mu  lowmet\\\hline
$\bar{t}t\to bWcH$ & $0.61$ & $0.23$ & $1.11$ & $0.34$\\\hline
$cg\to tH$ & $0.02$ & $0.01$ & $0.04$ & $0.01$\\\hline
tcH~merged~signal & $0.63$ & $0.24$ & $1.15$ & $0.35$\\\hline
$\bar{t}t\to bWuH$ & $0.64$ & $0.22$ & $1.18$ & $0.39$\\\hline
$ug\to tH$ & $0.05$ & $0.02$ & $0.28$ & $0.09$\\\hline
tuH~merged~signal & $0.70$ & $0.24$ & $1.45$ & $0.48$\\\hline
\end{tabular}
\begin{tabular}{|c|c|c|c|c|} \hline
 & 1l1tau1b2j os e  highmet & 1l1tau1b2j os e  lowmet & 1l1tau1b2j os mu  highmet & 1l1tau1b2j os mu  lowmet\\\hline
$\bar{t}t\to bWcH$ & $0.35$ & $0.13$ & $0.57$ & $0.25$\\\hline
$cg\to tH$ & $0.04$ & $0.01$ & $0.05$ & $0.01$\\\hline
tcH~merged~signal & $0.38$ & $0.14$ & $0.62$ & $0.27$\\\hline
$\bar{t}t\to bWuH$ & $0.36$ & $0.15$ & $0.60$ & $0.23$\\\hline
$ug\to tH$ & $0.21$ & $0.04$ & $0.35$ & $0.07$\\\hline
tuH~merged~signal & $0.55$ & $0.19$ & $0.92$ & $0.30$\\\hline
\end{tabular}
\begin{tabular}{|c|c|c|c|c|} \hline
 & 1l1tau1b2j ss e  highmet & 1l1tau1b2j ss e  lowmet & 1l1tau1b2j ss mu  highmet & 1l1tau1b2j ss mu  lowmet\\\hline
$\bar{t}t\to bWcH$ & $0.59$ & $0.19$ & $1.07$ & $0.35$\\\hline
$cg\to tH$ & $0.02$ & $0.00$ & $0.03$ & $0.01$\\\hline
tcH~merged~signal & $0.61$ & $0.20$ & $1.10$ & $0.36$\\\hline
$\bar{t}t\to bWuH$ & $0.64$ & $0.23$ & $1.13$ & $0.43$\\\hline
$ug\to tH$ & $0.03$ & $0.01$ & $0.23$ & $0.08$\\\hline
tuH~merged~signal & $0.68$ & $0.24$ & $1.35$ & $0.51$\\\hline
\end{tabular}
\begin{tabular}{|c|c|c|c|c|} \hline
 & 1l1tau1b3j os e  highmet & 1l1tau1b3j os e  lowmet & 1l1tau1b3j os mu  highmet & 1l1tau1b3j os mu  lowmet\\\hline
$\bar{t}t\to bWcH$ & $0.76$ & $0.28$ & $1.18$ & $0.50$\\\hline
$cg\to tH$ & $0.04$ & $0.01$ & $0.05$ & $0.02$\\\hline
tcH~merged~signal & $0.79$ & $0.29$ & $1.23$ & $0.51$\\\hline
$\bar{t}t\to bWuH$ & $0.77$ & $0.34$ & $1.28$ & $0.51$\\\hline
$ug\to tH$ & $0.22$ & $0.04$ & $0.30$ & $0.08$\\\hline
tuH~merged~signal & $0.98$ & $0.38$ & $1.58$ & $0.59$\\\hline
\end{tabular}
\begin{tabular}{|c|c|c|c|c|} \hline
 & 1l1tau1b3j ss e  highmet & 1l1tau1b3j ss e  lowmet & 1l1tau1b3j ss mu  highmet & 1l1tau1b3j ss mu  lowmet\\\hline
$\bar{t}t\to bWcH$ & $0.29$ & $0.09$ & $0.48$ & $0.16$\\\hline
$cg\to tH$ & $0.01$ & $0.00$ & $0.01$ & $0.00$\\\hline
tcH~merged~signal & $0.30$ & $0.09$ & $0.49$ & $0.17$\\\hline
$\bar{t}t\to bWuH$ & $0.31$ & $0.10$ & $0.54$ & $0.18$\\\hline
$ug\to tH$ & $0.03$ & $0.01$ & $0.06$ & $0.02$\\\hline
tuH~merged~signal & $0.33$ & $0.11$ & $0.60$ & $0.20$\\\hline
\end{tabular}
\begin{tabular}{|c|c|c|c|c|} \hline
 & 1l1tau2b1j ss e  highmet & 1l1tau2b1j ss e  lowmet & 1l1tau2b1j ss mu  highmet & 1l1tau2b1j ss mu  lowmet\\\hline
$\bar{t}t\to bWcH$ & $0.12$ & $0.07$ & $0.20$ & $0.06$\\\hline
$cg\to tH$ & $0.00$ & $0.00$ & $0.00$ & $0.00$\\\hline
tcH~merged~signal & $0.12$ & $0.07$ & $0.21$ & $0.06$\\\hline
$\bar{t}t\to bWuH$ & $0.04$ & $0.01$ & $0.08$ & $0.02$\\\hline
$ug\to tH$ & $0.01$ & $0.00$ & $0.02$ & $0.01$\\\hline
tuH~merged~signal & $0.04$ & $0.01$ & $0.10$ & $0.03$\\\hline
\end{tabular}
\begin{tabular}{|c|c|c|c|c|} \hline
 & 1l1tau2b2j os e  highmet & 1l1tau2b2j os e  lowmet & 1l1tau2b2j os mu  highmet & 1l1tau2b2j os mu  lowmet\\\hline
$\bar{t}t\to bWcH$ & $0.10$ & $0.03$ & $0.14$ & $0.07$\\\hline
$cg\to tH$ & $0.00$ & $0.00$ & $0.01$ & $0.00$\\\hline
tcH~merged~signal & $0.10$ & $0.03$ & $0.15$ & $0.07$\\\hline
$\bar{t}t\to bWuH$ & $0.03$ & $0.02$ & $0.07$ & $0.02$\\\hline
$ug\to tH$ & $0.02$ & $0.02$ & $0.03$ & $0.01$\\\hline
tuH~merged~signal & $0.04$ & $0.03$ & $0.10$ & $0.03$\\\hline
\end{tabular}
\begin{tabular}{|c|c|c|c|c|} \hline
 & 1l1tau2b2j ss e  highmet & 1l1tau2b2j ss e  lowmet & 1l1tau2b2j ss mu  highmet & 1l1tau2b2j ss mu  lowmet\\\hline
$\bar{t}t\to bWcH$ & $0.07$ & $0.06$ & $0.14$ & $0.04$\\\hline
$cg\to tH$ & $0.00$ & $0.00$ & $0.00$ & $0.00$\\\hline
tcH~merged~signal & $0.07$ & $0.06$ & $0.14$ & $0.04$\\\hline
$\bar{t}t\to bWuH$ & $0.04$ & $0.01$ & $0.07$ & $0.02$\\\hline
$ug\to tH$ & $0.00$ & $0.00$ & $0.02$ & $0.01$\\\hline
tuH~merged~signal & $0.05$ & $0.01$ & $0.08$ & $0.03$\\\hline
\end{tabular}
\begin{tabular}{|c|c|c|c|c|} \hline
 & 1l1tau2b3j os e  highmet & 1l1tau2b3j os e  lowmet & 1l1tau2b3j os mu  highmet & 1l1tau2b3j os mu  lowmet\\\hline
$\bar{t}t\to bWcH$ & $0.14$ & $0.03$ & $0.20$ & $0.10$\\\hline
$cg\to tH$ & $0.00$ & $0.00$ & $0.00$ & $0.00$\\\hline
tcH~merged~signal & $0.14$ & $0.03$ & $0.21$ & $0.10$\\\hline
$\bar{t}t\to bWuH$ & $0.07$ & $0.03$ & $0.10$ & $0.04$\\\hline
$ug\to tH$ & $0.01$ & $0.00$ & $0.03$ & $0.01$\\\hline
tuH~merged~signal & $0.08$ & $0.03$ & $0.13$ & $0.04$\\\hline
\end{tabular}
\begin{tabular}{|c|c|c|c|c|} \hline
 & 1l1tau2b3j ss e  highmet & 1l1tau2b3j ss e  lowmet & 1l1tau2b3j ss mu  highmet & 1l1tau2b3j ss mu  lowmet\\\hline
$\bar{t}t\to bWcH$ & $0.04$ & $0.01$ & $0.08$ & $0.02$\\\hline
$cg\to tH$ & $0.00$ & $0.00$ & $0.00$ & $0.00$\\\hline
tcH~merged~signal & $0.04$ & $0.01$ & $0.08$ & $0.02$\\\hline
$\bar{t}t\to bWuH$ & $0.03$ &  / & $0.04$ & $0.02$\\\hline
$ug\to tH$ & $0.00$ &  / & $0.01$ & $0.00$\\\hline
tuH~merged~signal & $0.03$ &  / & $0.04$ & $0.02$\\\hline
\end{tabular}
\begin{tabular}{|c|c|c|c|c|} \hline
 & 1l2tau1bnj os e  highmet & 1l2tau1bnj os e  lowmet & 1l2tau1bnj os mu  highmet & 1l2tau1bnj os mu  lowmet\\\hline
$\bar{t}t\to bWcH$ & $3.27$ & $1.47$ & $4.80$ & $2.71$\\\hline
$cg\to tH$ & $0.37$ & $0.12$ & $0.47$ & $0.26$\\\hline
tcH~merged~signal & $3.47$ & $1.55$ & $5.13$ & $2.88$\\\hline
$\bar{t}t\to bWuH$ & $3.55$ & $1.67$ & $5.04$ & $2.89$\\\hline
$ug\to tH$ & $0.65$ & $0.27$ & $2.30$ & $1.28$\\\hline
tuH~merged~signal & $3.92$ & $1.86$ & $6.76$ & $3.80$\\\hline
\end{tabular}
\begin{tabular}{|c|c|c|c|c|} \hline
 & 1l2tau1bnj ss e  highmet & 1l2tau1bnj ss e  lowmet & 1l2tau1bnj ss mu  highmet & 1l2tau1bnj ss mu  lowmet\\\hline
$\bar{t}t\to bWcH$ & $0.13$ & $0.15$ & $0.22$ & $0.14$\\\hline
$cg\to tH$ & $0.01$ & $0.01$ & $0.01$ & $0.01$\\\hline
tcH~merged~signal & $0.14$ & $0.16$ & $0.23$ & $0.14$\\\hline
$\bar{t}t\to bWuH$ & $0.18$ & $0.15$ & $0.24$ & $0.18$\\\hline
$ug\to tH$ & $0.05$ & $0.05$ & $0.07$ & $0.04$\\\hline
tuH~merged~signal & $0.22$ & $0.18$ & $0.31$ & $0.21$\\\hline
\end{tabular}
\begin{tabular}{|c|c|c|c|c|} \hline
 & 1l2tau2bnj os e  highmet & 1l2tau2bnj os e  lowmet & 1l2tau2bnj os mu  highmet & 1l2tau2bnj os mu  lowmet\\\hline
$\bar{t}t\to bWcH$ & $0.50$ & $0.60$ & $0.73$ & $0.38$\\\hline
$cg\to tH$ & $0.01$ & $0.02$ & $0.02$ & $0.01$\\\hline
tcH~merged~signal & $0.51$ & $0.61$ & $0.75$ & $0.39$\\\hline
$\bar{t}t\to bWuH$ & $0.15$ & $0.04$ & $0.18$ & $0.10$\\\hline
$ug\to tH$ & $0.01$ & $0.01$ & $0.07$ & $0.02$\\\hline
tuH~merged~signal & $0.16$ & $0.04$ & $0.25$ & $0.12$\\\hline
\end{tabular}
\begin{tabular}{|c|c|c|c|c|} \hline
 & 1l2tau2bnj ss e  highmet & 1l2tau2bnj ss mu  highmet & 1l2tau2bnj ss mu  lowmet & 1l1tau1b1j ss  highmet\\\hline
$\bar{t}t\to bWcH$ & $0.02$ & $0.03$ &  / & $1.26$\\\hline
$cg\to tH$ & $0.00$ & $0.01$ & $0.00$ & $0.05$\\\hline
tcH~merged~signal & $0.02$ & $0.03$ & $0.00$ & $1.31$\\\hline
$\bar{t}t\to bWuH$ & $0.01$ & $0.01$ &  / & $1.34$\\\hline
$ug\to tH$ &  / & $0.00$ &  / & $0.27$\\\hline
tuH~merged~signal & $0.01$ & $0.01$ &  / & $1.60$\\\hline
\end{tabular}
\begin{tabular}{|c|c|c|c|c|} \hline
 & 1l1tau1b1j ss  lowmet & 1l1tau1b2j os  highmet & 1l1tau1b2j os  lowmet & 1l1tau1b2j ss  highmet\\\hline
$\bar{t}t\to bWcH$ & $0.41$ & $0.67$ & $0.28$ & $1.22$\\\hline
$cg\to tH$ & $0.02$ & $0.06$ & $0.02$ & $0.03$\\\hline
tcH~merged~signal & $0.42$ & $0.73$ & $0.30$ & $1.26$\\\hline
$\bar{t}t\to bWuH$ & $0.45$ & $0.70$ & $0.28$ & $1.30$\\\hline
$ug\to tH$ & $0.09$ & $0.40$ & $0.08$ & $0.21$\\\hline
tuH~merged~signal & $0.53$ & $1.07$ & $0.35$ & $1.50$\\\hline
\end{tabular}
\begin{tabular}{|c|c|c|c|c|} \hline
 & 1l1tau1b2j ss  lowmet & 1l1tau1b3j os  highmet & 1l1tau1b3j os  lowmet & 1l1tau1b3j ss  highmet\\\hline
$\bar{t}t\to bWcH$ & $0.39$ & $1.40$ & $0.57$ & $0.56$\\\hline
$cg\to tH$ & $0.01$ & $0.06$ & $0.02$ & $0.01$\\\hline
tcH~merged~signal & $0.40$ & $1.46$ & $0.59$ & $0.57$\\\hline
$\bar{t}t\to bWuH$ & $0.48$ & $1.50$ & $0.61$ & $0.62$\\\hline
$ug\to tH$ & $0.07$ & $0.37$ & $0.09$ & $0.06$\\\hline
tuH~merged~signal & $0.55$ & $1.85$ & $0.70$ & $0.68$\\\hline
\end{tabular}
\begin{tabular}{|c|c|c|c|c|} \hline
 & 1l1tau1b3j ss  lowmet & 1l1tau2b1j ss  highmet & 1l1tau2b1j ss  lowmet & 1l1tau2b2j os  highmet\\\hline
$\bar{t}t\to bWcH$ & $0.19$ & $0.23$ & $0.08$ & $0.17$\\\hline
$cg\to tH$ & $0.00$ & $0.00$ & $0.00$ & $0.01$\\\hline
tcH~merged~signal & $0.19$ & $0.24$ & $0.08$ & $0.18$\\\hline
$\bar{t}t\to bWuH$ & $0.21$ & $0.09$ & $0.02$ & $0.07$\\\hline
$ug\to tH$ & $0.02$ & $0.02$ & $0.01$ & $0.04$\\\hline
tuH~merged~signal & $0.22$ & $0.11$ & $0.03$ & $0.11$\\\hline
\end{tabular}
\begin{tabular}{|c|c|c|c|c|} \hline
 & 1l1tau2b2j os  lowmet & 1l1tau2b2j ss  highmet & 1l1tau2b2j ss  lowmet & 1l1tau2b3j os  highmet\\\hline
$\bar{t}t\to bWcH$ & $0.07$ & $0.16$ & $0.06$ & $0.25$\\\hline
$cg\to tH$ & $0.00$ & $0.00$ & $0.00$ & $0.00$\\\hline
tcH~merged~signal & $0.07$ & $0.16$ & $0.06$ & $0.25$\\\hline
$\bar{t}t\to bWuH$ & $0.03$ & $0.08$ & $0.02$ & $0.13$\\\hline
$ug\to tH$ & $0.01$ & $0.02$ & $0.01$ & $0.03$\\\hline
tuH~merged~signal & $0.04$ & $0.09$ & $0.03$ & $0.15$\\\hline
\end{tabular}
\begin{tabular}{|c|c|c|c|c|} \hline
 & 1l1tau2b3j os  lowmet & 1l1tau2b3j ss  highmet & 1l1tau2b3j ss  lowmet & 1l2tau1bnj os  highmet\\\hline
$\bar{t}t\to bWcH$ & $0.10$ & $0.09$ & $0.02$ & $5.67$\\\hline
$cg\to tH$ & $0.00$ & $0.00$ & $0.00$ & $0.55$\\\hline
tcH~merged~signal & $0.10$ & $0.09$ & $0.02$ & $6.05$\\\hline
$\bar{t}t\to bWuH$ & $0.04$ & $0.05$ & $0.01$ & $6.01$\\\hline
$ug\to tH$ & $0.00$ & $0.01$ & $0.00$ & $2.33$\\\hline
tuH~merged~signal & $0.05$ & $0.05$ & $0.02$ & $7.71$\\\hline
\end{tabular}
\begin{tabular}{|c|c|c|c|c|} \hline
 & 1l2tau1bnj os  lowmet & 1l2tau1bnj ss  highmet & 1l2tau1bnj ss  lowmet & 1l2tau2bnj os  highmet\\\hline
$\bar{t}t\to bWcH$ & $3.00$ & $0.25$ & $0.19$ & $0.87$\\\hline
$cg\to tH$ & $0.27$ & $0.02$ & $0.01$ & $0.02$\\\hline
tcH~merged~signal & $3.19$ & $0.27$ & $0.19$ & $0.89$\\\hline
$\bar{t}t\to bWuH$ & $3.24$ & $0.29$ & $0.21$ & $0.22$\\\hline
$ug\to tH$ & $1.19$ & $0.09$ & $0.05$ & $0.06$\\\hline
tuH~merged~signal & $4.11$ & $0.37$ & $0.25$ & $0.29$\\\hline
\end{tabular}
\begin{tabular}{|c|c|c|c|c|} \hline
 & 1l2tau2bnj os  lowmet & 1l2tau2bnj ss  highmet & 1l2tau2bnj ss  lowmet & 1l1tau1b1j ss\\\hline
$\bar{t}t\to bWcH$ & $0.44$ & $0.03$ &  / & $1.33$\\\hline
$cg\to tH$ & $0.01$ & $0.00$ & $0.00$ & $0.05$\\\hline
tcH~merged~signal & $0.45$ & $0.03$ & $0.00$ & $1.37$\\\hline
$\bar{t}t\to bWuH$ & $0.10$ & $0.01$ &  / & $1.41$\\\hline
$ug\to tH$ & $0.02$ & $0.00$ &  / & $0.28$\\\hline
tuH~merged~signal & $0.12$ & $0.01$ &  / & $1.69$\\\hline
\end{tabular}
\begin{tabular}{|c|c|c|c|c|} \hline
 & 1l1tau1b2j os & 1l1tau1b2j ss & 1l1tau1b3j os & 1l1tau1b3j ss\\\hline
$\bar{t}t\to bWcH$ & $0.72$ & $1.28$ & $1.50$ & $0.59$\\\hline
$cg\to tH$ & $0.07$ & $0.04$ & $0.07$ & $0.01$\\\hline
tcH~merged~signal & $0.78$ & $1.32$ & $1.57$ & $0.60$\\\hline
$\bar{t}t\to bWuH$ & $0.74$ & $1.38$ & $1.61$ & $0.65$\\\hline
$ug\to tH$ & $0.40$ & $0.22$ & $0.38$ & $0.06$\\\hline
tuH~merged~signal & $1.11$ & $1.59$ & $1.98$ & $0.71$\\\hline
\end{tabular}
\begin{tabular}{|c|c|c|c|} \hline
 & 1l1tau2b1j ss & 1l1tau2b2j os & 1l1tau2b2j ss\\\hline
$\bar{t}t\to bWcH$ & $0.25$ & $0.18$ & $0.17$\\\hline
$cg\to tH$ & $0.00$ & $0.01$ & $0.00$\\\hline
tcH~merged~signal & $0.25$ & $0.19$ & $0.17$\\\hline
$\bar{t}t\to bWuH$ & $0.09$ & $0.08$ & $0.08$\\\hline
$ug\to tH$ & $0.02$ & $0.04$ & $0.02$\\\hline
tuH~merged~signal & $0.11$ & $0.11$ & $0.10$\\\hline
\end{tabular}
\begin{tabular}{|c|c|c|c|} \hline
 & 1l1tau2b3j os & 1l1tau2b3j ss & 1l2tau1bnj os\\\hline
$\bar{t}t\to bWcH$ & $0.26$ & $0.09$ & $6.34$\\\hline
$cg\to tH$ & $0.00$ & $0.00$ & $0.61$\\\hline
tcH~merged~signal & $0.27$ & $0.09$ & $6.77$\\\hline
$\bar{t}t\to bWuH$ & $0.13$ & $0.05$ & $6.74$\\\hline
$ug\to tH$ & $0.03$ & $0.01$ & $2.59$\\\hline
tuH~merged~signal & $0.16$ & $0.05$ & $8.64$\\\hline
\end{tabular}
\begin{tabular}{|c|c|c|c|} \hline
 & 1l2tau1bnj ss & 1l2tau2bnj os & 1l2tau2bnj ss\\\hline
$\bar{t}t\to bWcH$ & $0.30$ & $0.96$ & $0.03$\\\hline
$cg\to tH$ & $0.02$ & $0.02$ & $0.00$\\\hline
tcH~merged~signal & $0.31$ & $0.98$ & $0.03$\\\hline
$\bar{t}t\to bWuH$ & $0.35$ & $0.24$ & $0.01$\\\hline
$ug\to tH$ & $0.09$ & $0.06$ & $0.00$\\\hline
tuH~merged~signal & $0.43$ & $0.30$ & $0.01$\\\hline
\end{tabular}

\end{table}

\begin{table}
\caption{The statistical only significance in hadronic channels based on BDT discriminant.}
\label{tab:xTFW_significance}
\centering
\begin{tabular}{|c|c|c|c|c|} \hline
 & 1l1tau1b1j ss e  highmet & 1l1tau1b1j ss e  lowmet & 1l1tau1b1j ss mu  highmet & 1l1tau1b1j ss mu  lowmet\\\hline
$\bar{t}t\to bWcH$ & $0.61$ & $0.23$ & $1.11$ & $0.34$\\\hline
$cg\to tH$ & $0.02$ & $0.01$ & $0.04$ & $0.01$\\\hline
tcH~merged~signal & $0.63$ & $0.24$ & $1.15$ & $0.35$\\\hline
$\bar{t}t\to bWuH$ & $0.64$ & $0.22$ & $1.18$ & $0.39$\\\hline
$ug\to tH$ & $0.05$ & $0.02$ & $0.28$ & $0.09$\\\hline
tuH~merged~signal & $0.70$ & $0.24$ & $1.45$ & $0.48$\\\hline
\end{tabular}
\begin{tabular}{|c|c|c|c|c|} \hline
 & 1l1tau1b2j os e  highmet & 1l1tau1b2j os e  lowmet & 1l1tau1b2j os mu  highmet & 1l1tau1b2j os mu  lowmet\\\hline
$\bar{t}t\to bWcH$ & $0.35$ & $0.13$ & $0.57$ & $0.25$\\\hline
$cg\to tH$ & $0.04$ & $0.01$ & $0.05$ & $0.01$\\\hline
tcH~merged~signal & $0.38$ & $0.14$ & $0.62$ & $0.27$\\\hline
$\bar{t}t\to bWuH$ & $0.36$ & $0.15$ & $0.60$ & $0.23$\\\hline
$ug\to tH$ & $0.21$ & $0.04$ & $0.35$ & $0.07$\\\hline
tuH~merged~signal & $0.55$ & $0.19$ & $0.92$ & $0.30$\\\hline
\end{tabular}
\begin{tabular}{|c|c|c|c|c|} \hline
 & 1l1tau1b2j ss e  highmet & 1l1tau1b2j ss e  lowmet & 1l1tau1b2j ss mu  highmet & 1l1tau1b2j ss mu  lowmet\\\hline
$\bar{t}t\to bWcH$ & $0.59$ & $0.19$ & $1.07$ & $0.35$\\\hline
$cg\to tH$ & $0.02$ & $0.00$ & $0.03$ & $0.01$\\\hline
tcH~merged~signal & $0.61$ & $0.20$ & $1.10$ & $0.36$\\\hline
$\bar{t}t\to bWuH$ & $0.64$ & $0.23$ & $1.13$ & $0.43$\\\hline
$ug\to tH$ & $0.03$ & $0.01$ & $0.23$ & $0.08$\\\hline
tuH~merged~signal & $0.68$ & $0.24$ & $1.35$ & $0.51$\\\hline
\end{tabular}
\begin{tabular}{|c|c|c|c|c|} \hline
 & 1l1tau1b3j os e  highmet & 1l1tau1b3j os e  lowmet & 1l1tau1b3j os mu  highmet & 1l1tau1b3j os mu  lowmet\\\hline
$\bar{t}t\to bWcH$ & $0.76$ & $0.28$ & $1.18$ & $0.50$\\\hline
$cg\to tH$ & $0.04$ & $0.01$ & $0.05$ & $0.02$\\\hline
tcH~merged~signal & $0.79$ & $0.29$ & $1.23$ & $0.51$\\\hline
$\bar{t}t\to bWuH$ & $0.77$ & $0.34$ & $1.28$ & $0.51$\\\hline
$ug\to tH$ & $0.22$ & $0.04$ & $0.30$ & $0.08$\\\hline
tuH~merged~signal & $0.98$ & $0.38$ & $1.58$ & $0.59$\\\hline
\end{tabular}
\begin{tabular}{|c|c|c|c|c|} \hline
 & 1l1tau1b3j ss e  highmet & 1l1tau1b3j ss e  lowmet & 1l1tau1b3j ss mu  highmet & 1l1tau1b3j ss mu  lowmet\\\hline
$\bar{t}t\to bWcH$ & $0.29$ & $0.09$ & $0.48$ & $0.16$\\\hline
$cg\to tH$ & $0.01$ & $0.00$ & $0.01$ & $0.00$\\\hline
tcH~merged~signal & $0.30$ & $0.09$ & $0.49$ & $0.17$\\\hline
$\bar{t}t\to bWuH$ & $0.31$ & $0.10$ & $0.54$ & $0.18$\\\hline
$ug\to tH$ & $0.03$ & $0.01$ & $0.06$ & $0.02$\\\hline
tuH~merged~signal & $0.33$ & $0.11$ & $0.60$ & $0.20$\\\hline
\end{tabular}
\begin{tabular}{|c|c|c|c|c|} \hline
 & 1l1tau2b1j ss e  highmet & 1l1tau2b1j ss e  lowmet & 1l1tau2b1j ss mu  highmet & 1l1tau2b1j ss mu  lowmet\\\hline
$\bar{t}t\to bWcH$ & $0.12$ & $0.07$ & $0.20$ & $0.06$\\\hline
$cg\to tH$ & $0.00$ & $0.00$ & $0.00$ & $0.00$\\\hline
tcH~merged~signal & $0.12$ & $0.07$ & $0.21$ & $0.06$\\\hline
$\bar{t}t\to bWuH$ & $0.04$ & $0.01$ & $0.08$ & $0.02$\\\hline
$ug\to tH$ & $0.01$ & $0.00$ & $0.02$ & $0.01$\\\hline
tuH~merged~signal & $0.04$ & $0.01$ & $0.10$ & $0.03$\\\hline
\end{tabular}
\begin{tabular}{|c|c|c|c|c|} \hline
 & 1l1tau2b2j os e  highmet & 1l1tau2b2j os e  lowmet & 1l1tau2b2j os mu  highmet & 1l1tau2b2j os mu  lowmet\\\hline
$\bar{t}t\to bWcH$ & $0.10$ & $0.03$ & $0.14$ & $0.07$\\\hline
$cg\to tH$ & $0.00$ & $0.00$ & $0.01$ & $0.00$\\\hline
tcH~merged~signal & $0.10$ & $0.03$ & $0.15$ & $0.07$\\\hline
$\bar{t}t\to bWuH$ & $0.03$ & $0.02$ & $0.07$ & $0.02$\\\hline
$ug\to tH$ & $0.02$ & $0.02$ & $0.03$ & $0.01$\\\hline
tuH~merged~signal & $0.04$ & $0.03$ & $0.10$ & $0.03$\\\hline
\end{tabular}
\begin{tabular}{|c|c|c|c|c|} \hline
 & 1l1tau2b2j ss e  highmet & 1l1tau2b2j ss e  lowmet & 1l1tau2b2j ss mu  highmet & 1l1tau2b2j ss mu  lowmet\\\hline
$\bar{t}t\to bWcH$ & $0.07$ & $0.06$ & $0.14$ & $0.04$\\\hline
$cg\to tH$ & $0.00$ & $0.00$ & $0.00$ & $0.00$\\\hline
tcH~merged~signal & $0.07$ & $0.06$ & $0.14$ & $0.04$\\\hline
$\bar{t}t\to bWuH$ & $0.04$ & $0.01$ & $0.07$ & $0.02$\\\hline
$ug\to tH$ & $0.00$ & $0.00$ & $0.02$ & $0.01$\\\hline
tuH~merged~signal & $0.05$ & $0.01$ & $0.08$ & $0.03$\\\hline
\end{tabular}
\begin{tabular}{|c|c|c|c|c|} \hline
 & 1l1tau2b3j os e  highmet & 1l1tau2b3j os e  lowmet & 1l1tau2b3j os mu  highmet & 1l1tau2b3j os mu  lowmet\\\hline
$\bar{t}t\to bWcH$ & $0.14$ & $0.03$ & $0.20$ & $0.10$\\\hline
$cg\to tH$ & $0.00$ & $0.00$ & $0.00$ & $0.00$\\\hline
tcH~merged~signal & $0.14$ & $0.03$ & $0.21$ & $0.10$\\\hline
$\bar{t}t\to bWuH$ & $0.07$ & $0.03$ & $0.10$ & $0.04$\\\hline
$ug\to tH$ & $0.01$ & $0.00$ & $0.03$ & $0.01$\\\hline
tuH~merged~signal & $0.08$ & $0.03$ & $0.13$ & $0.04$\\\hline
\end{tabular}
\begin{tabular}{|c|c|c|c|c|} \hline
 & 1l1tau2b3j ss e  highmet & 1l1tau2b3j ss e  lowmet & 1l1tau2b3j ss mu  highmet & 1l1tau2b3j ss mu  lowmet\\\hline
$\bar{t}t\to bWcH$ & $0.04$ & $0.01$ & $0.08$ & $0.02$\\\hline
$cg\to tH$ & $0.00$ & $0.00$ & $0.00$ & $0.00$\\\hline
tcH~merged~signal & $0.04$ & $0.01$ & $0.08$ & $0.02$\\\hline
$\bar{t}t\to bWuH$ & $0.03$ &  / & $0.04$ & $0.02$\\\hline
$ug\to tH$ & $0.00$ &  / & $0.01$ & $0.00$\\\hline
tuH~merged~signal & $0.03$ &  / & $0.04$ & $0.02$\\\hline
\end{tabular}
\begin{tabular}{|c|c|c|c|c|} \hline
 & 1l2tau1bnj os e  highmet & 1l2tau1bnj os e  lowmet & 1l2tau1bnj os mu  highmet & 1l2tau1bnj os mu  lowmet\\\hline
$\bar{t}t\to bWcH$ & $3.27$ & $1.47$ & $4.80$ & $2.71$\\\hline
$cg\to tH$ & $0.37$ & $0.12$ & $0.47$ & $0.26$\\\hline
tcH~merged~signal & $3.47$ & $1.55$ & $5.13$ & $2.88$\\\hline
$\bar{t}t\to bWuH$ & $3.55$ & $1.67$ & $5.04$ & $2.89$\\\hline
$ug\to tH$ & $0.65$ & $0.27$ & $2.30$ & $1.28$\\\hline
tuH~merged~signal & $3.92$ & $1.86$ & $6.76$ & $3.80$\\\hline
\end{tabular}
\begin{tabular}{|c|c|c|c|c|} \hline
 & 1l2tau1bnj ss e  highmet & 1l2tau1bnj ss e  lowmet & 1l2tau1bnj ss mu  highmet & 1l2tau1bnj ss mu  lowmet\\\hline
$\bar{t}t\to bWcH$ & $0.13$ & $0.15$ & $0.22$ & $0.14$\\\hline
$cg\to tH$ & $0.01$ & $0.01$ & $0.01$ & $0.01$\\\hline
tcH~merged~signal & $0.14$ & $0.16$ & $0.23$ & $0.14$\\\hline
$\bar{t}t\to bWuH$ & $0.18$ & $0.15$ & $0.24$ & $0.18$\\\hline
$ug\to tH$ & $0.05$ & $0.05$ & $0.07$ & $0.04$\\\hline
tuH~merged~signal & $0.22$ & $0.18$ & $0.31$ & $0.21$\\\hline
\end{tabular}
\begin{tabular}{|c|c|c|c|c|} \hline
 & 1l2tau2bnj os e  highmet & 1l2tau2bnj os e  lowmet & 1l2tau2bnj os mu  highmet & 1l2tau2bnj os mu  lowmet\\\hline
$\bar{t}t\to bWcH$ & $0.50$ & $0.60$ & $0.73$ & $0.38$\\\hline
$cg\to tH$ & $0.01$ & $0.02$ & $0.02$ & $0.01$\\\hline
tcH~merged~signal & $0.51$ & $0.61$ & $0.75$ & $0.39$\\\hline
$\bar{t}t\to bWuH$ & $0.15$ & $0.04$ & $0.18$ & $0.10$\\\hline
$ug\to tH$ & $0.01$ & $0.01$ & $0.07$ & $0.02$\\\hline
tuH~merged~signal & $0.16$ & $0.04$ & $0.25$ & $0.12$\\\hline
\end{tabular}
\begin{tabular}{|c|c|c|c|c|} \hline
 & 1l2tau2bnj ss e  highmet & 1l2tau2bnj ss mu  highmet & 1l2tau2bnj ss mu  lowmet & 1l1tau1b1j ss  highmet\\\hline
$\bar{t}t\to bWcH$ & $0.02$ & $0.03$ &  / & $1.26$\\\hline
$cg\to tH$ & $0.00$ & $0.01$ & $0.00$ & $0.05$\\\hline
tcH~merged~signal & $0.02$ & $0.03$ & $0.00$ & $1.31$\\\hline
$\bar{t}t\to bWuH$ & $0.01$ & $0.01$ &  / & $1.34$\\\hline
$ug\to tH$ &  / & $0.00$ &  / & $0.27$\\\hline
tuH~merged~signal & $0.01$ & $0.01$ &  / & $1.60$\\\hline
\end{tabular}
\begin{tabular}{|c|c|c|c|c|} \hline
 & 1l1tau1b1j ss  lowmet & 1l1tau1b2j os  highmet & 1l1tau1b2j os  lowmet & 1l1tau1b2j ss  highmet\\\hline
$\bar{t}t\to bWcH$ & $0.41$ & $0.67$ & $0.28$ & $1.22$\\\hline
$cg\to tH$ & $0.02$ & $0.06$ & $0.02$ & $0.03$\\\hline
tcH~merged~signal & $0.42$ & $0.73$ & $0.30$ & $1.26$\\\hline
$\bar{t}t\to bWuH$ & $0.45$ & $0.70$ & $0.28$ & $1.30$\\\hline
$ug\to tH$ & $0.09$ & $0.40$ & $0.08$ & $0.21$\\\hline
tuH~merged~signal & $0.53$ & $1.07$ & $0.35$ & $1.50$\\\hline
\end{tabular}
\begin{tabular}{|c|c|c|c|c|} \hline
 & 1l1tau1b2j ss  lowmet & 1l1tau1b3j os  highmet & 1l1tau1b3j os  lowmet & 1l1tau1b3j ss  highmet\\\hline
$\bar{t}t\to bWcH$ & $0.39$ & $1.40$ & $0.57$ & $0.56$\\\hline
$cg\to tH$ & $0.01$ & $0.06$ & $0.02$ & $0.01$\\\hline
tcH~merged~signal & $0.40$ & $1.46$ & $0.59$ & $0.57$\\\hline
$\bar{t}t\to bWuH$ & $0.48$ & $1.50$ & $0.61$ & $0.62$\\\hline
$ug\to tH$ & $0.07$ & $0.37$ & $0.09$ & $0.06$\\\hline
tuH~merged~signal & $0.55$ & $1.85$ & $0.70$ & $0.68$\\\hline
\end{tabular}
\begin{tabular}{|c|c|c|c|c|} \hline
 & 1l1tau1b3j ss  lowmet & 1l1tau2b1j ss  highmet & 1l1tau2b1j ss  lowmet & 1l1tau2b2j os  highmet\\\hline
$\bar{t}t\to bWcH$ & $0.19$ & $0.23$ & $0.08$ & $0.17$\\\hline
$cg\to tH$ & $0.00$ & $0.00$ & $0.00$ & $0.01$\\\hline
tcH~merged~signal & $0.19$ & $0.24$ & $0.08$ & $0.18$\\\hline
$\bar{t}t\to bWuH$ & $0.21$ & $0.09$ & $0.02$ & $0.07$\\\hline
$ug\to tH$ & $0.02$ & $0.02$ & $0.01$ & $0.04$\\\hline
tuH~merged~signal & $0.22$ & $0.11$ & $0.03$ & $0.11$\\\hline
\end{tabular}
\begin{tabular}{|c|c|c|c|c|} \hline
 & 1l1tau2b2j os  lowmet & 1l1tau2b2j ss  highmet & 1l1tau2b2j ss  lowmet & 1l1tau2b3j os  highmet\\\hline
$\bar{t}t\to bWcH$ & $0.07$ & $0.16$ & $0.06$ & $0.25$\\\hline
$cg\to tH$ & $0.00$ & $0.00$ & $0.00$ & $0.00$\\\hline
tcH~merged~signal & $0.07$ & $0.16$ & $0.06$ & $0.25$\\\hline
$\bar{t}t\to bWuH$ & $0.03$ & $0.08$ & $0.02$ & $0.13$\\\hline
$ug\to tH$ & $0.01$ & $0.02$ & $0.01$ & $0.03$\\\hline
tuH~merged~signal & $0.04$ & $0.09$ & $0.03$ & $0.15$\\\hline
\end{tabular}
\begin{tabular}{|c|c|c|c|c|} \hline
 & 1l1tau2b3j os  lowmet & 1l1tau2b3j ss  highmet & 1l1tau2b3j ss  lowmet & 1l2tau1bnj os  highmet\\\hline
$\bar{t}t\to bWcH$ & $0.10$ & $0.09$ & $0.02$ & $5.67$\\\hline
$cg\to tH$ & $0.00$ & $0.00$ & $0.00$ & $0.55$\\\hline
tcH~merged~signal & $0.10$ & $0.09$ & $0.02$ & $6.05$\\\hline
$\bar{t}t\to bWuH$ & $0.04$ & $0.05$ & $0.01$ & $6.01$\\\hline
$ug\to tH$ & $0.00$ & $0.01$ & $0.00$ & $2.33$\\\hline
tuH~merged~signal & $0.05$ & $0.05$ & $0.02$ & $7.71$\\\hline
\end{tabular}
\begin{tabular}{|c|c|c|c|c|} \hline
 & 1l2tau1bnj os  lowmet & 1l2tau1bnj ss  highmet & 1l2tau1bnj ss  lowmet & 1l2tau2bnj os  highmet\\\hline
$\bar{t}t\to bWcH$ & $3.00$ & $0.25$ & $0.19$ & $0.87$\\\hline
$cg\to tH$ & $0.27$ & $0.02$ & $0.01$ & $0.02$\\\hline
tcH~merged~signal & $3.19$ & $0.27$ & $0.19$ & $0.89$\\\hline
$\bar{t}t\to bWuH$ & $3.24$ & $0.29$ & $0.21$ & $0.22$\\\hline
$ug\to tH$ & $1.19$ & $0.09$ & $0.05$ & $0.06$\\\hline
tuH~merged~signal & $4.11$ & $0.37$ & $0.25$ & $0.29$\\\hline
\end{tabular}
\begin{tabular}{|c|c|c|c|c|} \hline
 & 1l2tau2bnj os  lowmet & 1l2tau2bnj ss  highmet & 1l2tau2bnj ss  lowmet & 1l1tau1b1j ss\\\hline
$\bar{t}t\to bWcH$ & $0.44$ & $0.03$ &  / & $1.33$\\\hline
$cg\to tH$ & $0.01$ & $0.00$ & $0.00$ & $0.05$\\\hline
tcH~merged~signal & $0.45$ & $0.03$ & $0.00$ & $1.37$\\\hline
$\bar{t}t\to bWuH$ & $0.10$ & $0.01$ &  / & $1.41$\\\hline
$ug\to tH$ & $0.02$ & $0.00$ &  / & $0.28$\\\hline
tuH~merged~signal & $0.12$ & $0.01$ &  / & $1.69$\\\hline
\end{tabular}
\begin{tabular}{|c|c|c|c|c|} \hline
 & 1l1tau1b2j os & 1l1tau1b2j ss & 1l1tau1b3j os & 1l1tau1b3j ss\\\hline
$\bar{t}t\to bWcH$ & $0.72$ & $1.28$ & $1.50$ & $0.59$\\\hline
$cg\to tH$ & $0.07$ & $0.04$ & $0.07$ & $0.01$\\\hline
tcH~merged~signal & $0.78$ & $1.32$ & $1.57$ & $0.60$\\\hline
$\bar{t}t\to bWuH$ & $0.74$ & $1.38$ & $1.61$ & $0.65$\\\hline
$ug\to tH$ & $0.40$ & $0.22$ & $0.38$ & $0.06$\\\hline
tuH~merged~signal & $1.11$ & $1.59$ & $1.98$ & $0.71$\\\hline
\end{tabular}
\begin{tabular}{|c|c|c|c|} \hline
 & 1l1tau2b1j ss & 1l1tau2b2j os & 1l1tau2b2j ss\\\hline
$\bar{t}t\to bWcH$ & $0.25$ & $0.18$ & $0.17$\\\hline
$cg\to tH$ & $0.00$ & $0.01$ & $0.00$\\\hline
tcH~merged~signal & $0.25$ & $0.19$ & $0.17$\\\hline
$\bar{t}t\to bWuH$ & $0.09$ & $0.08$ & $0.08$\\\hline
$ug\to tH$ & $0.02$ & $0.04$ & $0.02$\\\hline
tuH~merged~signal & $0.11$ & $0.11$ & $0.10$\\\hline
\end{tabular}
\begin{tabular}{|c|c|c|c|} \hline
 & 1l1tau2b3j os & 1l1tau2b3j ss & 1l2tau1bnj os\\\hline
$\bar{t}t\to bWcH$ & $0.26$ & $0.09$ & $6.34$\\\hline
$cg\to tH$ & $0.00$ & $0.00$ & $0.61$\\\hline
tcH~merged~signal & $0.27$ & $0.09$ & $6.77$\\\hline
$\bar{t}t\to bWuH$ & $0.13$ & $0.05$ & $6.74$\\\hline
$ug\to tH$ & $0.03$ & $0.01$ & $2.59$\\\hline
tuH~merged~signal & $0.16$ & $0.05$ & $8.64$\\\hline
\end{tabular}
\begin{tabular}{|c|c|c|c|} \hline
 & 1l2tau1bnj ss & 1l2tau2bnj os & 1l2tau2bnj ss\\\hline
$\bar{t}t\to bWcH$ & $0.30$ & $0.96$ & $0.03$\\\hline
$cg\to tH$ & $0.02$ & $0.02$ & $0.00$\\\hline
tcH~merged~signal & $0.31$ & $0.98$ & $0.03$\\\hline
$\bar{t}t\to bWuH$ & $0.35$ & $0.24$ & $0.01$\\\hline
$ug\to tH$ & $0.09$ & $0.06$ & $0.00$\\\hline
tuH~merged~signal & $0.43$ & $0.30$ & $0.01$\\\hline
\end{tabular}

\end{table}



