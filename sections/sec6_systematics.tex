%-------------------------------------------------------------------------------
\section{Systematic uncertainties}
\label{sec:systematics}
%-------------------------------------------------------------------------------
The signal efficiency and the background estimations are affected by uncertainties associated with the detector simulation, the signal modelling and the data-driven background determination. In the combined fit, these uncertainties recommended by the various ATLAS working groups are called Nuisance Parameters (NP), as opposed to the parameter of interest, the signal strength, which is a scaling factor applied on the total number of signal events.

Any systematic effect on the  overall normalisation or shape of the final BDT distribution in the signal region is considered. In \texttt{TRExFitter} \cite{TRExFitter}, the NP pruning is applied, which means that NPs whose impact are less than a certain threshold are discarded. The lower thresholds to remove a shape systematic and a normalisation systematic from the fit are both $1\%$ in the fit.

%\input{\FCNCTables/NPlist}
%Ztautau theory Xsec
%top theory Xsec

\subsection{Luminosity}
\label{sec:systematic_Luminosity}
The integrated luminosity measurement has an uncertainty of $1.7\%$ for the combined Run-2 data, and it is applied to all simulated event samples including both signal and background.

\subsection{Detector-related uncertainties}
\label{sec:syst_det}

Uncertainties related to the detector effects are included for the signal and backgrounds that are estimated using simulation. These uncertainties are also taken into account for the simulated events that enter the data-driven background estimations. All instrumental systematic uncertainties arising from the reconstruction, identification and energy scale of electrons, muons, taus,($b$-)jets and the soft term of the $\met$ measurement are considered. The effect of the energy scale uncertainties on the objects is propagated to the $\met$ calculation. These systematics include uncertainty associated with:

\begin{itemize}
\item The electron and muon trigger, reconstruction, identification and isolation efficiencies. These are estimated with the tag-and-probe method on the $Z\to ll$, $J/\psi\to ll$ and $W\to l\nu$ events \cite{lep_sys}. There are 5NP for electron(El\_ChargeMisID\_STAT, El\_ChargeMisID\_SYST, El\_Reco, El\_ID\_TightLH, El\_Iso\_FCLoose), 8NP for muon(Mu\_TTVA\_STAT, Mu\_TTVA\_SYST, Mu\_ID\_STAT, Mu\_ID\_SYST, Mu\_ID\_STAT\_LOWPT, Mu\_ID\_SYST\_LOWPT, Mu\_Iso\_SYST, Mu\_Iso\_STAT, Mu\_Iso\_STAT), 1NP for single electron trigger, and 2NP for single muon trigger. 
  We also checked the recommended PLIV scale factors can be applied to the electron and muon from
  tau decays using $Z\rightarrow \tau\tau\rightarrow e\mu$ enriched control sample in App.~\ref{sec:CheckPLIV} and
  additional $\pm 2\%$ systematic uncertainty is assigned for PLIV cut for the tau-lepton in the lepton + $\thad$ channels as ``Tau\_PLIV''.
  This mixture of lepton SFs (FCLoose, PLIV) could introduce issues in any fits that use both simultaneously. But the effect is likely small
  because the PLIV SFs is applied on top of leptons using FCLoose cuts and there is no additional systematic assigned to PLIV,
  except $\pm 2\%$ assigned for the tau-lepton.
  
\item Electron energy and muon momentum scales. They are estimated from the early 13 TeV $Z\to ll$ events. For electron energy scales, we used 1 NP for EG\_RESOLUTION\_ALL, and 2 NP for EG\_SCALE\_ALL and EG\_SCALE\_AF2. For muon momentum scales, 1NP MUON\_SCALE is used.
\item Jet energy scale (JES) and resolution (JER). The JES uncertainty is estimated by varying the jet energies according to the uncertainties derived from simulation and in-situ calibration measurements using a model with a reduced set of 43 orthogonal NPs (30 NPs for JES and 13 NPs for full JER) \cite{jet_sys} which has up to 30\% correlation losses, which are assumed to be uncorrelated, and the induced changes can be added in quadrature. The individual scale variations on the jets are parameterised in $\pt$ and $\eta$. The total JES uncertianty is below 5\% for most jets and below 1\% for central jets with pT between 300 GeV and 2 TeV.
The difference between the JER in data and MC is represented by one NP. It is applied on the MC by smearing the jet $\pt$ within the prescribed uncertainty.
JVT is applied in the analysis to select jets from hard-scattered vertices. It was found that different MC generators (and different fragmentation models) lead to efficiency differences of up to $1\%$, and the uncertainty on the efficiency measurement was found to be around $0.5\%$. Two NPs are assigned for the JVT efficiency, one for the central and the other for the forward jets.
\item Calibration of the $\met$. The uncertainties on $\met$ due to systematic shifts in the corrections for leptons and jets are accounted for in a fully correlated way in their evaluation for those physics objects, and are therefore not considered independently here. The systematic uncertainty assigned to the track-based soft term used in the $\met$ definition quantifies the resolution and scale of the soft term measurement by using the balance between hard and soft contributions in $Z\to\mu\mu$ events. These uncertainties are studied using the differences between Monte Carlo generators, using Powheg+Pythia8 as the nominal generator \cite{met_sys}. One NP is assigned for the soft-track scale, and two NPs for the soft-track resolution.
\item Jet flavour tagging systematics. The uncertainties on the $b$-tagging are assessed independently for $b$, $c$ and light-flavour quark jets\cite{btag_sys1}. The efficiencies are measured in data using the methods described in \cite{btag_sys2}-\cite{btag_sys3} with the 2015, 2016, 2017, and 2018 data set. There are 82 NPs assigned for the flavour tagging systematics with 19 NPs for light flavor, 19 for $c$, 44 for $b$.
\item Pileup. The uncertainty on the pileup reweighting is evaluated by varying the pileup scale factors by 1$\sigma$ based on the reweighting of the average interactions per bunch crossing. However, this uncertainty is highly correlated with the luminosity uncertainty and may be an overestimate.
\item Tau object systematics. These include the $\tauhad$ reconstruction, identification and trigger efficiencies, the efficiency for tau-electron overlap removal of true $\tauhad$ and true electrons faking $\tauhad$, and the efficiency for a ``medium'' BDT electron rejection. There are also three NPs that cover the tau energy scale (TES) systematics due to the modeling of the detector geometry (\texttt{TAU\_TES\_DETECTOR}), the measurement in the tag-and-probe analysis (\texttt{TAU\_TES\_INSITU}) and the \texttt{Geant4} shower model (\texttt{TAU\_TES\_MODEL}). They are evaluated based on detailed MC variation study, as well as the Run-2 $Z\to\tau\tau$ data for insitu calibrations of the tau TES and trigger efficiencies, as documented in \cite{tau_sys1} and the dedicated software tools \cite{tau_sys2} recommended by the Tau CP Woking Group \cite{TauCP}.
\end{itemize}

\subsection{Uncertainties on fake background estimations}
\label{sec:syst_datadriven}

Systematic uncertainties on the fake background as described in Sec. \ref{sec:background} are considered in the final fit.
The uncertainties of the fake estimation are correlated among all the leptonic channels and among all the hadronic channels,
respectively, but independent between them.

In the leptonic channels, they are named \texttt{fakeSFNP\_Xp\_ptY\_*} (X=1,3 indicating the number of tracks. Y=0,1,2 indicating the $\pt$ bins.) for tau fakes modelled by MC and \texttt{ABCD*} for QCD lepton fakes modelled by ABCD method.

In the hadronic channels, they are named \texttt{FFNP\_Xprong\_ptbinY\_etabinZ} (X=1,3 indicating the number of tracks. Y=0,1,2 indicating the $\pt$ bin. Z=0,1 indicating the central taus and forward taus.) for the statistical uncertainties of the FFs derived from the W+jets control region. \texttt{FFNP\_OS\_CR} and \texttt{FFNP\_SS\_CR} are one-sided NP for the systematic uncertainties of the FFs derived in the OS and SS control regions respectively. \texttt{``Only $\tau_\mathrm{sub}$ real modelling''} is the uncertainty of the MC modelling of the events with leading tau fake but sub-leading tau real, which is varied by 50\% to be conservative according to the study in the leptonic channels.

\subsection{Theoretical uncertainties on the background}

Several independent variations of modeling $t\bar{t}$ events are considered: The choice of the renormalisation and factorisation scale in the matrix-element calculation, the choice of the matching scale when matching the
matrix elements to the parton show generator, the uncertainty in the value of $\alpha_s$ when modeling intial-state radiation (ISR), the choice of the renormalisation scale when modeling final-state radiation (FSR) and
uncertainties associated with the choice of the PS generator.

%%Theoretical uncertainties have been applied to the MC background in this analysis. 
The NNPDF3.0 systematic set (which has 100 variations: PDFset=26001-26100) is used to get the variation envelope around the nominal PDF.

The $\alpha_s$ uncertainty is applied using weights PDFset=26600,26500. The impact is insignificant and pruned before the fit.

The renormalization and factorization scales are varied by a factor of 0.5 and 2.0 around the nominal values. There are eight such variations. In the final BDT distributions, the largest variations of the eight per bin are taken. The name of this theoretical NP is called ``scale'' in the final fit.

The uncertainty in the choice of the value of the strong coupling constant $\alpha_s$ when modeling ISR is determined by varying the parameter Var3c within its uncertainties given by the A14 tune. These variations are reflected by weights associated to the nominal sample of $t\bar{t}$ events. The impact of modeling FSR is evaluated based on PS weights which vary the renormalisation scale for QCD emission in the FSR by a factor of 0.5 (FSR down variation) and 2.0 (FSR up variation), respectively. They are obtained by weights:
\begin{itemize}
	\item FSR up: ``isr:muRfac=10\_fsr:muRfac=05''
	\item FSR down: ``isr:muRfac=10\_fsr:muRfac=05''
	\item ISR up: ``Var3cUp''
	\item ISR down: ``Var3cDown''
\end{itemize}

The theory uncertainty are applied with both shapes and normalisations hence no additional k-factor normalisation uncertainty is applied.

The default $t\bar{t}$ MC events (DSID=410470) are showered with Pythia8. A separate sample showered with Herwig7 (DSID=410557+410558, version 7.0.4) is compared with Pythia8 sample (DSID=410470), and the difference is treated as fragmentation and hadronization systematics \cite{ttbarSys}. These two samples are both simulated with ATLFAST-II \cite{AFII}, and their difference is then applied to the default full-simulation $t\bar{t}$ sample, shown as ``$\bar{t}t$ PS'' in the ranking plots, which are treated correlated cross different regions. The studies with decorrelated PS variation between regions are also
done and can be found in the appendix~\ref{sec:decor} and the impact on the limits is small. 

%The default $t\bar{t}$ sample is generated with full simulated Powheg. Separate AFII aMC (DSID=410464+410465, version 2.3.3) and Powheg (DSID=411288) samples are generated, both with Matrix Element Correction (MEC) set to off. The difference between those two is treated as the hard scattering systematics \cite{ttbarSys}, shown as ``ME'' in the ranking plots.

The \texttt{hdamp} parameter (which controls the amount of radiation produced by the parton shower in \texttt{POWHEG-BOX v2}) is set to $1.5 m_{\text{top}}$ in the nominal case. Alternative AFII samples are generated with \texttt{hdamp}$=3 m_{\text{top}}$. The difference between it and AFII Powheg Pythia8 sample (DSID=410470) is treated as one of the systematics, named ``$\bar{t}t$ hdamp''.
%The Powheg+Pythia8 $t\bar{t}$ MC is also generated with different shower radiations (initial and final-state radiation modelling). For a sample with increased radiation, the factorisation and renormalization scales are scaled by 0.5 with respect to their nominal values, the \texttt{hdamp} parameter (which controls the amount of radiation produced by the parton shower in \texttt{POWHEG-BOX v2}) is set to $3 m_{\text{top}}$ and the \texttt{A14var3cUp} tune is used. Conversely, for a sample with decreased radiation, the two scales are scaled by 2 with respect to their nominal values, the \texttt{hdamp} is kept at the nominal value of $1.5 m_{\text{top}}$ and the \texttt{A14var3cDown} tune is used \cite{ttbarSys}.

%Uncertainty affecting the normalisation of the $V$+jets and di-boson background is estimated to be about $5\%$ according to the PMG. The uncertainty on single top sross section is $+5\%$/$-4\%$ \cite{WtXsec}\cite{st_uncert_1,st_uncert_2}, on $t\bar{t}V$ $15\%$ \cite{ttV_uncert_1,ttV_uncert_2}, and on $t\bar{t}H$ $+10\%$/$-13\%$ \cite{ttH_uncert}.

Another significancant background stems from the ztautau samples, the following sources of uncertainty are considered for ztautau samples based on the Htautau analysis~\cite{Htautau}.
\begin{itemize}
	\item PDF central value: evaluated considering the standarnd deviation of the 100 NNPDF replicas event weights of NNPDF3.0nnlo PDF set used in Sherpa
	\item renormalisation and factorisation scales - $\mu_{R}/\mu_{F}$: evaluated using event-weights provided by Sherpa
	\item ckkw: jet-to-parton matching uncertainty, evaluated using truth-level parameterisation as a function of jet multiplicity and $p_{T}(Z)$
	\item qsf: resummation scale uncertainty, evaluated using truth-level parameterisation as a function of jet multiplicity and $p_{T}(Z)$
	\item $\alpha_{S}$:evaluated using event-weights provided by Sherpa
	\item PDF alternative value: evalatued comparing predictions from NNPDF3.0nnlo PDF set (nominal) with MMHT2014nnlo68cl and CT14nnlo PDF sets
\end{itemize}

Finally, the other MC samples,including single top, diboson, wjet, SM higgs and ttV, contribute a very small fraction of background. 

For single top sample,to estimate the uncertainty originating from ISR modelling in the single top background, the same procedure as for $t\bar{t}$ is used. One NP reflects the symmetrised effect of the two shower variations Var3cDown and Var3cUp, and the other NP with name ``scale'' describe the symmetrised effect from the independent variations of the renormalisation and factorisation scale. The impact of the uncertainty from FSR modelling is estimated by reweighting the nominal single top sample. The two variations $\mu^{FSR}_{R} \times 0.5$ and $\mu^{FSR}_{R} \times 2$ are considered.  The standard deviation of 100 NNPDF3.0nnlo variations is calculated Uncertainties on the PDF are evaluated following the recommended prescription in Ref.~\cite{ttRun2}.

For ttV samples, to obtain the uncertainties related to the renormalization and factorization scales, the values of the scale-parameters $\mu_{f}$ and $\mu_{f}$ are varied by factors of 2.0 and 0.5 with respect to their default values and compared with the nominal predictions. Uncertainties on the PDF are evaluated following the same procedure in Ref.~\cite{ttZRun2}.

For diboson background, it is not possible to define a dedicated control region to constrain its modeling. MC simulation is used to estimate the diboson background. The normalization uncertainty in diboson prediction is estimated using Sherpa samples, which were generated with up to one additional parton at NLO and up to three partons at LO. Following the recommendations of the PMG group, the uncertainties to be considered include~\cite{dibosonRes}: 
\begin{itemize}
	\item   7 Scale variations of the renormalization and factorization scale
	\item   100 NNPDF3.0nnlo variations (PDF261000-PDF261100)
	\item   4 NNPDF3.0nnlo $\alpha_{S}$ variations
\end{itemize}

Several sources of modeling uncertainties may affect the shape of the Wjets samples in the signal regions. Following the recommendations of the PMG group, the list of modeling uncertainties to be considered include~\cite{dibosonRes}:
\begin{itemize}
\item 6 Scale variations of the renormalization and factorization scale
\item 100 NNPDF3.0nnlo variations (PDF261000-PDF261100)
\item 2 NNPDF3.0nnlo al phas variations (PDF269000 and PDF270000)
\item 2 nominal PDF set variations with respect to MMHT and CT14
\end{itemize}

At last, A $+5.8\%-9.2\%$ normalization uncertainty is considered for the ttH background, corresponding to the scale and $\alpha_{S}$ uncertainties in the NLO cross-section computation. A constant PDF uncertainty of
$\pm3.6$\% is assumed, as it was done for the previous measurement.Therefore, a overall sytematics with variation 9\% is entered in the final fit model~\cite{ttZRun2}.

\subsection{Uncertainties on the signal modelling}

An additional $13-12\%$ uncertainty on $\mathcal{B}(H\to\tau\tau)$ (Named as HttBR in the Ranking plots.) is also assigned based on the latest measurment result \cite{HiggsBR} besides the theoretical uncertainties: PDF,ISR, FSR and scale.

In the leptonic channels, the fake tau calibration is also applied to the fake tau part of the signal the same way as the background and treated correlatedly.

The PS uncertainty of the signal is applied by comparing nominal samples (showered with pythia8) with the corresponding Herwig7 samples. The difference is treated as 1$\sigma$ variation.
