%-------------------------------------------------------------------------------
\section{Systematic uncertainties}
\label{sec:systematics}
%-------------------------------------------------------------------------------
The signal efficiency and the background estimations are affected by uncertainties associated with the detector simulation, the signal modelling and the data-driven background determination. In the combined fit, these uncertainties are called Nuisance Parameters (NP), as opposed to the parameter of interest, the signal strength, which is a scaling factor applied on the total signal events.

Any systematic effect on the the overall normalisation or shape of the final BDT distribution in the signal region is considered. In \texttt{TRExFitter} \cite{TRExFitter}, the NP pruning is applied, which means that NPs whose impact are less than a certain threshold are discarded. The lower thresholds to remove a shape systematic and a normalisation systematic from the fit are both $1\%$ in the fit.

Table \ref{tab:tab_norm_NP} gives the QCD fake estimation for $\thadhad$ channel in \ref{sec:ss_method} and 24 scale factor NPs and 2 transfer factor for fake method mentioned in \ref{sec:sf_method}.
The lists of systematic NPs that survive the pruning are in Table \ref{tab:tab_specific_NP} and \ref{tab:tab_common_NP}, and their meanings are given below. All the NPs in Table \ref{tab:tab_common_NP} and the fake method NP in \ref{tab:tab_norm_NP} are fully correlated in all signal regions.

%\input{\FCNCTables/NPlist}
%Ztautau theory Xsec
%top theory Xsec

\subsection{Luminosity (TBD)}
\label{sec:systematic_Luminosity}
The integrated luminosity measurement has an uncertainty of $1.7\%$ for the combined Run-2 data, and it is applied to all simulated event samples.

\subsection{Detector-related uncertainties (TBD)}
\label{sec:syst_det}

Uncertainties related to the detector are included for the signal and backgrounds that are estimated using simulation. These uncertainties are also taken into account for the simulated events that enter the data-driven background estimations. All instrumental systematic uncertainties arising from the reconstruction, identification and energy scale of electrons, muons, ($b$-)jets and the soft term of the $\met$ measurement are considered. The effect of the energy scale uncertainties on the objects is propagated to the $\met$ calculation. These systematics include uncertainty associated with:

\begin{itemize}
\item The electron and muon trigger, reconstruction, identification and isolation efficiencies. These are estimated with the tag-and-probe method on the $Z\to ll$, $J/\psi\to ll$ and $W\to l\nu$ events \cite{lep_sys}.
\item Electron and muon momentum scales. They are estimated from the early 13 TeV $Z\to ll$ events.
\item Jet energy scale (JES) and resolution (JER). The JES uncertainty is estimated by varying the jet energies according to the uncertainties derived from simulation and in-situ calibration measurements using a model with a reduced set of 38 orthogonal NPs \cite{jet_sys} which has up to 30\% correlation losses, which are assumed to be uncorrelated, and the induced changes can be added in quadrature. The individual scale variations on the jets are parameterised in $\pt$ and $\eta$. The total JES uncertianty is below 5\% for most jets and below 1\% for central jets with pT between 300 GeV and 2 TeV.
The difference between the JER in data and MC is represented by one NP. It is applied on the MC by smearing the jet $\pt$ within the prescribed uncertainty.
JVT is applied in the analysis to select jets from hard-scattered vertices. It was found that different MC generators (and different fragmentation models) lead to efficiency differences of up to $1\%$, and the uncertainty on the efficiency measurement was found to be around $0.5\%$. Two NPs are assigned for the JVT efficiency, one for the central and the other for the forward jets.
\item Calibration of the $\met$. The uncertainties on $\met$ due to systematic shifts in the corrections for leptons and jets are accounted for in a fully correlated way in their evaluation for those physics objects, and are therefore not considered independently here. The systematic uncertainty assigned to the track-based soft term used in the $\met$ definition quantifies the resolution and scale of the soft term measurement by using the balance between hard and soft contributions in $Z\to\mu\mu$ events. The uncertainties are studied using the differences between Monte Carlo generators, using Powheg+Pythia8 as the nominal generator \cite{met_sys}. One NP is assigned for the soft-track scale, and two NPs for the soft-track resolution.
\item Jet flavour tagging systematics. The uncertainties on the $b$-tagging are assessed independently for $b$, $c$ and light-flavour quark jets, with extrapolation factors \cite{btag_sys1}. The efficiencies and mis-tag rates are measured in data using the methods described in \cite{btag_sys2}-\cite{btag_sys3} with the 2015, 2016 and 2017 data set. There are 19 NPs assigned for the flavour tagging systematics (so-called ``Loose'' reduced set, with 5 NPs for light flavor, 4 for $c$, 9 for $b$, and 1 for extrapolation).
\item Pileup. The uncertainty on the pileup reweighting is evaluated by varying the pileup scale factors by 1$\sigma$ based on the reweighting of the average interactions per bunch crossing. However, this uncertainty is highly correlated with the luminosity uncertainty and may be an overestimate.
\item Tau object systematics. These include the $\tauhad$ reconstruction, identification and trigger efficiencies, the efficiency for tau-electron overlap removal of true $\tauhad$, the one for tau-electron overlap removal of true electrons faking $\tauhad$, and the one for a ``medium'' BDT electron rejection. There are also three NPs that cover the tau energy scale (TES) systematics due to the modeling of the detector geometry (\texttt{TAU\_TES\_DETECTOR}), the measurement in the tag-and-probe analysis (\texttt{TAU\_TES\_INSITU}) and the \texttt{Geant4} shower model (\texttt{TAU\_TES\_MODEL}). The systematics are based on detailed MC variation study, as well as the Run-2 $Z\to\tau\tau$ data for insitu calibrations of the tau TES and trigger efficiencies, as documented in \cite{tau_sys1} and the dedicated software tools \cite{tau_sys2} recommended by the Tau CP Woking Group \cite{TauCP}.
\end{itemize}

\subsection{Uncertainties on fake background estimations}
\label{sec:syst_datadriven}

Systematic uncertainties as described in Sec. \ref{sec:background} are considered in the final fit. The uncertainties of the fake estimation for the leptonic channels and hadronic channels are correlated respectively but independent between them.

In the leptonic channels, they are named \texttt{fakeSFNP_Xprong_ptbinX_*} for tau fakings modelled by MC and \texttt{ABCD*} for QCD faking modelled by DD.

In the hadronic channels, they are named \texttt{FFNP_Xprong_ptbinX_etabin*} for the statistical uncertainties of the FFs derived from the W+jets control region. \texttt{FFNP_OS_CR} and \texttt{FFNP_SS_CR} are for the systematic uncertainties of the FFs derived in the OS and SS control region respectively. \texttt{only $\tau_\mathrm{sub}$ real modelling} is the uncertainty of the MC modelling of the events with leading tau fake but sub-leading tau real, which is varied by 50\% to be conservative according to the study in the leptonic channels.

\subsection{Theoretical uncertainties on the background (TBD)}

Theoretical uncertainties have been applied to the MC background in this analysis. The NNPDF3.0 systematic set (which has 100 variations) is used to get the variation envelope around the nominal PDF, and the renormalization and factorization scales are varied by a factor of 0.5 and 2.0 around the nominal values. There are eight such variations. In the final BDT distributions, the largest variations of the eight per bin are taken.

The default $t\bar{t}$ sample is generated with Powheg. A separate full-sim $t\bar{t}$ sample generated with Sherpa (0 and 1-jet at NLO, and $\ge$2 jets at LO) is compared with the Powheg sample, and the difference in final results is treated as the hard scattering systematics \cite{ttbarSys}.

The default $t\bar{t}$ MC events are showered with Pythia8. A separate sample showered with Herwig7 is compared with the Pythia8 sample, and the difference is treated as fragmentation and hadronization systematics \cite{ttbarSys}. These two samples are both generated with ATLFAST-II \cite{AFII}, and their difference is then applied to the default full-simulation $t\bar{t}$ sample.

The Powheg+Pythia8 $t\bar{t}$ MC is also generated with different shower radiations (initial and final-state radiation modelling). For a sample with increased radiation, the factorisation and renormalization scales are scaled by 0.5 with respect to their nominal values, the \texttt{hdamp} parameter (which controls the amount of radiation produced by the parton shower in \texttt{POWHEG-BOX v2}) is set to $3 m_{\text{top}}$ and the \texttt{A14var3cUp} tune is used. Conversely, for a sample with decreased radiation, the two scales are scaled by 2 with respect to their nominal values, the \texttt{hdamp} is kept at the nominal value of $1.5 m_{\text{top}}$ and the \texttt{A14var3cDown} tune is used \cite{ttbarSys}.

Uncertainty affecting the normalisation of the $V$+jets background is estimated to be about $30\%$ according to the study done in the FCNC $H\to b\bar{b}$ channel \cite{FCNC_Hbb}. The uncertainty on the diboson cross section is $5\%$ \cite{VV_uncert}, on single top $+5\%$/$-4\%$ \cite{WtXsec}\cite{st_uncert_1,st_uncert_2}, on $t\bar{t}V$ $15\%$ \cite{ttV_uncert_1,ttV_uncert_2}, and on $t\bar{t}H$ $+10\%$/$-13\%$ \cite{ttH_uncert}.

\subsection{Uncertainties on the signal modelling (TBD)}

Since the TT signal samples share the same production as the $\ttbar$ process, the systematics listed above for $\ttbar$ also apply to the signal. However, because the systematics variation samples are only generated for the SM decays of $\ttbar$, only the integral change of the yields observed for the $t\bar{t}$ background with real taus in the FR is used, and applied on the signal in the same region in a fully correlated way. An additional $1.6\%$ uncertainty on BR($H\to\tau\tau$) is also assigned \cite{HiggsBR}.

The fake calibration is also applied to the fake tau part of the signal the same way as the background. The 6 NPs are also applied to the signal and fully correlated with the background.