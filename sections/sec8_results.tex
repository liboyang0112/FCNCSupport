\section{Results}
\label{sec:results}

The significance of any small observed excess in data is evaluated by quoting the $p$-values to quantify the level of consistency of the data with the BR=0 hypothesis. The asymptotic approximation in \cite{CCGV} is used. The test statistic used for the exclusion limits derivation is the $\tilde{q}_\mu$ test statistic
and for the $p$-values the $q_{0}$ test statistic\footnote{The definition of the test statistics used in this search is the following:
\[ \tilde q_\mu = \left\{
  \begin{array}{l l}
    -2 \ln (\mathcal{L}(\mu,\hat{\hat{\theta}})/\mathcal{L}(0,\hat{\hat\theta})) & \quad \text{if $\hat\mu < 0$}\\
    -2 \ln (\mathcal{L}(\mu,\hat{\hat{\theta}})/\mathcal{L}(\hat \mu,\hat\theta)) & \quad \text{if $0 \leq \hat\mu \leq \mu$}\\
    0 & \quad \text{if $\hat\mu > \mu$}
  \end{array} \right.\]
and
\[ q_0 = \left\{
  \begin{array}{l l}
    -2 \ln (\mathcal{L}(0,\hat{\hat{\theta}})/\mathcal{L}(\hat\mu,\hat\theta)) &\quad \text{if $\hat\mu \geq 0$}\\
    0 & \quad \text{if $\hat\mu <0$}
  \end{array} \right.\]
where $\mathcal L(\mu,\theta)$ denotes the binned likelihood function, $\mu$ is the parameter of interest (i.e.
the signal strength parameter), and $\theta$ denotes the nuisance parameters. The pair $(\hat\mu, \hat\theta)$
corresponds to the global maximum of the likelihood, whereas $(x, \hat{\hat\theta})$ corresponds to a conditional
maximum in which $\mu$ is fixed to a given value $x$.
}
\cite{CCGV}.

The $95\%$ CL upper limits on tqH interaction with BR$(t\to Hq)=0.2\%$ as reference are given in Table~\ref{tab:tab_limit}. The best asimov fit values with S+B hypothesis are given in Table~\ref{tab:tab_bestfit}.

\begin{table}[htb]
\caption{ The expected $95\%$ CL exclusion upper limits on BR$(t\to Hc)$ and BR$(t\to Hu)$ (0.2\%) with the Asimov (B-only).}
\input{\FCNCTables/tthML/limits}
\label{tab:tab_limit}
\end{table}

\begin{table}[htb]
\caption{ The best asimov fit values with S+B hypothesis. }
\centering
\begin{tabular}{|c|c|c|} \hline
  & $tcH$ & $tuH$ \\ \hline
  $\thadhad$ & $1.00^{+0.23 +0.51}_{-0.22 -0.38}$ & $1.00^{+0.18 +0.44}_{-0.18 -0.33}$ \\ \hline
  leptonic channels & $1.00^{+0.56 +X.XX}_{-0.54 -X.XX}$ & $1.00^{+0.47 +X.XX}_{-0.46 -X.XX}$ \\ \hline

\end{tabular}
\label{tab:tab_limit}
\end{table}

%\begin{table}[htb]
%\caption{ The expected and observed $95\%$ CL exclusion upper limits on BR$(t\to Hq)$ (in \%) with the Asimov (B-only) and real data (with equal mixture if $t\to Hc$ and $t\to Hu$ assumed). }
%\centering
%\begin{tabular}{|c|c|c|c|} \hline
%  & $\tlhad$ & $\thadhad$ & combined \\ \hline
%  Expected & \multirow{2}{*}{$0.45^{+0.18}_{-0.13}$} & \multirow{2}{*}{$0.25^{+0.10}_{-0.07}$} & \multirow{2}{*}{$0.21^{+0.08}_{-0.06}$} \\
%  (stat-only) & & & \\ \hline
%  Expected & $0.57^{+0.22}_{-0.16}$ & $0.30^{+0.12}_{-0.08}$ & $0.25^{+0.10}_{-0.07}$ \\ \hline
%  Observed & $X.XX^{+X.XX}_{-X.XX}$ & $X.XX^{+X.XX}_{-X.XX}$ & $X.XX^{+X.XX}_{-X.XX}$ \\ \hline
%\end{tabular}
%\label{tab:tab_limit}
%\end{table}

%\begin{table}[htb]
%\caption{ The expected Asimov significances for the $t\to Hc$ and $t\to Hu$ signal with $1\%$ BR assumed, and the significances from real data. The significance for $t\to Hq$ with equal mixture if $t\to Hc$ %and $t\to Hu$ is also given. }
%\centering
%\begin{tabular}{|c|c|c|c|} \hline
%  & $t\to Hq$ & $t\to Hc$ & $t\to Hu$ \\ \hline
%  Expected & 4.69 & 4.62 & 4.73 \\ \hline
%  Observed & -0.52 & -0.44 & -0.58 \\ \hline
%\end{tabular}
%\label{tab:tab_significance}
%\end{table}
%
%\section{Conclusions}
%\label{sed:conclusions}

The search for the FCNC decay $t\to Hq, H\to\tau\tau$ with the ATLAS detector at the LHC using 13~\TeV{} data was presented in this note. The best-fit values for BR($t\to Hc$) and BR($t\to Hu$) are found to be $-X.XX^{+X.XX}_{-X.XX}\%$ and $-X.XX^{+X.XX}_{-X.XX}\%$ respectively, based on 140 $\ifb$ of data collected from 2015 to 2018. The observed (expected) $95\%$ CL upper limits on BR($t\to Hc$) and BR($t\to Hu$) are found to be $X.XX\%$ ($X.XX^{+X.XX}_{-X.XX}\%$) and $X.XX\%$ ($X.XX^{+X.XX}_{-X.XX}\%$), respectively.
